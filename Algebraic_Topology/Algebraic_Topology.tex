% Options for packages loaded elsewhere
\PassOptionsToPackage{unicode}{hyperref}
\PassOptionsToPackage{hyphens}{url}
\PassOptionsToPackage{dvipsnames,svgnames*,x11names*}{xcolor}
%
\documentclass[]{article}
\usepackage{lmodern}
\usepackage{amssymb,amsmath}
\usepackage{ifxetex,ifluatex}
\ifnum 0\ifxetex 1\fi\ifluatex 1\fi=0 % if pdftex
  \usepackage[T1]{fontenc}
  \usepackage[utf8]{inputenc}
  \usepackage{textcomp} % provide euro and other symbols
\else % if luatex or xetex
  \usepackage{unicode-math}
  \defaultfontfeatures{Scale=MatchLowercase}
  \defaultfontfeatures[\rmfamily]{Ligatures=TeX,Scale=1}
\fi
% Use upquote if available, for straight quotes in verbatim environments
\IfFileExists{upquote.sty}{\usepackage{upquote}}{}
\IfFileExists{microtype.sty}{% use microtype if available
  \usepackage[]{microtype}
  \UseMicrotypeSet[protrusion]{basicmath} % disable protrusion for tt fonts
}{}
\makeatletter
\@ifundefined{KOMAClassName}{% if non-KOMA class
  \IfFileExists{parskip.sty}{%
    \usepackage{parskip}
  }{% else
    \setlength{\parindent}{0pt}
    \setlength{\parskip}{6pt plus 2pt minus 1pt}}
}{% if KOMA class
  \KOMAoptions{parskip=half}}
\makeatother
\usepackage{xcolor}\pagecolor[RGB]{28,30,38} \color[RGB]{213,216,218}
\IfFileExists{xurl.sty}{\usepackage{xurl}}{} % add URL line breaks if available
\IfFileExists{bookmark.sty}{\usepackage{bookmark}}{\usepackage{hyperref}}
\hypersetup{
  pdftitle={Algebraic Topology},
  pdfauthor={Kexing Ying},
  colorlinks=true,
  linkcolor=Maroon,
  filecolor=Maroon,
  citecolor=Blue,
  urlcolor=red,
  pdfcreator={LaTeX via pandoc}}
\urlstyle{same} % disable monospaced font for URLs
\usepackage[margin = 1.5in]{geometry}
\usepackage{graphicx}
\makeatletter
\def\maxwidth{\ifdim\Gin@nat@width>\linewidth\linewidth\else\Gin@nat@width\fi}
\def\maxheight{\ifdim\Gin@nat@height>\textheight\textheight\else\Gin@nat@height\fi}
\makeatother
% Scale images if necessary, so that they will not overflow the page
% margins by default, and it is still possible to overwrite the defaults
% using explicit options in \includegraphics[width, height, ...]{}
\setkeys{Gin}{width=\maxwidth,height=\maxheight,keepaspectratio}
% Set default figure placement to htbp
\makeatletter
\def\fps@figure{htbp}
\makeatother
\setlength{\emergencystretch}{3em} % prevent overfull lines
\providecommand{\tightlist}{%
  \setlength{\itemsep}{0pt}\setlength{\parskip}{0pt}}
\setcounter{secnumdepth}{5}
\usepackage{tikz}
\usepackage{physics}
\usepackage{amsthm}
\usepackage{mathtools}
\usepackage{esint}
\usepackage[ruled,vlined]{algorithm2e}
\usepackage{tikz-cd}
\theoremstyle{definition}
\newtheorem{theorem}{Theorem}
\newtheorem{definition*}{Definition}
\newtheorem{prop}{Proposition}
\newtheorem{corollary}{Corollary}[theorem]
\newtheorem*{remark}{Remark}
\theoremstyle{definition}
\newtheorem{definition}{Definition}[section]
\newtheorem{lemma}{Lemma}[section]
\newtheorem{proposition}{Proposition}[section]
\newtheorem{example}{Example}[section]
\newcommand{\diag}{\mathop{\mathrm{diag}}}
\newcommand{\Arg}{\mathop{\mathrm{Arg}}}
\newcommand{\hess}{\mathop{\mathrm{Hess}}}

% `calc` is necessary to draw curved arrows.
\usetikzlibrary{calc}
% `pathmorphing` is necessary to draw squiggly arrows.
\usetikzlibrary{decorations.pathmorphing}
\tikzset{curve/.style={settings={#1},to path={(\tikztostart)
    .. controls ($(\tikztostart)!\pv{pos}!(\tikztotarget)!\pv{height}!270:(\tikztotarget)$)
    and ($(\tikztostart)!1-\pv{pos}!(\tikztotarget)!\pv{height}!270:(\tikztotarget)$)
    .. (\tikztotarget)\tikztonodes}},
    settings/.code={\tikzset{quiver/.cd,#1}
        \def\pv##1{\pgfkeysvalueof{/tikz/quiver/##1}}},
    quiver/.cd,pos/.initial=0.35,height/.initial=0}

% TikZ arrowhead/tail styles.
\tikzset{tail reversed/.code={\pgfsetarrowsstart{tikzcd to}}}
\tikzset{2tail/.code={\pgfsetarrowsstart{Implies[reversed]}}}
\tikzset{2tail reversed/.code={\pgfsetarrowsstart{Implies}}}
% TikZ arrow styles.
\tikzset{no body/.style={/tikz/dash pattern=on 0 off 1mm}}


\title{Algebraic Topology}
\author{Kexing Ying}

\begin{document}
\maketitle

{
\hypersetup{linkcolor=}
\setcounter{tocdepth}{2}
\tableofcontents
}
\newpage

\section{Introduction}

Let us introduce/recall some basic definitions which will be used throughout this 
course. 

\begin{definition}[Path]
  A path in a topological space \(X\) is a continuous map \(\gamma : [0, 1] \subseteq \mathbb{R} \to X\).
  In the case that \(\gamma(0) = \gamma(1)\), we call \(\gamma\) a loop/closed path.
\end{definition}

\begin{definition}[Homotopy]
  Given two paths \(\gamma_0, \gamma_1 : [0, 1] \to X\) with the same end points 
  (i.e. \(\gamma_0(0) = \gamma_1(0)\) and \(\gamma_0(1) = \gamma_1(1)\) are 
  said to be homotopic with fixed endpoints if there exists a continuous map 
  \[H : [0, 1] \times [0, 1] \to X\]
  such that 
  \begin{itemize}
    \item \(H(t, 0) = \gamma_0(t)\) for all \(t \in [0, 1]\);
    \item \(H(t, 1) = \gamma_1(t)\) for all \(t \in [0, 1]\);
    \item for all \(u \in [0, 1]\), 
    \(H(0, u) = \gamma_0(0) = \gamma_1(0)\) and \(H(1, u) = \gamma_0(1) = \gamma_1(1)\).
  \end{itemize}
\end{definition}

Thus, graphically, two paths are homotopic if you can continuously deform a path 
into the other without moving the starting and ending points (see second year 
complex analysis for more details).

\begin{definition}[Free Homotopy]
  The loops \(\gamma_0, \gamma_1\) in \(X\) is said to be freely homotopic if 
  there exists a continuous \(H : [0, 1] \times [0, 1] \to X\) such that 
  \begin{itemize}
    \item \(H(t, 0) = \gamma_0(t)\) for all \(t \in [0, 1]\);
    \item \(H(t, 1) = \gamma_1(t)\) for all \(t \in [0, 1]\);
    \item for all \(u \in [0, 1]\), \(H(0, u) = H(1, u)\).
  \end{itemize}
\end{definition}

\begin{definition}[Simply Connected]
  \(X\) is said to be simply connected if any loop in \(X\) is freely homotopic 
  to a constant loop. 
\end{definition}

Thus, informally, in a simply connected space, any loop can be contracted into 
a single point.

\begin{proposition}
  \(S^2\) is simply connected.
\end{proposition}

Simply connectedness is a important notion and relates to many difficult problems 
in geometry. 

\begin{theorem}
  \(S^2\) and \(\mathbb{R}^2\) are, up to homeomorphism, the only two simply-connected 
  2-dimensional manifolds.
\end{theorem}

\begin{theorem}[Poincaré Conjecture]
  The only compact, simply connected 3-dimensional manifold is the sphere \(S^3\) 
  (up to homeomorphism).
\end{theorem}

\subsection{The Torus}

An important example in algebraic topology is the torus. We will now provide a 
proof that the torus is not simply connected.

\begin{definition}[Torus]
  The 2-torus \(T^2\) is the product topological space \(S^1 \times S^1\).
\end{definition}

We will now provide an alternative method of constructing the torus.
Define the homeomorphisms
\[T_1 : \mathbb{R}^2 \to \mathbb{R}^2 : (x_1, x_2) \mapsto (x_1 + 1, x_2); 
  T_2 : \mathbb{R}^2 \to \mathbb{R}^2 : (x_1, x_2) \mapsto (x_1, x_2 + 1).\]
It is clear that \(T_1\) and \(T_2\) commutes and the map 
\[\psi : \mathbb{Z}^2 \to \text{Homeo}(\mathbb{R}^2) : (n, m) \mapsto T_1^n \circ T_2^m\]
is a group homomorphism. As this map is injective, we have in some sense embedded 
\(\mathbb{Z}\) inside of \(\text{Homeo}(\mathbb{R}^2)\). Now, defining the equivalence 
relation on \(\mathbb{R}^2\) by 
\[(x_1, x_2) \sim (y_1, y_2) \iff \exists (n, m) \in \mathbb{Z}^2, \psi(n, m)(x_1, x_2) = (y_1, y_2),\]
or equivalently, there exists \((n, m) \in \mathbb{Z}^2\) such that \((x_1 + n, x_2 + m) = (y_1, y_2)\), 
we define the quotient topology \(X := \mathbb{R}^2 / \sim\). 

\begin{lemma}
  Let \(X\) and \(Y\) be topological spaces and \(\sim\) be an equivalence relation 
  on \(X\). Then, if \(p : X \to X / \sim : x \mapsto [x]_{\sim}\) is the quotient map, 
  any map \(f : X / \sim \to Y\) is continuous if and only if \(f \circ p\) is 
  continuous.
\end{lemma}
\begin{proof}
  Exercise.
\end{proof}

We see that the above lemma together with the universal property for quotients provides 
the universal property for topological spaces. Namely, if \(f : X \to Y\) is a continuous 
map and for all \(x \sim y \in X\), \(f(x) = f(y)\), then the unique map \(\tilde f\) 
obtained such that \(\tilde f \circ p = f\) is continuous.

\begin{lemma}
  Let \(X\) be a compact space and \(Y\) Hausdorff. Then, any continuous bijective 
  map \(f : X \to Y\) is a homeomorphism.
\end{lemma}
\begin{proof}
  Exercise.
\end{proof}

\begin{proposition}
  \(X\) is homeomorphic to the torus \(T^2\).
\end{proposition}
\begin{proof}
  Define the map \(\pi : \mathbb{R}^2 \to S^1 \times S^1\) such that 
  \(\pi(x, y) = (e^{2 \pi i x}, e^{2 \pi i y})\) for all \((x, y) \in \mathbb{R}^2\).
  We observe that \(\pi(x, y) = \pi(u, v)\) if and only if \((x, y) \sim (u, v)\),
  and so, by the universal property for topological spaces, we obtain the unique 
  continuous map defined by 
  \[\tilde \pi : X \to S^1 \times S^1 : [(x, y)] \mapsto \pi(x, y).\]
  This map is clearly bijective and continuous by the universal property, and 
  thus, by the above lemma, it suffices to show \(X\) is compact. But, this 
  is clear since the \(p([0, 1]^2) = \mathbb{R}^2 / \sim\) and the continuous 
  image of a compact set is compact.
\end{proof}

While the map \(\pi\) as described above is not injective, it is locally so (exercise).
Thus, given a path on the torus, we may think of lifting it to \(\mathbb{R}^2\) by 
lifting the paths piecewise via. the local homeomorphisms induced by \(\pi\). 

\begin{lemma}[Pasting Lemma]
  Let \(X, Y\) be both open subsets of a topological space and let \(B\) be another topological space, 
  then \(f : X \cup Y \to B\) is continuous if and only if \(f \mid_A\) and \(f\mid_B\) are continuous.
\end{lemma}
\begin{proof}
  Exercise.
\end{proof}

\begin{corollary}
  If \(f_1 : X \to B, f_2 Y \to B\) are continuous and agree on \(X \cap Y\), then 
  the map 
  \[f : X \cup Y \to B : x \mapsto \begin{cases}
    f_1(x) & \text{if } x \in X \\
    f_2(x) & \text{otherwise }
  \end{cases}\]
  is continuous.
\end{corollary}

\begin{proposition}[Lifting of Paths]
  For any \(\gamma : [0, 1] \to T^2\) a path, \(\tilde x \in \mathbb{R}^2\) such that 
  \(\pi(\tilde x) =: x = \gamma(0)\), there exists a unique path 
  \(\tilde \gamma : [0, 1] \to \mathbb{R}^2\) such that \(\gamma = \pi \circ \tilde \gamma\) 
  and \(\tilde \gamma(0) = \tilde x\).
\end{proposition}
\begin{proof}
  As \(\pi\) restricts to local homeomorphisms, we obtain an open cover of the path. 
  Invoking compactness, we obtain a finite subcover for which we may lift the path 
  locally such that the paths are compatible on intersections. With this, we 
  obtain the required path by the pasting lemma.

  For uniqueness, we observe that if \(\pi \circ \tilde \gamma_1 = \pi \circ \tilde \gamma_2\),
  then, for all \(t, \pi(\tilde \gamma_1(t)) = \pi(\tilde \gamma_2(t))\) and hence, 
  \(\tilde \gamma_1(t) \sim \tilde \gamma_2(t)\). Thus, the map 
  \[\delta := \tilde \gamma_1 - \tilde \gamma_2\]
  must take value in \(\mathbb{Z}^2\), and so, is a constant as continuous maps are 
  constant on connected components. Now, as \(\tilde \gamma_1(0) = \tilde x = \tilde \gamma_2(0)\) 
  and so, \(\delta = 0\) and \(\tilde \gamma_1 = \tilde \gamma_2\).
\end{proof}

\begin{proposition}[Free Homotopy Classes of Loops on \(T^2\)]
  Given a loop \(\gamma : [0, 1] \to T^2\) and its lift onto \(\mathbb{R}^2\) 
  the number 
  \[\rho(\gamma) := \tilde \gamma(1) - \tilde \gamma(0) \in \mathbb{Z}^2\]
  is well-defined and for all \(\gamma_1\) freely homotopic to \(\gamma_2\),
  \(\rho(\gamma_1) = \rho(\gamma_2)\).
\end{proposition}
\begin{proof}
  Since for a loop, \(\gamma(0) = \gamma(1)\), 
  \(\tilde \gamma(0) - \tilde \gamma(1) \in \mathbb{Z}^2\) is well-defined 
  since two lifts differ only by a constant.

  Suppose now \(\gamma_1, \gamma_2\) are two freely homotopic loops on the torus, 
  i.e. there exists some continuous map \(H : [0, 1] \times [0, 1] \to T^2\) 
  such that 
  \[H(0, \cdot) = \gamma_0; H(1, \cdot) = \gamma_1\]
  and for all \(u \in [0, 1]\), the map \(t \mapsto H(u, t)\) is closed. Let 
  \(\tilde x_0 \in \pi^{-1}(\gamma_0(0))\) and consider the map 
  \[\delta : [0, 1] \to T^2 : t \mapsto H(t, 0),\]
  (i.e. the path of base points of the free homotopy) let \(\tilde \delta\) to 
  be the lift of \(\delta\) starting at \(\tilde x_0\). Now, define 
  \(\tilde \gamma_t\) to be the lift of \(u \mapsto H(t, u) =: \gamma_t\) based at \(\tilde \delta(t)\), 
  I claim, \(\tilde H : [0, 1]^2 \to \mathbb{R}^2 : (t, u) \mapsto \tilde \gamma_t(u)\) 
  is a free homotopy from \(\tilde \gamma_0\) to \(\tilde \gamma_1\). Thus, 
  as \(t \mapsto \rho(\gamma_t) = \tilde \gamma_t(1) - \tilde \gamma_t(0)\) is continuous 
  and take value in \(\mathbb{Z}^2\), it must be a constant, concluding the proof.
\end{proof}

Suppose now we denote \(P := \{\text{loops on }T^2\}\) and \(\sim\) the freely homotopic 
equivalence relation on \(P\), we have the following proposition.

\begin{proposition}
  The map \(\rho : L := P / \sim \to \mathbb{Z}^2 : [\gamma] \mapsto \rho(\gamma)\)
  is a bijection.
\end{proposition}
\begin{proof}
  Surjectivity follows by considering the loop \(\gamma : t \mapsto \pi(tu, tm)\) for all 
  \((n, m) \in \mathbb{Z}^2\). Then, \(\rho(\gamma) = (n, m)\) and hence the map 
  is surjective.

  Suppose on the other hand \(\gamma_0, \gamma_1\) are loops on the torus such that 
  \(\rho(\gamma_0) = \rho(\gamma_1)\), injectivity follows by showing the loops are 
  freely homotopic. Let \(\tilde \gamma_0\) and \(\tilde \gamma_1\) be lifts of 
  \(\gamma_0\) and \(\gamma_1\) respectively. Then, define the homotopy from 
  \(\gamma_0\) to \(\gamma_1\) by
  \[\tilde H(t, u) := (1 - t)\tilde \gamma_0(u) + t \tilde \gamma_1,\]
  and define \(H = \pi \circ \tilde H\). For all \(t\), \(H(t, \cdot)\) is a 
  closed path since 
  \[\tilde H(t, 1) - \tilde H(t, 0) = (1 - t)(\tilde \gamma_0(1) - \tilde \gamma_0(0))
    + t(\tilde \gamma_1(1) - \tilde \gamma_1(0)) = (1 - t)\rho(\gamma_0) + t(\rho(\gamma_1)).\]
  By assumption, we have \(\rho(\gamma_0) = \rho(\gamma_1)\) and so, 
  \[\tilde H(t, 1) - \tilde H(t, 0) = \rho(\gamma_0) \in \mathbb{Z}^2\]
  implying that the loop is closed. Hence, \(H\) is a free homotopy between 
  \(\gamma_1\) and \(\gamma_2\) as required.
\end{proof}

\begin{theorem}
  The torus \(T^2\) is not simply connected.
\end{theorem}
\begin{proof}
  If the loop \(\gamma\) on the torus is freely homotopic to the constant path, 
  \(\rho(\gamma) = \rho(c) = 0\). But we have provided examples where this is not the case, 
  and hence, not all loops are freely homotopic to a constant path.
\end{proof}

This procedure for proving a space is not simply connected will is common. In particular, 
we will provide a proposition for situations where we have a space \(X\), a simply 
connected space \(\tilde X\) and a group \(\Gamma\) 
characterising the lack of simply connectedness of \(X\).

\newpage
\section{Fundamental Group}

\subsection{Definition}

Given a topological space \(X\), we will consider the set of all loops on 
\(X\) quotiented by the homotopy relation. Then, by equipping this quotient with 
the operation of gluing paths together, we obtain a group on this quotient. This 
group is known as the Fundamental group.

\begin{definition}[Concatenation of Paths]
  Let \(\gamma_1, \gamma_2 : [0, 1] \to X\) be paths such that \(\gamma_1(1) = \gamma_2(0)\), 
  then, we define the path 
  \[\gamma_1 * \gamma_2 : [0, 1] \to X : t \mapsto 
  \begin{cases}
    \gamma_1(2t), \ & t \le 1 / 2,\\
    \gamma_2(2t - 1), \ & t > 1 / 2. 
  \end{cases}\] 
  It is not difficult to see that this is continuous and hence is a path.
\end{definition}

\begin{definition}
  We define the equivalence relation \(\sim\) on the space of paths with the 
  same end points such that for \(\gamma_1, \gamma_2 : [0, 1] \to X\) with 
  the same end points, \(\gamma_1 \sim \gamma_2\) if and only if 
  \(\gamma_1\) is homotopic to \(\gamma_2\).
\end{definition}

\begin{definition}[Fundamental Group]
  Let \(X\) be a path-connected space and let \(x_0 \in X\). Then, we define 
  the fundamental group as 
  \[\pi_1(X, x_0) := \{\text{loops at \(x_0\)}\} / \sim.\]
\end{definition}

We would like to equip the above quotient with a group structure using the 
concatenation operation. To achieve this, we need to check that, for loops 
at \(x_0\), \(\gamma_1\) and \(\gamma_2\), we have 
\[[\gamma_1 * \gamma_2]_\sim = [\gamma_1]_\sim * [\gamma_2]_\sim\]
and the equivalence class is compatible to concatenation independently of the 
end representative of the equivalence class.

\begin{proposition}
  Let \(\gamma_1, \gamma_1'\) be paths from \(x\) to \(y\) and let \(\gamma_2, \gamma_2'\) 
  be paths from \(y\) to \(z\), then if \(\gamma_1 \sim \gamma_1'\) and 
  \(\gamma_2 \sim \gamma_2'\) then \(\gamma_1 * \gamma_2 \sim \gamma_1' * \gamma_2'\).
\end{proposition}
\begin{proof}
  Concatenate the homotopies. Namely, if \(H_1\) is a homotopy between \(\gamma_1\) 
  and \(\gamma_1'\) and \(H_2\) is a homotopy between \(\gamma_2\) and \(\gamma_2'\), 
  then define \(H(t, u) := H_1(t, u) * H_2(t, u)\). 
\end{proof}

With the above proposition, we define the group operation on \(\pi_1(X, x_0)\) 
such that \([\gamma_1] * [\gamma_2] := [\gamma_1 * \gamma_2]\).

\begin{lemma}
  Let \(\gamma : [0, 1] \to X\) be a path and let \(\phi : [0. 1] \to [0, 1]\) be 
  a continuous function such that \(\phi(0) = 0\) and \(\phi(1) = 1\) (such \(\phi\) is known 
  as a reparametrisation. Then, \(\gamma \circ \phi \sim \gamma\). 
\end{lemma}
\begin{proof}
  We define the homotopy \(H(t, u) := \gamma((1 - u)t + u\phi(t))\). \(H\) is 
  clearly continuous, \(H(0, u) = \gamma(0), H(1, u) = \gamma(1)\) and 
  \(\gamma(t) = H(t, 0), \gamma(\phi(t)) = H(t, 1)\).
\end{proof}

\begin{proposition}
  \((\pi_1(X, x_0), *)\) form a group.
\end{proposition}
\begin{proof}
  The identity element of this group is the class of the constant loop \(\text{id} : t \mapsto x_0\). 
  Indeed, for any loops \(\gamma\) at \(x_0\), we have 
  \[\text{id} * \gamma : t \mapsto \begin{cases}
    \gamma(2t), \ & t \le 1 / 2,\\
    x_0, \ & t > 1 / 2. 
  \end{cases}\]
  Thus, we may construct the homotopy between \(\gamma\) and \(\text{id} * \gamma\) by 
  defining 
  \[H(t, u) := \begin{cases}
    \gamma((1 + u)t), \ & t \le 1 / (1 + u),\\
    x_0, \ & t > 1 / (1 + u).
  \end{cases}\]
  This is a homotopy by the glueing lemma and thus, \([\text{id} * \gamma] = [\gamma]\).
  Alternatively, we observe \(\text{id} * \gamma\) is a reparametrisation of \(\gamma\) 
  with \(\phi(t) = 2t\) for \(t \le 1 / 2\) and \(\phi(t) = 1\) for all \(t > 1 / 2\).

  It is to check that, with the above definition of the identity, the inverse of the 
  loop \([\gamma]\) is simply \([t \mapsto \gamma(1 - t)]\). Thus, it remains to 
  check associativity.

  Let \(\gamma_1, \gamma_2, \gamma_3\) be loops at \(x_0\). We note that 
  \((\gamma_1 * \gamma_2) * \gamma_3 \neq \gamma_1 * (\gamma_2 * \gamma_3)\) though 
  the two paths remain to be homotopic. Indeed, we see that the two paths are 
  simply reparametrisations of each other with 
  \[\phi : [0, 1] \to [0, 1] : t \mapsto \begin{cases}
    2t, \ & t \le 1 / 4,\\
    t + 1 / 4, \ & 1 / 4 < t \le 3 / 4,\\
    t / 2 + 3 / 4, \ & t > 3 / 4,
  \end{cases}\] 
  such that \((\gamma_1 * \gamma_2) * \gamma_3 \circ \phi = \gamma_1 * (\gamma_2 * \gamma_3)\). 
  Hence, by the above lemma, the two paths are homotopic and so, 
  \[([\gamma_1] * [\gamma_2]) * [\gamma_2] = [\gamma_1] * ([\gamma_2] * [\gamma_2]),\]
  as required.
\end{proof}

\begin{proposition}
  Let \(x_0, x_1 \in X\), and let \(\delta : [0, 1] \to X\) be a path from \(x_0\) 
  to \(x_1\). Then \(\delta\) induces and isomorphism between \(\pi_1(X_, x_0)\) 
  and \(\pi_1(X_, x_1)\) by 
  \[[\gamma] \mapsto [\delta^{-1} * \gamma * \delta].\]
\end{proposition}
\begin{proof}
  Clearly, for the constant path, \(\delta^{-1} * \text{id} * \delta\) is an 
  reparametrisation of \(\delta^{-1} * \delta\) which is homotopic to \(\text{id}\).
  
  Now, given \(\gamma_1, \gamma_2\) loops at \(x_0\), 
  \[\begin{split}
    [\delta^{-1} * \gamma_1 * \gamma_2 * \delta] & = [\delta^{-1} * \gamma_1] * [\gamma_2 * \delta] \\
    & = [\delta^{-1} * \gamma_1] * [\delta * \delta^{-1}] * [\gamma_2 * \delta]  \\
    & = [\delta^{-1} * \gamma_1 * \delta] * [\delta^{-1} * \gamma_2 * \delta].
  \end{split}\]
  Finally, as \(\delta^{-1}\) by symmetry induces a homomorphism from \(\pi_1(X, x_1)\) 
  to \(\pi_1(X, x_0)\) and these two homomorphisms are inverses, the induced map
  in an isomorphism as required.
\end{proof}

With the above proposition in mind, we see that the fundamental group of a space is 
independent (up to isomorphism) of the choice of the base point. So, we have also 
that a space is simply connected if and only if its fundamental group is trivial.

\subsection{Covering Spaces}

In the section we generalize the method introduced for toruses to general spaces.

\begin{definition}[Covering Map]
  A map \(\pi : \tilde X \to X\) is a covering map if there exists an open cover 
  \((U_\alpha)_{\alpha \in A}\) of \(X\) for all \(\alpha \in A\), \(\pi^{-1}(U_\alpha)\) 
  is a disjoint union of open sets of \(\tilde X\) each of which is homeomorphic to 
  \(U_\alpha\) with \(\pi\) (i.e. the \(\pi\) restricts on such an open set is 
  homeomorphic to \(U_\alpha\)).

  Given a covering map \(\pi\), we call \(X\) the base space and \(\tilde X\) the 
  covering map.
\end{definition}

In some sense, the covering map provides a local homeomorphism for some specific 
open cover which each open set of the cover is represented in the domain as disjoint 
copies. 

\begin{proposition}
  If \(\pi : \tilde X \to X\) is a covering map for some connected \(X\), then 
  the cardinalities of the fibres of \(\pi\) is constant. Namely the map
  \[s : X \to \overline{\mathbb{N}} : x \mapsto \# \pi^{-1}(\{x\})\]
  is constant. We call this constant the number of sheets of \(\pi\).
\end{proposition}
\begin{proof}
  It is clear that for \(x, y \in U_\alpha\) where \(\alpha \in A\), the fibres of 
  \(x\) and \(y\) have the same cardinalities. Thus, as \((U_\alpha)_{\alpha \in A}\) is 
  an open cover, for all \(x \in X\),
  \[s^{-1}\{s(x)\} = \bigcup_{\substack{\exists x' \in U_\alpha, \\s(x') = s(x)}} U_\alpha.\] 
  Hence, \(s^{-1}\{s(x)\}\) is open. Then, then we have the disjoint open cover of \(X\)
  \[\{s^{-1}\{s(x)\} \mid x \in X\}.\]
  But, since \(X\) is connected, any disjoint open cover of \(X\) can have at most 1 
  element, and thus, \(s\) is a constant as required.
\end{proof}

\begin{proposition}
  Suppose \(\pi : \tilde X \to X\) be a covering map and let \(Y\) be a topological space and 
  \(f : [0, 1] \times Y \to X\) is a continuous map such that there exists some 
  \(\tilde f_0 : Y \to \tilde X\) such that \(\pi \circ \tilde f_0 = f(0, \cdot)\).
  Then, there exists a unique \(\tilde f : [0, 1] \times Y \to \tilde X\) such that 
  \(\tilde f(0, \cdot) = \tilde f_0\) and \(\pi \circ \tilde f = f\).
\end{proposition}
\begin{proof}
  For all \(y \in Y\), there exists a neighbourhood \(N\) of \(y\) such that for 
  all \(t \in [0, 1]\), there exists some \(I_t \subseteq [0, 1]\) containing 
  \(t\) such that \(f(I_t \times N) \subseteq U_\alpha\) for some \(\alpha\) since 
  \(f\) is continuous on the compact set \([0, 1]\).
  
  With this in mind, we can build the lift on \([0, 1] \times N\). for \(\{0\} \times N\), 
  by assumption, the lift must be \(\tilde f_0\). By the construction of \(N\), 
  there exists some \(U_\alpha\) such that \(f(\{0\} \times N) \subseteq U_\alpha\).
  Then, by continuity, there exists a neighbourhood \(I_0 \subseteq [0, 1]\) such 
  that \(f(I_0 \times N) \subseteq U_\alpha\) and so, we may define 
  \[\tilde f : N \times I_0 \to \tilde X : (t, y) \mapsto (\pi|_{\tilde U_\alpha(\beta)})^{-1} \circ f\] 
  as \(\pi^{-1}\) is a homeomorphism on \(\tilde U_\alpha(\beta)\) where 
  \(\pi^{-1}(U_\alpha) = \bigsqcup_{\beta \in B} \tilde U_\alpha(\beta).\)

  Now, by the compactness of \([0, 1]\), there exists 
  \[0 = t_0 < t_1 < \cdots < t_n < t_{n + 1} = 1,\]
  such that for all \(k = 0, \cdots, n\), \(f([t_k, t_{k + 1}] \times N) \subseteq U_{\alpha_k}\) 
  for some \(\alpha_k \in A\). Now, by induction, we may extend \(\tilde f\) from 
  \([0, t_k]\) to \([0, t_{k + 1}]\) resulting in an extension of \(\tilde f\) to 
  the whole set of \([0, 1]\). 

  Suppose now both \(\tilde f, \bar f\) lifts \(f\) with 
  \(\tilde f(0, \cdot) = \bar f(0, \cdot) = \tilde f_0\). Let 
  \[A := \{z \in [0, 1] \times N \mid \tilde f(z) = \bar f(z)\}.\]
  Then, for all \(z \in A\), there exists some \(V_2 \subseteq N \times [0, 1]\) 
  such that \(\tilde f(V_z), \bar f(V_z) \subseteq \tilde U_\alpha(\beta)\). 
  On \(V_z\), it is clear that \(\tilde f = \bar f = (\pi\mid_{\tilde U_\alpha(\beta)})^{-1} \circ f\).
  Now, since \(N\) is a union of connected components, we obtain uniqueness.
\end{proof}

\subsection{Induced Maps}

In the case that we have a map between to topological spaces, we can define an 
induced map on their fundamental group by composing the map with the loops.

\begin{definition}[Induced Map]
  Let \(X, Y\) be topological spaces and let \(x, y\) be elements of \(X\) and \(Y\) respectively,
  then if \(f : X \to Y\) is a continuous map such that \(f(x) = y\), then, the induced map 
  \(f_*\) is defined to be 
  \[f_* : \pi_1(X, x) \to \pi_1(Y. y) : [\gamma] \mapsto [f \circ \gamma].\]
\end{definition}

It is clear that this map is well-defined since if \(H\) is a homotopy between 
\(\gamma_1, \gamma_2\) loops in \(X\) based at \(x\), then 
\(f \circ H\) is a homotopy between \(f \circ \gamma_1\) and \(f \circ \gamma_2\).

\begin{proposition}
  The induced map \(f_* : \pi_1(X, x) \to \pi_1(Y, y)\) of \(f\) is a group homomorphism.
\end{proposition}
\begin{proof}
  Clearly, \(f \circ \text{id}\) is the constant loop based at \(y\) and thus, 
  \(f_*\) maps the identity to the identity. 

  Now, let \(\gamma_1, \gamma_2\) be loops based at \(x\), it suffices to show that 
  \((f \circ \gamma_1) * (f \circ \gamma_2)\) is homotopic to \(f \circ (\gamma_1 * \gamma_2)\).
  But, in fact, the two paths above are equal. Hence, homotopic and thus, \(f_*\) 
  is a group homomorphism as required.
\end{proof}

\begin{proposition}
  Let \(\pi : \tilde X \to X\) be a covering space and suppose \(\pi(\tilde x) = x\) 
  for some \(\tilde x \in \tilde X\) and \(x \in X\). Then, the induced map of \(\pi\),
  \[\pi_* : \pi_1(\tilde X, \tilde x) \to \pi_1(X, x)\]
  is injective.

  Since \(\pi_*\) is a injective group homomorphism, in some sense, the covering 
  space allows us to consider a fundamental group of the covering space as a subgroup 
  of the fundamental group of the original space.

  In this sense, the induced map of the covering map provides a correspondence 
  between the covering space (quotiented by some relation) and the subgroups of the 
  fundamental group.
\end{proposition}
\begin{proof}
  We will show that \(\ker \pi_*\) is trivial. Suppose \(\gamma\) is a loop based at 
  \(\tilde x\) such that \(\pi \circ \gamma\) is trivial and let \(H\) be the 
  homotopy between \(\pi \circ \gamma\) and the constant loop based at \(x\). 
  We will lift \(H\) along \(\pi\). 

  Let \(\tilde H : [0, 1] \times [0, 1] \to \tilde X\) be the lift of \(H\) such that 
  \(\gamma(t) = \tilde H(0, t)\) and \(\pi \circ \tilde H = H\). As \(\pi\) is locally 
  homeomorphic and \(\tilde H\) is continuous, as
  \(\tilde H(0, 0) = \tilde H(0, 1) = \gamma(0) = \tilde x\), we have 
  \(\tilde H(u, 0) = \tilde H(u, 1) = \tilde x\). Now, by continuity, \(\tilde H(1, t)\) 
  is constant with \(H(1, t) = x\) we have \(\tilde H(1, t) = \tilde x\) and thus, 
  \(\tilde H\) is a homotopy between \(\gamma\) and the constant path as required. 
\end{proof}

An important example is the covering map \(\pi : S^1 \to S^1 : z \mapsto z^n\), 
then the induced map \(\pi_* : \mathbb{Z} \to \mathbb{Z}\) (as \(\pi_1(S^1, 1)\) 
is isomorphic to \(\mathbb{Z}\)) is the map \(k \mapsto nk\). 

\subsection{Universal Covers}

\begin{definition}[Universal Cover]
  Let \(X\) be a topological space. A universal cover of \(X\) is a space \(\tilde X\) 
  that covers \(X\) (i.e. there exists a covering map \(\pi : \tilde X \to X\)) and 
  is simply connected.
\end{definition}

\begin{theorem}
  If a topological space \(X\) is path connected, locally path connected and locally simply 
  connected, then \(X\) has a unique universal cover.

  We say a space has a property locally if every basis of neighbourhood has that 
  property.
\end{theorem}

Although this is a powerful theorem, without any specific construction of the universal 
cover, we will not be able to make much conclusion about the space. Therefore, 
this theorem is not very useful for our purpose. We will provide universal covers 
for most spaces we work with explicitly. For this reason, we will only provide the 
proof for the uniqueness.

\begin{lemma}
  Let \(Y\) be a topological space that is simply connected and locally path connected
  and suppose \(\pi : \bar X \to X\) is a covering map, \(f : Y \to X\) be 
  continuous such that \(f(y) = \pi(\bar x) = x\) for some \(y \in Y, \bar x \in \bar X, x \in X\). 
  Then, there exists a continuous unique map \(\bar f : Y \to \bar X\) such that 
  \(\pi \circ \bar f = f\). i.e. the following diagram commutes
  \[\begin{tikzcd}
    && {\bar X \ni \bar x} \\
    {y \in Y} && {X\ni x}
    \arrow["\pi", from=1-3, to=2-3]
    \arrow["f"', from=2-1, to=2-3]
    \arrow["{\exists! \bar f}", dashed, from=2-1, to=1-3]
  \end{tikzcd}\]
\end{lemma}
\begin{proof}
  For all \(y' \in Y\), let \(\gamma\) a path 
  from \(y\) to \(y'\), take \(\bar f(y')\) be the end point of the unique lift of \(f \circ \gamma\) 
  starting at \(\bar x\). This is well-defined since if \(\gamma'\) is another 
  path from \(y\) to \(y'\), then \(\gamma'\) is homotopic to \(\gamma'\) and thus, 
  \(f \circ \gamma\) and \(f \circ \gamma'\) are homotopic and hence, their lifts 
  are also homotopic with the same end points.

  We will now show \(\bar f\) is continuous. Let \(U \subseteq X\) be a open neighbourhood 
  of \(f(y')\) such that \(\pi^{-1}(U)\) is a disjoint copies of sets \(\{U_\alpha\}_{I'}\)
  such that, on each copy, the restriction of \(\pi\) form a homeomorphism onto 
  \(U\). Suppse \(\bar U\) is one such copy and let \(V\) be a sufficiently small 
  neighbourhood of \(y'\) such that \(V \subseteq f^{-1}(U)\). I claim 
  \(V \subseteq \bar f^{-1}(\bar U)\). Indeed, any paths from \(y\) to \(y'\) 
  can be concatenated to a path with end point \(y''\) in \(V\). By definition, 
  \(f(y'')\) is the end point of the lift of this concatenated path composed with \(f\) 
  based at \(\bar x\), and thus, since \(\bar f(y') \in \bar U\), the lift remains 
  in \(U\) and so \(y'' \in V\). Hence, as \(\bar U\) form a basis of open sets of \(\bar X\), 
  \(\bar f\) is continuous.

  Finally, to show uniqueness, we observe that, for any \(y' \in Y\) and 
  a path \(\gamma\) from \(y\) to \(y'\), \(f \circ \gamma = \pi \circ \bar f \circ \gamma\), 
  i.e. \(\bar f \circ \gamma\) equals the lift of \(f \circ \gamma\) based at some point. 
  Now, as we require \(\bar f(y) = \bar x\), this lift must be based at \(\bar x\) 
  and hence uniqueness follows by the uniqueness of the lift. 
\end{proof}

\begin{proposition}
  If \(\pi_1 : \tilde X_1 \to X\) and \(\pi_2 : \tilde X_2 \to X\) are covering maps 
  from simply connected spaces and \(X\) is path-connected such that for some 
  \(x \in X, \tilde x_1 \in \tilde X_1, \tilde x_2 \in \tilde X_2\),
  \(\pi_1(\tilde x_1) = x = \pi_2(\tilde x_2)\), then there exists a unique  
  homeomorphism \(\phi : \tilde X_1 \to \tilde X_2\) such that the following diagram commutes.
  \[\begin{tikzcd}
    {\tilde X_1} && {\tilde X_2} \\
    & X
    \arrow["\exists! \phi", from=1-1, to=1-3]
    \arrow["{\pi_1}"', from=1-1, to=2-2]
    \arrow["{\pi_2}", from=1-3, to=2-2]
  \end{tikzcd}\]
\end{proposition}
\begin{proof}
  Take \(\phi\) to be the lift of \(\pi_1\) to \(\tilde X_2\) and one may check 
  that the lift of \(\pi_2\) to \(\tilde X_1\) provides the inverse.
\end{proof}

\begin{proposition}
  Let \(\pi : (\tilde X, \tilde x) \to (X, x)\) be a universal covering map where \(X\) is locally 
  path-connected and let \(p : (\bar X, \bar x) \to (X, x)\) be another covering map.
  Then, there exists a unique \(f : (\tilde X, \tilde x) \to (\bar X, \bar x)\)
  such that \(\pi = p \circ f\). i.e. the following diagram commutes.
\end{proposition}
\begin{proof}
  Clear by lifting \(\pi\) to \(\tilde X \to \bar X\).
\end{proof}

\subsection{Classification of Covering Maps}

Let us recall some definitions from group theory.

\begin{definition}[Group Action]
  An action of a group \(\Gamma\) on a topological space \(X\) is a group homomorphism 
  \(\rho : \Gamma \to \text{Homeo}(X)\).
\end{definition}

Equivalently, an action is a map 
\[\rho : \Gamma \times X \to X : (\gamma, x) \mapsto \gamma \cdot x\]
is continuous whenever \(\Gamma\) is endowed with the discrete topology and 
for all \(\gamma_1, \gamma_2 \in \Gamma\),
\begin{itemize}
  \item \(\gamma_1 \cdot (\gamma_2 \cdot x) = (\gamma_1 \gamma_2) \cdot x\);
  \item \(e \cdot x = x\).
\end{itemize}

\begin{definition}
  A group action from \(\Gamma\) to \(X\) is said to be ``nice'' if for all 
  \(x \in X\), there exists \(U_x\) a neighbourhood of \(x\) such that for all 
  \(\gamma \neq e\), \(\gamma \cdot U_x \cap U_x = \varnothing\).
\end{definition}

\begin{proposition}
  If \(\Gamma\) acts nicely on \(X\), then for all \(\gamma_1, \gamma_2 \in \Gamma\), 
  \(\gamma_1 \neq \gamma_2\), there exists some \(U_x\) a neighbourhood of \(x\) 
  such that \(\gamma_1 \cdot U_x \cap \gamma_2 \cdot U_x = \varnothing\).
\end{proposition}
\begin{proof}
  Since \(\Gamma\) acts nicely on \(X\), there exists some \(U_x\) a neighbourhood 
  of \(x\) such that \(\gamma_2^{-1} \gamma_1 \cdot U_x \cap \cdot U_x = \varnothing\).
  Thus, \(\varnothing = \gamma_2 \cdot (\gamma_2^{-1} \gamma_1 \cdot U_x )\gamma_2 \cdot U_x
  = \gamma_1 \cdot U_x \cap \gamma_2 \cdot U_x\) as required.
\end{proof}

Let us introduce the equivalence relation where \(x \sim y\) if and only if 
there exists some \(\gamma \in \Gamma\) such that \(\gamma \cdot x = y\). 
We denote the quotient topology by this equivalence relation by \(X / \Gamma\).

\begin{theorem}
  Assume that \(\Gamma\) acts nicely on \(X\). Then, the projection map 
  \[\pi : X \to X / \Gamma\]
  is a covering map.
\end{theorem}
\begin{proof}
  Let \(x \in X\) and take \(U_x\) to be a neighbourhood such that for all 
  \(\gamma_1, \gamma_2 \in \Gamma\), \(\gamma_1 \neq \gamma_2\), 
  \(\gamma_1 \cdot U_x \cap \gamma_2 \cdot U_x = \varnothing\). By the definition 
  of the equivalence relation, \(\pi(\gamma_1 \cdot U_x) = \pi(\gamma_2 \cdot U_x)\)
  for all \(\gamma_1, \gamma_2 \in \Gamma\) and so, taking \(V := \pi(U_x)\), 
  \(\pi^{-1}(V) = \bigsqcup_{\gamma \in \Gamma} \gamma \cdot U_x\) and it remains 
  to show \(\pi\mid_{\gamma \cdot U_x}\) is a homeomorphism for all \(\gamma\). 
  This is however clear since it is continuous and has the continuous inverse 
  (since the action is continuous)
  \[u \in U_x \mapsto \gamma \cdot u \in \gamma \cdot U_x.\]
  Hence, taking the open cover to be \(\{U_x\}_{x \in X}\), we have \(\pi\) is 
  a covering map.
\end{proof}

\begin{proposition}
  Let \(X\) be simply connected, path connected and locally simply and path connected 
  and suppose \(\Gamma\) acts nicely on \(X\). Then, 
  \begin{itemize}
    \item \(X\) is the universal cover of \(X / \Gamma\),
    \item for all \([x] \in X / \Gamma\), \(\Gamma \simeq \pi_1(X / \Gamma, [x])\).
  \end{itemize}
\end{proposition}
\begin{proof}
  We know that \(\pi : X \to X / \Gamma\) is a covering map from the above theorem. 
  Now, since \(X\) is simply connected, by uniqueness, \(\pi\) is the universal 
  covering of \(X / \Gamma\).

  Let us define the following map 
  \[\Phi : \Gamma \to \pi_1(X / \Gamma, [x]) : \gamma \mapsto [x \to \gamma \cdot x] (:= [\text{path from } x \text{ to } \gamma \cdot x]).\]
  This map is well-defined since \(X\) is simply connected. This map is a group 
  homomorphism since it clearly maps \(e\) to \([\text{id}]\) and for \(\gamma_1, \gamma_2 \in \Gamma\), 
  \[\begin{split}
    \Phi(\gamma_1 \gamma_2) & = [x \to \gamma_1 \gamma_2 \cdot x] = [x \to \gamma_1 * \gamma_1 \cdot x \to \gamma_1 \gamma_2 \cdot x]\\
  & = [x \to \gamma_1 * \gamma_1(x \to \gamma_2 \cdot x)] = [x \to \gamma_1] * [x \to \gamma_2] = \phi(\gamma_1) \Phi(\gamma_2).
  \end{split}\]
  \(\Phi\) is injective, since for all \(\gamma \in \ker \Phi\), \(\pi \circ \delta\) 
  is homotopic to the constant path for all \(\delta : x \to \gamma \cdot x\). 
  Then, taking the lift of the homotopy, we have that \(\delta\) has the same fixed 
  points implying \(\gamma \cdot x = x\) and so, \(\gamma = e\) (as the action is nice). 

  Finally, \(\Phi\) is surjective since for all \(\gamma \in \pi_1(X / \Gamma, [x])\), we 
  may lift \(\gamma\) as a path to \(X\) based at \(x\), \(\tilde \gamma : x \to \gamma' \cdot x\) 
  for some \(\gamma' \in \Gamma\). Thus, \(\Phi(\gamma') = [\gamma]\) implying 
  surjectivity.
\end{proof}

Let \(X\) be a connected, path-connected and locally simply and path-connected 
topological space. Then, if \(\pi : \tilde X \to X\) is a covering map, we would like 
to construct an action of \(\pi_1(X, x)\) for some \(x \in X\) \(\pi(\tilde x) = x\).

Let \(\gamma \in \pi_1(X, x)\), then for any \(\tilde x\), let \(\tilde \gamma\) 
be the unique lift of \(\gamma\) based 
at \(\tilde x\). Then, we may define the action of \(\pi_1(X, x)\) on \(\tilde X\) 
by taking \(\gamma \cdot \tilde x = \tilde \gamma(1)\). It is an easy exercise 
to check that this is indeed a group action. This action is in fact nice. Indeed,
for all \(x \in X\), by the definition of covering map, let \(U\) be a neighbourhood 
of \(x\) satisfying the covering property. Then, given \(\gamma \neq \text{id}\),
\(U_{\tilde x} \cap U_{\gamma \tilde x} = \varnothing\), and thus, 
\(U_x \cap \gamma \cdot U_x = \pi(U_{\tilde x}) \cap \pi(U_{\gamma \tilde x}) = \varnothing\).

We will in the remainder of this section assume \(X\) is path-connected, 
locally path- and simply-connected.

Let \(\pi : (\bar X, \bar x) \to (X, x)\) be a covering map. Then, as we have 
seen, this induces an injective group homomorphism 
\(\pi_* : \pi_1(\bar X, \bar x) \to \pi_1(X, x)\). As alluded to before, 
we may interpret the fundamental group in the covering space as a subgroup 
of the fundamental group in the space by taking 
\[\Gamma = \pi_*(\pi_1(\bar X, \bar x)) \le \pi_1(X, x),\]
and \(\Gamma \simeq \pi_1(\bar X, \bar x)\). We ask whether or not the reverse 
is true. Namely, for all subgroups \(\Gamma\) of \(\pi_1(X, x)\), does there exist a 
covering space \(p : \bar X \to X\) such that 
\[\Gamma = p_*(\pi_1(\bar X, \bar x)).\]
Furthermore, if \(p_1 : (\bar X, \bar x) \to (X, x)\), 
\(p_2 : (\tilde X, \tilde x) \to (X, x)\) are two 
covering maps which induces the same subgroup, then the two covering 
spaces are homeomorphic which maps \(\bar x\) to \(\tilde x\).

\begin{theorem}
  If \(\Gamma\) is a subgroup of \(\pi_1(X, x)\), then there exists a covering 
  map \(\pi : \bar X \to X\) such that \(\Gamma = \pi_*(\pi_1(\bar X, \bar x))\).
\end{theorem}
\begin{proof}
  By the assumption on \(X\), there exists a universal cover 
  \(\pi : (\tilde X, \tilde x) \to (X, x)\) together with a nice action of 
  \(\pi_1(X, x)\) on \(\tilde X\) such that \(X \simeq \tilde X / \pi_1(X, x)\).
  Now, by restricting the action onto \(\Gamma\), we obtain an action of \(\Gamma\) 
  on \(\tilde X\) which is nice. In particular, we have the following chain 
  of quotients.
  \[\begin{tikzcd}
    {(\tilde X, \tilde x)} & {(\tilde X / \Gamma, [\tilde x]_\Gamma)} && {(X, x)}
    \arrow["{q_\Gamma}", from=1-1, to=1-2]
    \arrow["{q_{\pi_1(X, x)}}", from=1-2, to=1-4]
    \arrow["\pi"', curve={height=18pt}, from=1-1, to=1-4]
  \end{tikzcd}\]
  Now, since \(\Gamma\) acts nicely on \(\tilde X\), \(q_\Gamma\) is a covering 
  map, and so, \(q_{\pi_1(X, x)}\) is also a covering map. Furthermore, 
  we have \(\pi_1(\tilde X / \Gamma, [x]_\Gamma) \simeq \Gamma\), and thus, it suffices 
  to show \((q_{\pi_1(X, x)})_*(\pi_1(\tilde X / \Gamma, [x]_\Gamma)) = \Gamma\).
  This follows by lifting \(\gamma \in \Gamma\) to \(\tilde X\) to show 
  surjectivity.
\end{proof}

We will now show the uniqueness of this covering map. 

\begin{theorem}
  Suppose now \(p : (\bar X, \bar x) \to (X, x)\) is a covering map such that 
  \[\Gamma = p_*(\pi_1(X, x)),\]
  then there exists a homeomorphism 
  \(\Phi : (\tilde X / \Gamma, [x]_\Gamma) \to (\bar X, \bar x)\) such that 
  the following diagram commutes.
  \[\begin{tikzcd}
    {(\tilde X / \Gamma, [x]_\Gamma)} && {(\bar X, \bar x)} \\
    \\
    {(X, x)} && {(X, x)}
    \arrow["\pi"', from=1-1, to=3-1]
    \arrow["p", from=1-3, to=3-3]
    \arrow["\Phi", from=1-1, to=1-3]
    \arrow["{\text{id}_X}", from=3-1, to=3-3]
  \end{tikzcd}\]
  where we denote \(\pi : (\tilde X, \tilde x) \to (X, x)\) the universal 
  cover.
\end{theorem}
\begin{proof}
  Redo this.
\end{proof}

\subsection{Van Kampen Theorem}

Review free groups. 

Given any finitely generated group \(G = \langle g_1, \cdots, g_n\rangle\), 
we can define the homomorphism 
\[\Phi : \mathbb{F}_n \to G : a_i \mapsto g_i\]
where \(a_1, \cdots, a_n\) are characters of \(\mathbb{F}_n\). Thus, with this 
in mind, \(G \simeq \mathbb{F}_n/ \ker \Phi\). We note that \(\ker \Phi\) encodes 
``relations'' on elements of \(G\). With this isomorphism, we can construct 
groups from \(\mathbb{F}_n\). Let \(r_i \in I\) be reduced words and 
denote \(\langle r_i \mid i \in I\rangle\) the normal subgroup generated 
by \(r_i \in I\). Then, we denote 
\[\langle a_1, \cdots, a_n \mid r_i(a_1, \cdots, a_n) = 1, 
  i \in I\rangle := \frac{\mathbb{F}_n}{\langle r_i \mid i \in I\rangle}.\]
We call this a presentation of the group. This is the construction we are 
referring to whenever we say to compute a group. 

\begin{definition}[Free Product]
  Given finitely generated groups \(G_1, G_2\) with presentations 
  \(\langle a_1, \cdots, a_n \mid r_i(a_1, \cdots, a_n) = 1, i \in I\rangle\) 
  and \(\langle b_1, \cdots, b_m \mid s_j(b_1, \cdots, b_m) = 1, j \in J\rangle\),
  we define the free product between \(G_1\) and \(G_2\) to be 
  \[G_1 * G_2 := \langle a_1, \cdots, a_n, b_1, \cdots, b_m \mid 
  r_i(a_1, \cdots, a_n) = 1, i \in I, s_j(b_1, \cdots, b_m) = 1, 
  j \in J\rangle.\]
\end{definition}

\begin{definition}[Amalgamated Products]
  Let \(H\) be a group and let \(h_1 : H \to G_1\), \(h_2 : H \to G_2\) be 
  group homomorphisms- Then, the amalgamated product of \(G_1\) and \(G_2\) 
  along \(H\) is 
  \[G_1 *_H G_2 := 
    \frac{G_1 * G_2}{\langle h_1(h) h_2(h)^{-1} \mid h \in H\rangle}.\]
\end{definition}

\begin{theorem}[Van Kampen Theorem]
  If \((X, x_0)\) is a path-connected pointed space and \(A, B \subseteq X\) are 
  open subsets of \(X\), \(A \cup B = X\) and \(x_0 \in A \cap B\) of which 
  \(A \cap B\) is path connected. Then, 
  \[\pi_1(X, x_0) \simeq \pi_1(A, x_0) *_{\pi_1(A \cap B, x_0)} \pi_(B, x_0).\]
\end{theorem}
\begin{proof}
  Let \(j_A : A \hookrightarrow X\) and \(j_B : B \hookrightarrow X\) be the 
  inclusion maps which induces corresponding group homomorphisms. Then, 
  by the universal property of the free product, we obtain the group homomorphism 
  \[\Phi : \pi_1(A, x_0) * \pi_1(B, x_0) \to \pi_1(X, x_0).\]
  Thus, it suffices to show \(\Phi\) is surjective with \(\ker \Phi = 
  \langle (j_A)_*(h) (j_B)_*(h)^{-1} \mid h \in \pi_1(A \cap B, x_0) \rangle\).

  We will omit the injective part and only show surjectivity. Let \(\gamma\) be 
  a loop on \(X\) based at \(x_0\). Then, since \(\gamma\) is continuous 
  and \(A \cup B = X\), for all \(t\) there exists an open interval \(I_t \subseteq [0, 1]\) 
  containing \(t\) such that \(\gamma(I_t) \subseteq A\) or 
  \(\gamma(I_t) \subseteq B\) (compactness argument). Then, there exists 
  some 
  \[t_0 = 0 < t_1 < \cdots < t_n < t_{n + 1} = 1\]
  such that \(\gamma([t_i, t_{i + 1}]) \subseteq A\) or \(B\) and for all 
  \(i, \gamma(t_i) =: x_i \in A \cap B\). Now, since \(A \cap B\) is path 
  connected, there exists \(\delta_i\) paths from \(x_0\) to \(x_i\). Then, 
  denoting \(\gamma_k := \gamma|_{[t_i, t_{i + 1}]}\), we have 
  \[[\gamma] = 
    [(\delta_0 * \gamma_0 * \delta_1^{-1}) 
    * (\delta_1 * \gamma_2 * \delta_2^{-1})
    * \cdots * (\delta_n * \gamma_n * \delta_{n + 1}^{-1})],\]
  where \(\gamma_i' := \delta_i \gamma_i \delta_{i + 1}^{-1}\) 
  is either in \(A\) or \(B\) and is a loop based at \(x_0\). Hence, 
  \([\gamma]\) is a product of elements of either 
  \(\pi_1(A, x_0)\) or \(\pi_1(B, x_0)\) and so, 
  \([\gamma] \in \Phi(\pi_1(A, x_0) * \pi_1(B, x_0))\).
\end{proof}

\newpage
\section{Homology}

\subsection{\texorpdfstring{\(\Delta\)}--Complex}

\begin{definition}[Simplex]
  An \(n\)-dimensional simplex is the set 
  \[\Delta_n := \{(x_0, \cdots. x_n) \in \mathbb{R}^{n + 1} \mid x_i \ge 0, \sum x_i = 1\}.\]
\end{definition}

We see that \(\Delta_0 = \{1\} \subseteq \mathbb{R}\) and so, \(\Delta_0\) is 
homeomorphic to a point. Similarly, 
\(\Delta_1 = \{(x_0, x_1) \in \mathbb{R}^2 \mid x_0, x_1 \ge 0, x_0 + x_1 = 1\}\)
is the line corresponding to \(x_0 + x_1 = 1\) in the first quadrant. So, 
\(\Delta_1 \simeq [0, 1]\). As an exercise, one may show that 
\[\Delta_k \simeq \overline{B_0(1)} \subseteq \mathbb{R}^k.\]
While the simplex is homeomorphic to the closed ball, it processes some additional 
geometric properties we care about.  

We observe that \(\partial \Delta_k\) is a union of \(k + 1\)-copies of 
\(\Delta_{k - 1}\). Indeed, defining \((e_i)_{i = 0}^k\) to be the standard basis 
of \(\mathbb{R}^{k + 1}\), we observe 
\[\Delta_k = \text{conv}(e_i \mid i = 0, \cdots, k) \subseteq \mathbb{R}^{k + 1},\]
where \(\text{conv}(S) = \{\sum_{i = 0}^k t_i s_i \mid s_i \in S, \sum_{i = 0}^k t_i = 1\}\) 
is the convex hull of \(S\). Then, defining 
\[\partial_i \Delta_k := \{\sum_{i \neq j} t_j e_j \mid \sum t_j = 1\},\]
it is clear \(\Delta_{k - 1} \simeq \partial_i \Delta_k\). Thus, by observing 
\(\partial \Delta_k = \bigcup_{i = 0}^{k} \partial_i \Delta_{k - 1}\), we have 
\(\partial \Delta_k\) is \(k + 1\) copies of \(\Delta_{k - 1}\). We note that 
this union is not disjoint and in fact, 
\(\partial_i \Delta_k \cap \partial_k \Delta_k\) is embedded canonically in 
\(\Delta_{k - 2}\).

\begin{definition}[\(\Delta\)-complex]
  A \(n\)-dimensional \(\Delta\)-complex structure on a set \(X\) is 
  the data of finitely many continuous maps 
  \[\phi_i^k : \Delta_k \to X\]
  for \(k \le n\) such that 
  \begin{itemize}
    \item for all \(i, k\), the restriction
      \(\phi_i^k|_{\Delta_k^\circ} : \Delta_k^\circ \to X\) is injective; 
    \item for all \(i, j\) and \(k \le n\), the restriction 
      \(\phi_i^k|_{\partial_j \Delta_k} : \partial_j \Delta_k \to X\) is equal 
      to \(\phi^{k - 1}_l : \Delta_{k - 1} \to X\) for some \(l\); 
    \item \(\bigcup_{k = 0}^n \bigcup_{i} \phi_i^k(\Delta_k) = X\).
  \end{itemize}
\end{definition}

As an example let us again consider the torus. We observe that, by cutting 
the square along its diagonal, we can embed two copies of \(\Delta_2\) inside 
the torus. This results in 3 copies of \(\Delta_1\) and 1 copy of \(\Delta_0\) 
after the quotient.  This provides a \(\Delta\)-complex structure on the torus 
by taking the maps to be the natural embeddings. 

We note that the definition of \(\Delta\)-complexes did not require \(X\) 
to have a topology. Nonetheless, we required \(\phi_i^k\) to be continuous. 
Thus, to be this make sense, simple take the topology on \(X\) to be the 
one induced by \((\phi_i^k)_{i, k}\).

We note that there is no unique \(\Delta\)-complex structure on a given 
space though as we shall see, there exists an invariance of the space independent 
of the \(\Delta\)-complex.

\subsection{Simplicial Homology}

The idea with simplicial homology is that for any topological space \(X\), 
we associate it to an abelian group \(H_k(X)\) for some \(k \in \mathbb{N}\) 
where \(H_k(X)\) encodes the information of the \(k\)-dimensional shapes 
embedded in \(k\). 

Let us first introduce some algebraic notions we will require. 

\begin{definition}[Chain Complex]
  A chain complex is a sequence of abelian groups \((C_n)_{n \in \mathbb{N}}\)
  and a sequence of group homomorphisms \(\partial_n : C_n \to C_{n - 1}\),
  \[\begin{tikzcd}
    \cdots & {C_{n+1}} & {C_n} & {C_{n-1}} & \cdots
    \arrow["{\partial_{n+2}}", from=1-1, to=1-2]
    \arrow["{\partial_{n+1}}", from=1-2, to=1-3]
    \arrow["{\partial_n}", from=1-3, to=1-4]
    \arrow["{\partial_{n-1}}", from=1-4, to=1-5]
  \end{tikzcd}\]
  such that \(\ker(\partial_n) \supseteq \text{Im}(\partial_{n + 1})\) for all \(n\).
  We bundle the two together by writing \((C_n, \partial_n)_{n \in \mathbb{N}}\).
\end{definition}

\begin{definition}[Homology Group]
  The homology group associated to the chain complex 
  \(\mathcal{C} := (C_n, \partial_n)_{n \in \mathbb{N}}\) are 
  \[H_n(\mathcal{C}) := \ker(\partial_n) / \text{Im}(\partial_{n + 1}).\]
\end{definition}

Suppose now \(X\) is equipped with a \(\Delta\)-complex. Then, we may define 
\[C_k(X) := \bigoplus_i \mathbb{Z} \cdot [\phi_i^k],\]
that is the free \textbf{abelian} group generated by symbols corresponding to 
\(\phi_i^k\) for each \(i\). Then, we define \(\partial_k : C_k(X) \to C_{k - 1}(X)\)
such that 
\[\partial_k([\phi_i^k]) = \sum_{j = 0}^k (-1)^{j + 1}[\partial_j \phi_i^k]\]
where we denote \(\partial_j \phi_i^k := \phi_i^k \mid_{\partial_j \Delta_k}\).

\begin{lemma}
  For all \(k \ge 1\), \(\partial_k \circ \partial_{k + 1} = 0\) (and so, 
  \(\text{Im}(\partial_{k + 1}) \subseteq \ker(\partial_k)\)).
\end{lemma}
\begin{proof}
  It suffices to show that \(0 = \partial_{k - 1} \circ \partial_k([\phi_i^k])\).
  By denoting \(\partial_{jl}\Delta_k\) the simplex generated by the ordered 
  tuple \((e_0, \cdots, e_{j - 1}, e_{j + 1}, \cdots, e_{l - 1}, e_{l + 1}, \cdots, e_k)\),
  we observe \(\partial_l (\partial_j \phi_i^k) = \phi_i^k |_{\partial_{jl} \Delta_k}\) for 
  all \(l \le k\) and \(\partial_l(\partial_j \phi_i^k) = \phi_i^k |_{\partial_{j(l+1)}\Delta_k}\) 
  for \(l > k\). Thus, 
  \[\begin{split}
    \partial_{k - 1}(\partial_k [\phi_i^k]) 
    & = \partial_{k - 1}\left(\sum_{j = 0}^k (-1)^{j + 1}[\partial_j \phi_i^k]\right)
      = \sum_{j = 0}^k (-1)^{j + 1}\partial_{k - 1}([\partial_{j(k+1)} \phi_i^k])\\
    & = \sum_{j = 0}^k (-1)^{j + 1}\sum_{l = 0}^{k - 1}(-1)^{l + 1}
      [\partial_l(\partial_{j(k+1)} \phi_i^k)]\\
    & = \sum_{j = 0}^k \sum_{l = 0}^{j}(-1)^{j + l}
      [\partial_{jl} \phi_i^k] 
      + \sum_{j = 0}^k \sum_{l = j + 1}^{k - 1}(-1)^{j + l}
      [\partial_{jl} \phi_i^k]\\
    & = \sum_{j = 0}^k \sum_{l = 0}^{j}(-1)^{j + l}
      [\phi_i^k |_{\partial_{jl} \Delta_k}] 
      + \sum_{j = 0}^k \sum_{l = j + 1}^{k - 1}(-1)^{j + l}
      [\phi_i^k |_{\partial_{j(l+1)}\Delta_k}] = 0,
  \end{split}\]
  where the last equality follows by change of indices. 
\end{proof}

With this lemma in mind, we have \((C_k(X), \partial_k)_{k \in \mathbb{N}}\) 
is a chain complex and we may define a homology group on this chain.

\begin{definition}[Homology Group of a \(\Delta\)-complex on \(X\)]
  The simplicial homology group of \(X\) is the homology group of the chain 
  \((C_k(X), \partial_k)_{k \in \mathbb{N}}\) where and we denote it by 
  \(H^\Delta_k(X)\).
\end{definition}

So far, we note that the definition of the simplicial homology depends on the 
\(\Delta\)-complex structure. However, as we shall see, the definition 
is in fact invariant under different \(\Delta\)-complexes. 

\subsection{Singular Homology}

We will now provide another construction of the homology group known as 
singular homology. This turns out to be the same homology group as the 
simplicial homology yet easier to manipulate formally. Yet, they are much 
more difficult to compute, hence our current construction.

\begin{definition}
  Let \(X\) be a topological space. For all \(n \in \mathbb{N}\), we define 
  \[C_n(X) := \bigoplus_{\substack{\Phi : \Delta_n \to X\\\text{continuous}}} 
    \mathbb{Z} \cdot [\Phi].\]
\end{definition}

Recall that \(n\)-dimensional \(\Delta\)-simplex is defined to be 
\(\text{conv}(e_0, \cdots, e_n) \subseteq \mathbb{R}^{n + 1}\) where \(e_i\) 
has 1 in the \(i\)-th position and 0 everywhere else. Thus, if 
\(A : \mathbb{R}^{n + 1} \to \mathbb{R}^{n + 1}\) is a linear map induced by 
a permutation on the \(e_i\)s, we would like to identify \([\Phi]\) with
\([\Phi \circ A]\) if \(A\) preserves orientation (namely the permutation 
correspondence to \(A\) as positive sign) and \([\Phi]\) with \(-[\Phi \circ A]\) 
if the permutation has negative sign. 

\begin{definition}
  Given the definition above, we define the group homomorphism 
  \(\partial_n : C_n(X) \to C_{n - 1}(X)\) by extending
  \[\partial_n ([\Phi]) := \sum_{i = 0}^n (-1)^i [\Phi|_{\partial_i \Delta_n}]\]
  linearly.
\end{definition}

\begin{proposition}
  For all \(n \ge 1\), \(\partial_{n - 1} \circ \partial_n = 0\).
\end{proposition}
\begin{proof}
  Same proof as simplicial homology.
\end{proof}

\begin{corollary}
  \((C_n(X), \partial_n)\) is a chain complex.
\end{corollary}

\begin{definition}[Singular Homology]
  The singular homology groups of the topological space \(X\) are the homology 
  groups of the chain complex \((C_n(X), \partial_n)\), i.e. 
  \[H_n(X) := \ker(\partial_{n}) / \text{Im}(\partial_{n + 1}).\]
\end{definition}

\begin{definition}[Induced Map]
  If \(f : X \to Y\) is a continuous map and \(\Phi : \Delta_n \to X \in C_n(X)\). 
  Then, \(f \circ \Phi : \Delta_n \to Y \in C_n(Y)\). Thus, \(f\) induces 
  a group homomorphism \(f_* : H_n(X) \to H_n(Y)\).
\end{definition}

With the above definitions, we may introduce a notion known as 
relative homology. 

Let \(A \subseteq X\) and we can consider for all \(k\),
\(C_k(A) \subseteq C_k(X)\). We observe 
\(\partial_k(C_k(A)) \subseteq C_{k - 1}(A)\) implying \(\partial_k\) induces 
a group homomorphism 
\[\partial_k : C_k(X) / C_k(A) \to C_{k - 1}(X) / C_{k - 1}(A).\]
With this in mind, we denote \(C_k(X, A) := C_k(X) / C_k(A)\). 

\begin{proposition}
  \((C_k(X, A), \partial_k)_{k \in \mathbb{N}}\) is a chain complex.
\end{proposition}
\begin{proof}
  Follows as \(\partial_{n - 1} \circ \partial_n = 0\) as maps from 
  \(C_n(X) \to C_{n - 2}(X)\).
\end{proof}

\begin{definition}[Relative Homology]
  The homology groups of \(X\) relative to \(A\) are the homology groups 
  of the chain complex \((C_k(X, A), \partial_k)_{k \in \mathbb{N}}\).
  We denote them by \(H_n(X, A)\).
\end{definition}

Now, by denoting \(i : A \hookrightarrow X\) as the inclusion map, 
\(\pi : C_n(X) \to C_n(X) / C_n(A)\) the quotient map, we have induced maps 
\(i_* : H_n(A) \to H_n(X)\) and \(\pi_* : H_n(X) \to H_n(X, A)\). These 
two maps are related with something called a long exact sequence.

\begin{definition}[Exact Sequence]
  Given groups \(A, B, C\) and group homomorphisms \(f : A \to B, g : B \to C\), 
  the chain it forms is an exact sequence if
  \[\begin{tikzcd}
    A & B & C
    \arrow["f", from=1-1, to=1-2]
    \arrow["g", from=1-2, to=1-3]
  \end{tikzcd}\]
  \(\text{Im}(f) = \ker(g)\).
\end{definition}

\begin{definition}[Long Exact Sequence]
  Let \((A_i)_{i \in \mathbb{N}}\) be a sequence of groups and let 
  \(f_i : A_{i + 1} \to A_i\) be the group homomorphisms. Then, the chain 
  \[\begin{tikzcd}
    \cdots & {A_{i + 1}} & {A_i} & {A_{i - 1}} & \cdots
    \arrow["{f_i}", from=1-2, to=1-3]
    \arrow["{f_{i + 1}}", from=1-1, to=1-2]
    \arrow["{f_{i - 1}}", from=1-3, to=1-4]
    \arrow["{f_{i - 2}}", from=1-4, to=1-5]
  \end{tikzcd}\]
  is exact if for all \(i\), \(\text{Im}(f_i) = \ker(f_i)\).
\end{definition}

Now, for all \(u \in H_n(X, A) = \ker(\partial_n) / \text{Im}(\partial_{n + 1})\),
\(u\) is represented by some \(c \in C_n(X)\) such that 
\(\partial_n c \in C_{n - 1}(A)\). With this in mind, we may define the 
group homomorphism 
\[\partial : H_n(X, A) \to H_{n - 1}(A) : [c] \mapsto [\partial_n c].\]

\begin{theorem}
  The chain
  \[\begin{tikzcd}
    \cdots & {H_{n + 1}(X, A)} & {H_n(A)} & {H_n(X)} & {H_n(X, A)} & \cdots
    \arrow["\pi_*", from=1-1, to=1-2]
    \arrow["\partial", from=1-2, to=1-3]
    \arrow["{i_*}", from=1-3, to=1-4]
    \arrow["{\pi_*}", from=1-4, to=1-5]
    \arrow["\partial", from=1-5, to=1-6]
  \end{tikzcd}\]
  is a long exact sequence.
\end{theorem}
\begin{proof}
  We will first show \(\text{Im}(\partial) = \ker(i_*) \subseteq H_n(A)\). 
  Indeed, for all \([\partial_{n + 1} c] \in \text{Im}(\partial)\), 
  \([\partial_{n + 1} c] \in \ker(\partial_n)\) (as \(\partial_n \circ \partial_{n + 1} = 0\)).
  Thus, as \(\partial_{n + 1}c \in \text{Im}(\partial_{n + 1})\), we have 
  \([\partial_{n + 1} c] = 0\) implying \(\text{Im}(\partial) \subseteq \ker(i_*)\).
  Conversely, if \(a \in \ker(i_*)\), \(a \in \text{Im}(\partial_{n + 1})\), 
  implying there exists some \(c \in C_{n + 1}(X)\) such that 
  \(a = \partial_{n + 1} c = \partial c\). Thus, 
  \(\text{Im}(\partial) = \ker(i_*)\) as required. 

  The other two equalities are straight forward checks also.
\end{proof}

\end{document}

% Options for packages loaded elsewhere
\PassOptionsToPackage{unicode}{hyperref}
\PassOptionsToPackage{hyphens}{url}
\PassOptionsToPackage{dvipsnames,svgnames*,x11names*}{xcolor}
%
\documentclass[]{article}
\usepackage{lmodern}
\usepackage{amssymb,amsmath}
\usepackage{ifxetex,ifluatex}
\ifnum 0\ifxetex 1\fi\ifluatex 1\fi=0 % if pdftex
  \usepackage[T1]{fontenc}
  \usepackage[utf8]{inputenc}
  \usepackage{textcomp} % provide euro and other symbols
\else % if luatex or xetex
  \usepackage{unicode-math}
  \defaultfontfeatures{Scale=MatchLowercase}
  \defaultfontfeatures[\rmfamily]{Ligatures=TeX,Scale=1}
\fi
% Use upquote if available, for straight quotes in verbatim environments
\IfFileExists{upquote.sty}{\usepackage{upquote}}{}
\IfFileExists{microtype.sty}{% use microtype if available
  \usepackage[]{microtype}
  \UseMicrotypeSet[protrusion]{basicmath} % disable protrusion for tt fonts
}{}
\makeatletter
\@ifundefined{KOMAClassName}{% if non-KOMA class
  \IfFileExists{parskip.sty}{%
    \usepackage{parskip}
  }{% else
    \setlength{\parindent}{0pt}
    \setlength{\parskip}{6pt plus 2pt minus 1pt}}
}{% if KOMA class
  \KOMAoptions{parskip=half}}
\makeatother
\usepackage{xcolor}\pagecolor[RGB]{28,30,38} \color[RGB]{213,216,218}
\IfFileExists{xurl.sty}{\usepackage{xurl}}{} % add URL line breaks if available
\IfFileExists{bookmark.sty}{\usepackage{bookmark}}{\usepackage{hyperref}}
\hypersetup{
  pdftitle={Functional Analysis},
  pdfauthor={Kexing Ying},
  colorlinks=true,
  linkcolor=Maroon,
  filecolor=Maroon,
  citecolor=Blue,
  urlcolor=red,
  pdfcreator={LaTeX via pandoc}}
\urlstyle{same} % disable monospaced font for URLs
\usepackage[margin = 1.5in]{geometry}
\usepackage{graphicx}
\makeatletter
\def\maxwidth{\ifdim\Gin@nat@width>\linewidth\linewidth\else\Gin@nat@width\fi}
\def\maxheight{\ifdim\Gin@nat@height>\textheight\textheight\else\Gin@nat@height\fi}
\makeatother
% Scale images if necessary, so that they will not overflow the page
% margins by default, and it is still possible to overwrite the defaults
% using explicit options in \includegraphics[width, height, ...]{}
\setkeys{Gin}{width=\maxwidth,height=\maxheight,keepaspectratio}
% Set default figure placement to htbp
\makeatletter
\def\fps@figure{htbp}
\makeatother
\setlength{\emergencystretch}{3em} % prevent overfull lines
\providecommand{\tightlist}{%
  \setlength{\itemsep}{0pt}\setlength{\parskip}{0pt}}
\setcounter{secnumdepth}{5}
\usepackage{tikz}
\usepackage{physics}
\usepackage{amsthm}
\usepackage{mathtools}
\usepackage{esint}
\usepackage[ruled,vlined]{algorithm2e}
\usepackage{tikz-cd}
\theoremstyle{definition}
\newtheorem{theorem}{Theorem}
\newtheorem{definition*}{Definition}
\newtheorem*{remark}{Remark}
\theoremstyle{definition}
\newtheorem{definition}{Definition}[section]
\newtheorem{lemma}{Lemma}[section]
\newtheorem{proposition}{Proposition}[section]
\newtheorem{corollary}{Corollary}[section]
\newtheorem{example}{Example}[section]
\newcommand{\diag}{\mathop{\mathrm{diag}}}
\newcommand{\Arg}{\mathop{\mathrm{Arg}}}
\newcommand{\hess}{\mathop{\mathrm{Hess}}}

\title{Functional Analysis}
\author{Kexing Ying}
\date{July 24, 2021}

\begin{document}
\maketitle

{
\hypersetup{linkcolor=}
\setcounter{tocdepth}{2}
\tableofcontents
}
\newpage

\section{Introduction}

We have thus far looked at abstract vector spaces in linear algebra and 
(metric) topological spaces in topology. In this course, we will combine these 
concepts and study linear metric space. In particular, we will study 
vector spaces equipped with a topology such that certain properties are 
satisfies. 

In this course, we will often study the space of functions and hence 
the name of the course. As we have seen before, given that the codomain space 
possesses a certain structure, it is possible to define point-wise addition and 
scalar multiplications on functions, and thus, possible to equip the space 
with a vector space structure. 

Let us recall some definitions.

\begin{definition}[Metric]
  A metric \(\rho\) on a non-empty set \(X\) is a function with type signature 
  \(X \times X \to \mathbb{R}^+\) such that for all \(x, y, z \in X\)
  \begin{itemize}
    \item \(\rho(x, y) = 0 \iff x = y\);
    \item \(\rho(x, y) = \rho (y, x)\);
    \item \(\rho(x, y) \le \rho(x, z) + \rho(z, y)\).
  \end{itemize}
\end{definition}

\begin{definition}[Translation Invariant]
  A metric space \((V, \rho)\) where \(V\) is equipped with the binary operation 
  \((+) : V \times V \to V\) is translational invariant if for all \(w, z, v \in V\), 
  \[\rho(w + v, z + v) = \rho(w, z).\]
\end{definition}

\begin{definition}[Norm]
  A norm \(\|\cdot\|\) on the vector space \(V\) (over the field 
  \(\mathbb{K}\) equipped with a modulus \(|\cdot|\)) 
  is a function with type signature \(X \to \mathbb{R}^+\) such that for all 
  \(x, y \in V, k \in \mathbb{K}\),
  \begin{itemize}
    \item \(\|x\| = 0 \iff x = 0\);
    \item \(\|k \cdot x\| = |k| \|x\|\);
    \item \(\|x + y\| \le \|x\| + \|y\|\).
  \end{itemize} 
\end{definition}

We recall that a norm induces a metric by defining \(\rho(x, y) = \|x - y\|\). 
In this case, it is possible to show that \((+)\) and \((\cdot)\) are 
continuous with respect to this metric and \(\rho\) is translational invariant. 

\begin{definition}[Banach Space]
  A normed space is said to be a Banach space if it is complete, i.e. every 
  Cauchy sequence converge. 
\end{definition}

\begin{definition}[Separable]
  A topological space is said to be separable if there exists a dense 
  countable subset.
\end{definition}

As we shall see, for \(0 < 0 < \infty\), \(\ell_p\) is separable while 
\(\ell_\infty\) is not.

\begin{definition}[Compact]
  A topological space is said to be compact if every open cover has a finite 
  sub-cover. 
\end{definition}

Unlike what we have seen before, as we consider infinite dimensional spaces, 
we will see that the Heine-Borel property will no longer hold, i.e.
closed and bounded is no longer equivalent to compact.

\newpage

\section{Linear Spaces}

\begin{definition}[Equivalent Norms and Metrics]
  Two norms \(\|\cdot\|_k\) for \(k = 1, 2\) are said to be equivalent if there 
  exists some \(M > 0\) such that for all \(x\), 
  \[\frac{1}{M} \|x\|_1 \le \|x\|_2 \le M \|x\|_1.\]
  Similarly, two metrics \(\rho_k\) are said to be equivalent if there exists
  some \(M > 0\) such that for all \(x, y\), 
  \[\frac{1}{M} \rho_1(x, y) \le \rho_2(x, y) \le M \rho_1(x, y).\]
\end{definition}

It is clear that equivalent is a symmetric relation and as we have seen before,
all norms on a finite dimensional space are equivalent.

\begin{definition}[Concave and Convex Function]
  A function \(f : V \to \mathbb{R}\) is 
  \begin{itemize}
    \item concave if for all \(s \in [0, 1], x, y \in V\), we have  
      \[sf(x) + (1 - s)f(y) \le f(sx + (1 - s)y);\]
    \item convex if for all \(s \in [0, 1], x, y \in V\), we have  
      \[sf(x) + (1 - s)f(y) \ge f(sx + (1 - s)y).\] 
  \end{itemize}
\end{definition}

\begin{proposition}
  If \(f : \mathbb{R}^+ \to \mathbb{R}^+\) is concave and \(f(0) = 0\). Then 
  \[f(x + y) \le f(x) + f(y).\]
\end{proposition}
\begin{proof}
  Clear by taking \(s = \frac{y}{x + y}\), we have 
  \[(1 - s)f(x + y) = s f(0) + (1 - s)f(x + y) \le f(s \cdot 0 + (1 - s)(x + y))
    = f(x),\]
  and 
  \[sf(x + y) = sf(x + y) + (1 - s)f(0) \le f(s(x + y) + (1 - s) \cdot 0) = 
    f(y).\]
  Adding the two equations, we have
  \[f(x + y) = (1 - s)f(x + y) + sf(x + y) \le f(x) + f(y).\]
\end{proof}

\begin{corollary}
  If \(\rho\) is a metric and \(\eta : \mathbb{R}^+ \to \mathbb{R}^+\) is a 
  concave and vanishing at \(0\), then \(\rho \circ \eta\) is also a metric.
\end{corollary}

\begin{definition}[Linear Metric Space]
  A vector space \(V\) over the field \(\mathbb{K}\) equipped with a metric 
  \(\rho\) on \(V\) and a metric \(|\cdot - \cdot|\) on \(\mathbb{K}\) is a 
  linear metric space if \((+) : V \times V \to V\) and 
  \((\cdot) : \mathbb{K} \times V \to V\) are continuous with respect to the 
  induced metric.
\end{definition}

\begin{proposition}
  Any normed space is a linear metric space.
\end{proposition}
\begin{proof}
  Let \((x_n, y_n) \to (x, y)\) in \(V^2\), then we have 
  \[\|(x_n + y_n) - (x + y)\| \le \|x_n - x\| + \|y_n - y\| \to 0.\]
  Thus, \((+)\) is continuous. 
  
  Similarly, if \((\lambda_n, x_n) \to (\lambda, x)\) in \(\mathbb{K} \times V\),
  \[\begin{split}
    \|\lambda_n x_n - \lambda x\| 
      & = \|\lambda_n x_n - \lambda_n x + \lambda_n x - \lambda x\| \\
      & \le \|\lambda_n x_n - \lambda_n x\| + \|\lambda_n x - \lambda x\| \\
      & = |\lambda_n| \|x_n - x\| + |\lambda_n - \lambda| \|x\|.
  \end{split}\]
  Now, since \((\lambda_n)\) is convergent, it is bounded by some \(M > 0\) and 
  thus, 
  \[\|\lambda_n x_n - \lambda x\| 
    \le |\lambda_n| \|x_n - x\| + |\lambda_n - \lambda| \|x\|
    \le M \|x_n - x\| + |\lambda_n - \lambda| \|x\| \to 0,\]
  implying \((\cdot)\) is continuous.
\end{proof}

\subsection{Classical Spaces}

We recall the \(L_p\) spaces from second year measure theory, and in particular, 
when we consider the counting measure \(\mu\), we have the nice property that 
\[\int f \dd \mu = \sum_{n = 0}^\infty f(n),\]
and we no longer require a quotient to define the linear space as the only 
null-set is the empty set (thus, two function are a.e equal if and only if they 
are equal). In this special case, we call the resulting space \(\ell_p\) with 
the \(p\)-norm 
\[\|f\|_p = \left(\int |f|^p \dd \mu\right)^{\frac{1}{p}} = 
  \left(\sum_{n = 0}^\infty |f(n)|^p \right)^{\frac{1}{p}}.\]
We will use the sequence notation and write \(a_n := a(n)\) for \(a \in \ell_p\).

As this is simply a special case of the \(L_p\) space, the inequalities proved 
on the \(L_p\) space remains. We will recall them here for \(\ell_p\) spaces.

\begin{proposition}[Hölder's Inequality]
  Let \(\frac{1}{p} + \frac{1}{q} = 1\) where \(p, q \in (1, \infty)\). Then 
  for \(a = (a_i)_{i \in \mathbb{N}}, b = (b_i)_{i \in \mathbb{N}} \in \ell_p\), 
  we have 
  \[|\langle a, b\rangle| \le \|a\|_p \|b\|_q,\] 
  where \(\langle a, b\rangle := \sum_{i \in \mathbb{N}} a_i b_i\).
\end{proposition}

\begin{proposition}[Minkowski's Inequality]
  Let \(a, b \in \ell_p\) for some \(1 \le p \le \infty\). 
  Then \(a + b \in \ell_p\) and
  \[\|a + b\|_p \le \|a\|_p + \|b\|_p.\]
\end{proposition}

\begin{lemma}
  Let \(I \subseteq \mathbb{R}\) be an interval and 
  \(\phi : I \to \mathbb{R}^+\) be a function. 
  Then \(\phi\) is convex if and only if for all \(y \in I\), there exists 
  a \(\gamma \in \mathbb{R}\), such that for all \(x \in I\),
  \[\gamma(x - y) \le \phi(x) - \phi(y).\]
\end{lemma}

\begin{proposition}[Jensen's Inequality]
  let \(\phi \ge 0\) be convex and suppose \(\sum_{i \in \mathbb{N}} \eta_i = 1\), 
  \(|\langle \alpha \rangle| < \infty\) and \(\langle \phi(\alpha) \rangle\) 
  (where \(\langle \beta \rangle = \sum_{i \in \mathbb{N}} \eta_i \beta_i\)), 
  then 
  \[\phi(\langle \alpha \rangle) \le \langle \phi(\alpha) \rangle.\]
\end{proposition}
\begin{proof}
  By the above lemma, there exists some \(\gamma\) such that 
  \[\gamma(\alpha_j - \langle \alpha \rangle) \le 
    \phi(\alpha_j) - \phi(\langle \alpha \rangle).\]
  Thus, 
  \[\begin{split}
    0 = \sum \eta_i \gamma(\alpha_j - \langle \alpha \rangle) 
      & \le \sum \eta_i (\phi(\alpha_j) - \phi(\langle \alpha \rangle)) \\
      & = \sum \eta_i \phi(\alpha_j) - \phi(\langle \alpha \rangle) \sum \eta_i \\
      & = \langle \phi(\alpha_j) \rangle - \phi(\langle \alpha \rangle)
    \end{split}\]
  where the first equality follows as \(\gamma\) is independent of the index 
  \(i\).
\end{proof}

\begin{proposition}
  For \(p < p'\), \(\ell_p \subseteq \ell_{p'}\). On the other hand, if 
  \(\sum \eta_i = 1\), we have \(\ell_p(\eta) \supseteq \ell_{p'}(\eta)\).
\end{proposition}

\begin{definition}
  Let \(\phi\) be a convex function such that \(\phi(0) = 0\), 
  \(\phi(x) \to \infty\) as \(x \to \infty\). If \(\phi\) has the doubling 
  property such that there exists some \(M > 0\) such that for all 
  \(x \in \mathbb{R}\), \(\phi(2 |x|) \le M \phi(|x|)\), then 
  \[V := \left\{a : \mathbb{N} \to \mathbb{R} \mid \sum \eta_i 
    \phi(|a_i|) < \infty\right\}\]
  where \(\sum \eta_i = 1\) is a vector space with point-wise operations.
\end{definition}

\begin{proposition}
  Given a metric space \((X, \rho)\), there exists a complete metric space 
  \((\tilde X, \tilde \rho)\) and an isometric embedding 
  \(\iota : X \to \tilde X\) such that for all \(x, x' \in X\),
  \(\tilde \rho(\iota(x), \iota(x')) = \rho(x, x')\).
\end{proposition}
\begin{proof}
  We have seen similar ideas in the completion of \(\mathbb{Q}\) though 
  completing the space with equivalence classes of mutually Cauchy sequences 
  as elements of \(\tilde X\).
\end{proof}

As we have seen last year, the \(L_p\) spaces are complete, and thus are 
Banach spaces. Thus, we have \(\ell_p\) spaces are also complete and are 
Banach spaces. In fact, the proof that \(\ell_p\) spaces are complete is 
easier, in that one may show completeness through showing the point-wise 
limit of the sequences indeed belong to \(\ell_p\).

\begin{definition}
  We define 
  \begin{itemize}
    \item \(c_0 := \{x \in \ell_\infty \mid \lim_{n \to \infty} x_n = 0\}\),
    \item \(c := \{x \in \ell_\infty \mid \exists \lim_{n \to \infty} x_n\}\),
  \end{itemize}
  be subspaces of \(\ell_\infty\).
\end{definition}

\begin{proposition}
  \(c_0\) is complete.
\end{proposition}
\begin{proof}
  Let \((x_i^n) \subseteq c_0\) be a Cauchy sequence and let 
  \(x_i = \lim_{n \to \infty} x_i^n\), then it suffices to show 
  \(\lim_{i \to \infty} x_i = 0\). 

  Since \((x_i^n)\) is Cauchy, for all \(\epsilon > 0\), there exists some
  \(N \in \mathbb{N}\) such that for all \(n \ge N\), 
  \[\|x^n - x^N\|_\infty < \frac{\epsilon}{2}.\]
  Furthermore, as \(x_i^N \to 0\) as \(i \to \infty\), there exists some 
  \(I \in \mathbb{N}\) such that for all \(i \ge I\), 
  \(|x_i^N| < \epsilon / 2\). Thus, for all \(i \ge I\), we have 
  \[\frac{\epsilon}{2} > \|x^n - x^N\|_\infty > |x_i^n - x_i^N| >
    |x_i^n| - \frac{\epsilon}{2} \implies \epsilon > |x_i^n|,\]
  for all \(n \ge N\). Hence, taking \(n \to \infty\), we have 
  \[\epsilon > \lim_{n \to \infty} |x_i^n| = |x_i|,\]
  implying \(x_i \to 0\) as \(i \to \infty\).
\end{proof}

\begin{proposition}
  \(c\) is complete.
\end{proposition}
\begin{proof}
  Similar to above, let \((x_i^n)\) be Cauchy and define \((x^n)\) such that for 
  all \(n \in \mathbb{N}\), \(x^n = \lim_{i \to \infty} x_i^n\) and  
  \((x_i)\) such that \((x_i^n) \to (x_i)\) in \(\ell_\infty\). It is not 
  difficult to see that \((x^n)\) is Cauchy and thus converges to some \(x\).
  \[\begin{tikzcd}
    (x_i^n) \arrow{r}{n \to \infty} 
    \arrow[swap]{d}{i \to \infty} & (x_i) \arrow{d}{i \to \infty} \\
    (x^n) \arrow{r}{n \to \infty}& x
    \end{tikzcd}\]
  Indeed, as \((x_i^n)\) is Cauchy, for all \(\epsilon > 0\), there exists some 
  \(N \in \mathbb{N}\) such that for all 
  \(p, q \ge N\), \(\|x_i^p - x_i^q\|_\infty < \epsilon / 3\).
  Furthermore, there exists some \(I \in \mathbb{N}\) such that for all 
  \(j \ge I\), \(|x_j^p - x^p|, |x_j^q - x^q| < \epsilon / 3\). Thus, 
  \[|x^p - x^q| < |x_j^p - x^p| + |x_j^p - x_j^q| + |x_j^q - x^q| 
    < \frac{\epsilon}{3} + \|x_i^p - x_i^q\|_\infty + \frac{\epsilon}{3} 
    = \epsilon.\]
  Hence, it remains to show \(x_i \to x\) as \(i \to \infty\). 
  Fix \(\epsilon > 0\), then there exists \(N \in \mathbb{N}\) such that for 
  all \(n \ge N\), \(|x^n - x| < \epsilon / 3\). Furthermore, there exists 
  \(M \in \mathbb{N}\) such that for all \(n \ge M\), 
  \(\|x_i^n - x_i\|_\infty < \epsilon / 3\). Then, for all 
  \(p \ge \max\{N, M\}\), there exists \(I \in \mathbb{N}\) such that 
  for all \(j \ge I\), \(|x_j^p - x^p| < \epsilon / 3\). Finally, 
  \[|x_j - x| \le |x_j - x_j^p| + |x_j^p - x^p| + |x^p - x| \le 
    \|x_i - x_i^p\|_\infty + \frac{\epsilon}{3} + \frac{\epsilon}{3} < \epsilon,\]
  implying \(x_i \to x\) as \(i \to \infty\) and \((x_i) \in c\).
\end{proof}

\subsubsection{Separability}

In this section we will consider the separability of different classical spaces.

\begin{proposition}
  The space \(\ell_p\) is separable for all \(p \ge 1\).
\end{proposition}
\begin{proof}
  It is easy to see that the subspace of all finite sequences is dense in 
  \(\ell_p\). Indeed, if \(x \in \ell_p\) and \(x^n\) is the sequence such that 
  \(x^n_i = x_i\) for all \(i \le n\) and \(x^n_i = 0\) for all \(i > n\), 
  we have 
  \[\|x - x^n\|_p^p = \sum_{i = 0}^\infty |x_i - x_i^n|^p = 
    \sum_{i = n + 1}^\infty |x_i|^p\]
  which tends to \(0\) as \(n \to \infty\). Now, by defining 
  \(\ell_{\mathbb{Q}, n}\) as the set of rational sequences with length \(n\),
  we have \(d := \bigcup_{n \in \mathbb{N}} \ell_{\mathbb{Q}, n}\) is dense in the 
  space of all finite sequences. Now, since \(d\) is countable as it is a countable 
  union of countable sets, we have found a countable dense set of \(\ell_p\).
\end{proof}

\begin{lemma}
  A metric space is \(X\) not separable if there exists an uncountable subset \(S\) 
  such that for some some \(k > 0\), \(d(x, y) > k\) for all \(x, y \in S\).
\end{lemma}
\begin{proof}
  As subset of a separable \textit{metric} space is separable, \(S\), must be 
  separable. But as \(d(x, y) > k\) for all \(x, y \in S\), no sequences of 
  \(S\) but the constant sequence can converge. Thus, for all countable subsets 
  of \(S\), simply picking a point of \(S\) not in that subset suffices.
\end{proof}

\begin{proposition}
  \(\ell_\infty\) is not separable.
\end{proposition}
\begin{proof}
  Clearly the set containing all sequences of only \(0\) and \(1\)s is 
  a subset of \(\ell_\infty\). Furthermore, this sequence is uncountable as 
  it bijects \(\mathbb{R}\) be considering the binary representation of a real 
  number. Thus, since all distinct elements in this set have distance 1 apart, 
  the conclusion follows by the above lemma.
\end{proof}

\begin{proposition}
  \(C^k([a, b])\) (the set of \(k\)-differentiable functions from the interval 
  \([a, b]\)) for all \(k \in \mathbb{N}\) is separable.
\end{proposition}
\begin{proof}
  Recalling the Weierstass approximation theorem, we have for all continuous 
  function \(f\), there exists a sequence of polynomials \(f_n\) such that 
  \(f_n \to f\) uniformly. Thus, as the set of polynomials with rational 
  coefficients is countable, and dense in the space of all polynomials, we 
  have found a countable dense set of \(C^0([a, b])\). Now as 
  \(C^k([a, b]) \subseteq C^0([a, b])\) for all \(k \in \mathbb{N}\), we have 
  \(C^k([a, b])\) is separable as required. 
\end{proof}

\begin{definition}[Absolutely Convergent]
  Let \(X\) be a normed space and let \((x_n)\) be a sequence of \(X\), then 
  a series \(\sum_{n \in \mathbb{N}} x_n\) is said to be absolutely convergent 
  if \(\sum_{n \in \mathbb{N}} \|x_n\| < \infty\).
\end{definition}

\begin{proposition}
  A normed space \(X\) is complete if and only if for all sequences \((x_n)\) 
  of \(X\) such that \(\sum x_n\) is absolutely convergent implies \(x_n\) 
  converges. Thus, a normed space is Banach if and only if this property is 
  satisfied.
\end{proposition}
\begin{proof}
  See problem sheet.
\end{proof}

By recalling the proof of the completeness of \(L_p\) spaces, we note that 
this property was used extensively. As a consequence, we see that \(L_p(\mu)\) 
is separable if the measure \(\mu\) is separable. In particular, the spaces 
\(L_p(\mathbb{R}^n, \lambda)\) are separable for all \(n\). On the other hand, 
\(L_\infty\) is in general not separable.

\subsection{Hamel and Schauder Basis}

In this small section we will introduce two new notions of basis for infinite 
dimensional spaces. In particular, we will introduce the Hamel basis which is 
a natural extension of the definition of basis for the finite dimensional case 
and the Schauder basis which is a notion of basis that incorporates the 
topological properties of the space.

\begin{definition}[Linear Independent]
  A set \(W \subseteq V\) is linear independent if for all 
  \((\lambda_i)_{i = 1}^m \subseteq \mathbb{K}\), \((w_i)_{i = 1}^n \subseteq W\), 
  \[\sum \lambda_i w_i = 0 \implies \lambda_i = 0\]
  for all \(i = 1, \cdots, n\).
\end{definition}

\begin{definition}[Hamel Basis]
  A set \(W \subseteq V\) is a Hamel basis for a linear space \(V\) if 
  \(W\) is linearly independent and for all \(x \in V\), there exists a unique 
  finite linear combination of vectors \((w_i)_{i = 1}^n \subseteq W\), 
  \((\lambda_i)_{i = 1}^m \subseteq \mathbb{K}\) 
  such that 
  \[x = \sum \lambda_i w_i.\]
\end{definition}

\begin{proposition}
  Every linear space has a Hamel basis.
\end{proposition}
We will come back to the proof of this proposition after discussing an important 
result known as the Hahn-Banach theorem.

\begin{definition}[Schauder Basis]
  A set \(W \subseteq V\) is a Schauder basis for a normed space \(V\) if 
  \(W\) is countable, linearly independent, and for all \(x \in V\), there 
  exists a unique (possibly infinite)-sequence 
  \((\lambda_i)_{i = 1}^\infty \subseteq \mathbb{K}\) and 
  \((w_i)_{i = 1}^\infty \subseteq W\) such that
  \[x = \sum_{i = 1}^\infty \lambda_i w_i.\]
\end{definition}

\begin{proposition}
  If a Banach space \(X\) has a Schauder basis, then it is separable.
\end{proposition}

The proof of the above proposition is left as an exercise. It is notable that 
the reverse of the above is not true and was in fact a Scottish book problem 
which an counterexample was given by Per Enflo in 1972 who was awarded a live 
goose.

It is clear that for \(p \ge 1\), the set \(W := \{e_j \mid j \in \mathbb{N}\}\) 
where \((e_j)_i = \delta_{ij}\) is a Schauder basis of \(\ell_p\). 

Recall that \(c \subseteq \ell_\infty\) is the set of sequences which has 
a limit. By considering the sequence \(x_n = 1\), we see that the set of 
standard basis vectors no longer form a basis of \(c\) as 
\[\left\|x - \sum^n_{i = 1} e_i\right\|_\infty = 1,\]
for all \(n\). Defining \(W := \{e_0 := (1, 1, \cdots)\} \cup 
\{e_j \mid j \in \mathbb{N}\}\), for all \((a_n) \in c\), let 
\(\lambda_0 := \lim_{n \to \infty} a_n\) and \(\lambda_n = a_n - \lambda_0\).
Then, for all \(\epsilon > 0\), there exists some \(N \in \mathbb{N}\) such that 
for all \(n \ge N\), \(|a_n - \lambda_0| \leq \epsilon\). Thus, 
\[\left\|\lambda_0 e_0 + \sum_{i = 1}^n \lambda_i e_i - a\right\|_\infty 
  = \|(0, \cdots, 0, \lambda_0 - a_n, \lambda_0 - a_{n + 1}, \cdots)\|_\infty
  < \epsilon,\]
implying \(W\) is a Schauder basis of \(c\).

Furthermore, one may show that \(C([0, 1])\) has a Schauder basis consisting of 
all the ``spike'' functions at the points \(k2^{-n}\) for all \(n \in \mathbb{N}\), 
\(k = 1, \cdots 2^{-n}\).

\subsection{Hilbert Spaces}

Recall the definition of sesquilinear forms over some vector space \(\mathbb{H}\).

\begin{definition}[Sesquilinear Form]
  A sesquilinear form \(\langle \cdot, \cdot \rangle : 
  \mathbb{H} \times \mathbb{H} \to \mathbb{K}\) on \(\mathbb{H}\) where 
  \(\mathbb{K}\) is \(\mathbb{R}\) or \(\mathbb{C}\) is a function such that
  \begin{itemize}
    \item it is linear with respect to the second argument;
    \item and is conjugate symmetric.
  \end{itemize}
  We say \(\langle \cdot, \cdot \rangle\) is nondegenerate if \(x = 0\) if and 
  only if \(\langle x, y\rangle = 0\) for all \(y \in \mathbb{H}\). 
  Nondegenerate sesquilinear forms are called scalar products and a 
  vector space equipped with a scalar product is called a unitary space 
  (or a scalar product space).
\end{definition}

We see that a scalar product induces a norm by defining 
\(\|f\|^2 := \langle f, f \rangle\). In particular, we recall the 
\(\ell_2, C([a, b])\) and \(L_2\) are all scalar product spaces.

\begin{proposition}
  Let \(\mathbb{H}\) be a unitary space, then 
  \begin{itemize}
    \item \(\|f + g\|^2 + \|f - g\|^2 = 2 \|f\|^2 + 2\|g\|^2\) 
      (parallelogram identity);
    \item \(\langle f, g\rangle = \frac{1}{4}(\|f + g\|^2 - |f - g\|^2)\) if 
      \(\mathbb{K} = \mathbb{R}\) and 
      \(\langle f, g\rangle = \frac{1}{4} \sum_{k = 0, \cdots 3} 
      i^k \|f + i^k g\|^2\) if \(\mathbb{K} = \mathbb{C}\)
      (polarisation identity).
  \end{itemize}
\end{proposition}

\begin{definition}[Hilbert Space]
  A unitary space is a Hilbert space if it is complete with respect to its 
  induced norm.
\end{definition}

\begin{definition}[Convex Set]
  A set \(S\) is said to be convex if for all \(x, y \in S\), 
  \((1 - t)x + ty \in S\) for all \(t \in [0, 1]\).
\end{definition}

\begin{proposition}[Nearest Point Property]
  Every nonempty closed convex set \(\mathcal{K}\) in a Hilbert space 
  \(\mathbb{H}\) contains a vector of the smallest norm. Moreover, if 
  \(h \in \mathbb{H}\), there exists a unique \(h_0 \in \mathcal{K}\) 
  such that 
  \[\|h - h_0\| = \text{dist}(h, \mathcal{K}) := \inf_{k \in K} \|h - k\|.\]
\end{proposition}
\begin{proof}
  By the definition of infimum, there exists a sequence 
  \((k_n) \subseteq \mathcal{K}\) such that 
  \[\lim_{n \to \infty} \|k_n\| \to d := \inf_{k \in \mathbb{K}} \|k\|.\]
  Consider, for all \(n, m \in \mathbb{N}\), by the parallelogram identity,
  \[\left\|\frac{1}{2}(k_n - k_m)\right\|^2 = 
    \frac{1}{2}(\|k_n\|^2 + \|k_m\|^2) - \left\|\frac{1}{2}(k_n + k_m)\right\|^2.\]
  Then, as \(\mathbb{K}\) is convex, \(\frac{1}{2}(k_n + k_m) \in \mathbb{K}\) 
  and so \(\|(k_n + k_m) / 2\|^2 \ge d^2\) and 
  \[\left\|\frac{1}{2}(k_n - k_m)\right\|^2 \le 
    \frac{1}{2}(\|k_n\|^2 + \|k_m\|^2) - d^2.\]
  Thus, taking \(n, m \to \infty\), the right hand side tends to zero and so 
  \((k_n)\) is Cauchy and hence convergent. Now as \(\mathcal{K}\) is closed, 
  the limit is in \(\mathcal{K}\) and hence the statement.
\end{proof}

\begin{definition}[Orthogonal Systems]
  Let \((\mathbb{H}, \langle \cdot, \cdot \rangle)\) be an unitary space. 
  A set of vectors \(\{e_j \in \mathbb{H} \mid j \in J\}\) for some index 
  set \(J\) is called an orthogonal system if for all distinct \(i, j \in J\), 
  \[\langle e_i, e_j \rangle = 0.\]
  If furthermore, \(\langle e_i, e_i \rangle = 1\) for all \(i \in J\), then 
  we say the system is orthonormal.
\end{definition}

\begin{proposition}
  Every orthogonal system is linearly independent.
\end{proposition}
\begin{proof}
  Exercise.
\end{proof}

Using Zorn's lemma, one can show that every unitary space contains an orthogonal 
basis.

\begin{definition}[Fourier Coefficients]
  Let \((\mathbb{H}, \langle \cdot, \cdot \rangle)\) be an unitary space 
  and let \(\{e_j \in \mathbb{H} \mid j \in J\}\) be an orthogonal system. 
  Then for each \(f \in \mathbb{H}\), the Fourier coefficients with respect to 
  the orthogonal system are 
  \[c_j := \frac{\langle e_j, f \rangle}{\|e_j\|^2}.\]
\end{definition} 

\begin{proposition}
  If \(f = \sum_{k \in \mathbb{N}} \alpha_k e_k\), then \(a_j = c_j\).
\end{proposition}
\begin{proof}
  Let \(n < m\) and \(S_m := \sum_{k = 1}^m \alpha_k e_k\). Then, 
  \(\langle S_m, e_n\rangle = \overline{\alpha_n} \|e_n\|^2\). Thus, we have 
  \[|\overline{\alpha_n} \|e_n\|^2 - \langle f, e_n\rangle| = 
    |\langle S_m, e_n\rangle - \langle f, e_n\rangle | = 
    |\langle S_m - f, e_n \rangle|\le\|S_m - f\| \|e_n\| \to 0\]
  as \(m \to \infty\). 
\end{proof}

\begin{proposition}
  Suppose \((e_j)_{j \in \mathbb{N}}\) is orthogonal and 
  \(g(a_1, \cdots, a_n) := \left\| f - \sum_{j = 1}^n a_je_j\right\|^2\). 
  Then \(g\) attains its minimum at \(a_j = c_j\). Furthermore, 
  we have 
  \[\sum_{j = 1}^\infty |c_j|^2\|e_k\|^2 \le \|f\|^2.\]
  This inequality is known as Bessel's inequality.
\end{proposition}
\begin{proof}
  Exercise.
\end{proof}

\begin{definition}
  An orthogonal system \(\{e_j \mid j \in J\}\) is called complete if 
  for all \(f \in \mathbb{H}\), if \(\langle f, e_j \rangle = 0\) for all 
  \(j\), then \(f = 0\).
\end{definition}

\begin{proposition}
  The following are equivalent
  \begin{enumerate}
    \item \((e_j)_{j \in \mathbb{N}}\) is complete;
    \item \(\left\|f - \sum_{j = 1}^n c_je_j\right\| \to 0\) as \(k \to \infty\) where 
      \(c_j\) are the Fourier coefficients;
    \item \(\|f\|^2 = \sum_{j = 1}^\infty |c_j|^2\|e_k\|^2\) for all 
      \(f \in \mathbb{H}\).
  \end{enumerate}
\end{proposition}
\begin{proof}
  Exercise.
\end{proof}

\end{document}

% Options for packages loaded elsewhere
\PassOptionsToPackage{unicode}{hyperref}
\PassOptionsToPackage{hyphens}{url}
\PassOptionsToPackage{dvipsnames,svgnames*,x11names*}{xcolor}
%
\documentclass[]{article}
\usepackage{lmodern}
\usepackage{amssymb,amsmath}
\usepackage{ifxetex,ifluatex}
\ifnum 0\ifxetex 1\fi\ifluatex 1\fi=0 % if pdftex
  \usepackage[T1]{fontenc}
  \usepackage[utf8]{inputenc}
  \usepackage{textcomp} % provide euro and other symbols
\else % if luatex or xetex
  \usepackage{unicode-math}
  \defaultfontfeatures{Scale=MatchLowercase}
  \defaultfontfeatures[\rmfamily]{Ligatures=TeX,Scale=1}
\fi
% Use upquote if available, for straight quotes in verbatim environments
\IfFileExists{upquote.sty}{\usepackage{upquote}}{}
\IfFileExists{microtype.sty}{% use microtype if available
  \usepackage[]{microtype}
  \UseMicrotypeSet[protrusion]{basicmath} % disable protrusion for tt fonts
}{}
\makeatletter
\@ifundefined{KOMAClassName}{% if non-KOMA class
  \IfFileExists{parskip.sty}{%
    \usepackage{parskip}
  }{% else
    \setlength{\parindent}{0pt}
    \setlength{\parskip}{6pt plus 2pt minus 1pt}}
}{% if KOMA class
  \KOMAoptions{parskip=half}}
\makeatother
\usepackage{xcolor}\pagecolor[RGB]{28,30,38} \color[RGB]{213,216,218}
\IfFileExists{xurl.sty}{\usepackage{xurl}}{} % add URL line breaks if available
\IfFileExists{bookmark.sty}{\usepackage{bookmark}}{\usepackage{hyperref}}
\hypersetup{
  pdftitle={Functional Analysis},
  pdfauthor={Kexing Ying},
  colorlinks=true,
  linkcolor=Maroon,
  filecolor=Maroon,
  citecolor=Blue,
  urlcolor=red,
  pdfcreator={LaTeX via pandoc}}
\urlstyle{same} % disable monospaced font for URLs
\usepackage[margin = 1.5in]{geometry}
\usepackage{graphicx}
\makeatletter
\def\maxwidth{\ifdim\Gin@nat@width>\linewidth\linewidth\else\Gin@nat@width\fi}
\def\maxheight{\ifdim\Gin@nat@height>\textheight\textheight\else\Gin@nat@height\fi}
\makeatother
% Scale images if necessary, so that they will not overflow the page
% margins by default, and it is still possible to overwrite the defaults
% using explicit options in \includegraphics[width, height, ...]{}
\setkeys{Gin}{width=\maxwidth,height=\maxheight,keepaspectratio}
% Set default figure placement to htbp
\makeatletter
\def\fps@figure{htbp}
\makeatother
\setlength{\emergencystretch}{3em} % prevent overfull lines
\providecommand{\tightlist}{%
  \setlength{\itemsep}{0pt}\setlength{\parskip}{0pt}}
\setcounter{secnumdepth}{5}
\usepackage{tikz}
\usepackage{physics}
\usepackage{amsthm}
\usepackage{mathtools}
\usepackage{esint}
\usepackage[ruled,vlined]{algorithm2e}
\usepackage{tikz-cd}
\theoremstyle{definition}
\newtheorem{theorem}{Theorem}
\newtheorem{definition*}{Definition}
\newtheorem*{remark}{Remark}
\theoremstyle{definition}
\newtheorem{definition}{Definition}[section]
\newtheorem{lemma}{Lemma}[section]
\newtheorem{proposition}{Proposition}[section]
\newtheorem{corollary}{Corollary}[section]
\newtheorem{example}{Example}[section]
\newcommand{\hess}{\mathop{\mathrm{Hess}}}
\newcommand{\weak}{\rightharpoonup}
% the redefinition for the missing \setminus must be delayed
\AtBeginDocument{\renewcommand{\setminus}{\mathbin{\backslash}}}

\title{Functional Analysis}
\author{Kexing Ying}
\date{July 24, 2021}

\begin{document}
\maketitle

{
\hypersetup{linkcolor=}
\setcounter{tocdepth}{2}
\tableofcontents
}
\newpage

\section{Introduction}

We have thus far looked at abstract vector spaces in linear algebra and 
(metric) topological spaces in topology. In this course, we will combine these 
concepts and study linear metric space. In particular, we will study 
vector spaces equipped with a topology such that certain properties are 
satisfies. 

In this course, we will often study the space of functions and hence 
the name of the course. As we have seen before, given that the codomain space 
possesses a certain structure, it is possible to define point-wise addition and 
scalar multiplications on functions, and thus, possible to equip the space 
with a vector space structure. 

Let us recall some definitions.

\begin{definition}[Metric]
  A metric \(\rho\) on a non-empty set \(X\) is a function with type signature 
  \(X \times X \to \mathbb{R}^+\) such that for all \(x, y, z \in X\)
  \begin{itemize}
    \item \(\rho(x, y) = 0 \iff x = y\);
    \item \(\rho(x, y) = \rho (y, x)\);
    \item \(\rho(x, y) \le \rho(x, z) + \rho(z, y)\).
  \end{itemize}
\end{definition}

\begin{definition}[Translation Invariant]
  A metric space \((V, \rho)\) where \(V\) is equipped with the binary operation 
  \((+) : V \times V \to V\) is translational invariant if for all \(w, z, v \in V\), 
  \[\rho(w + v, z + v) = \rho(w, z).\]
\end{definition}

\begin{definition}[Norm]
  A norm \(\|\cdot\|\) on the vector space \(V\) (over the field 
  \(\mathbb{K}\) equipped with a modulus \(|\cdot|\)) 
  is a function with type signature \(X \to \mathbb{R}^+\) such that for all 
  \(x, y \in V, k \in \mathbb{K}\),
  \begin{itemize}
    \item \(\|x\| = 0 \iff x = 0\);
    \item \(\|k \cdot x\| = |k| \|x\|\);
    \item \(\|x + y\| \le \|x\| + \|y\|\).
  \end{itemize} 
\end{definition}

We recall that a norm induces a metric by defining \(\rho(x, y) = \|x - y\|\). 
In this case, it is possible to show that \((+)\) and \((\cdot)\) are 
continuous with respect to this metric and \(\rho\) is translational invariant. 

\begin{definition}[Banach Space]
  A normed space is said to be a Banach space if it is complete, i.e. every 
  Cauchy sequence converge. 
\end{definition}

\begin{definition}[Separable]
  A topological space is said to be separable if there exists a dense 
  countable subset.
\end{definition}

As we shall see, for \(0 < 0 < \infty\), \(\ell_p\) is separable while 
\(\ell_\infty\) is not.

\begin{definition}[Compact]
  A topological space is said to be compact if every open cover has a finite 
  sub-cover. 
\end{definition}

Unlike what we have seen before, as we consider infinite dimensional spaces, 
we will see that the Heine-Borel property will no longer hold, i.e.
closed and bounded is no longer equivalent to compact.

\newpage

\section{Linear Spaces}

\begin{definition}[Equivalent Norms and Metrics]
  Two norms \(\|\cdot\|_k\) for \(k = 1, 2\) are said to be equivalent if there 
  exists some \(M > 0\) such that for all \(x\), 
  \[\frac{1}{M} \|x\|_1 \le \|x\|_2 \le M \|x\|_1.\]
  Similarly, two metrics \(\rho_k\) are said to be equivalent if there exists
  some \(M > 0\) such that for all \(x, y\), 
  \[\frac{1}{M} \rho_1(x, y) \le \rho_2(x, y) \le M \rho_1(x, y).\]
\end{definition}

It is clear that equivalent is a symmetric relation and as we have seen before,
all norms on a finite dimensional space are equivalent.

\begin{definition}[Concave and Convex Function]
  A function \(f : V \to \mathbb{R}\) is 
  \begin{itemize}
    \item concave if for all \(s \in [0, 1], x, y \in V\), we have  
      \[sf(x) + (1 - s)f(y) \le f(sx + (1 - s)y);\]
    \item convex if for all \(s \in [0, 1], x, y \in V\), we have  
      \[sf(x) + (1 - s)f(y) \ge f(sx + (1 - s)y).\] 
  \end{itemize}
\end{definition}

\begin{proposition}
  If \(f : \mathbb{R}^+ \to \mathbb{R}^+\) is concave and \(f(0) = 0\). Then 
  \[f(x + y) \le f(x) + f(y).\]
\end{proposition}
\begin{proof}
  Clear by taking \(s = \frac{y}{x + y}\), we have 
  \[(1 - s)f(x + y) = s f(0) + (1 - s)f(x + y) \le f(s \cdot 0 + (1 - s)(x + y))
    = f(x),\]
  and 
  \[sf(x + y) = sf(x + y) + (1 - s)f(0) \le f(s(x + y) + (1 - s) \cdot 0) = 
    f(y).\]
  Adding the two equations, we have
  \[f(x + y) = (1 - s)f(x + y) + sf(x + y) \le f(x) + f(y).\]
\end{proof}

\begin{corollary}
  If \(\rho\) is a metric and \(\eta : \mathbb{R}^+ \to \mathbb{R}^+\) is a 
  concave and vanishing at \(0\), then \(\rho \circ \eta\) is also a metric.
\end{corollary}

\begin{definition}[Linear Metric Space]
  A vector space \(V\) over the field \(\mathbb{K}\) equipped with a metric 
  \(\rho\) on \(V\) and a metric \(|\cdot - \cdot|\) on \(\mathbb{K}\) is a 
  linear metric space if \((+) : V \times V \to V\) and 
  \((\cdot) : \mathbb{K} \times V \to V\) are continuous with respect to the 
  induced metric.
\end{definition}

\begin{proposition}
  Any normed space is a linear metric space.
\end{proposition}
\begin{proof}
  Let \((x_n, y_n) \to (x, y)\) in \(V^2\), then we have 
  \[\|(x_n + y_n) - (x + y)\| \le \|x_n - x\| + \|y_n - y\| \to 0.\]
  Thus, \((+)\) is continuous. 
  
  Similarly, if \((\lambda_n, x_n) \to (\lambda, x)\) in \(\mathbb{K} \times V\),
  \[\begin{split}
    \|\lambda_n x_n - \lambda x\| 
      & = \|\lambda_n x_n - \lambda_n x + \lambda_n x - \lambda x\| \\
      & \le \|\lambda_n x_n - \lambda_n x\| + \|\lambda_n x - \lambda x\| \\
      & = |\lambda_n| \|x_n - x\| + |\lambda_n - \lambda| \|x\|.
  \end{split}\]
  Now, since \((\lambda_n)\) is convergent, it is bounded by some \(M > 0\) and 
  thus, 
  \[\|\lambda_n x_n - \lambda x\| 
    \le |\lambda_n| \|x_n - x\| + |\lambda_n - \lambda| \|x\|
    \le M \|x_n - x\| + |\lambda_n - \lambda| \|x\| \to 0,\]
  implying \((\cdot)\) is continuous.
\end{proof}

\subsection{Classical Spaces}

We recall the \(L_p\) spaces from second year measure theory, and in particular, 
when we consider the counting measure \(\mu\), we have the nice property that 
\[\int f \dd \mu = \sum_{n = 0}^\infty f(n),\]
and we no longer require a quotient to define the linear space as the only 
null-set is the empty set (thus, two function are a.e equal if and only if they 
are equal). In this special case, we call the resulting space \(\ell_p\) with 
the \(p\)-norm 
\[\|f\|_p = \left(\int |f|^p \dd \mu\right)^{\frac{1}{p}} = 
  \left(\sum_{n = 0}^\infty |f(n)|^p \right)^{\frac{1}{p}}.\]
We will use the sequence notation and write \(a_n := a(n)\) for \(a \in \ell_p\).

As this is simply a special case of the \(L_p\) space, the inequalities proved 
on the \(L_p\) space remains. We will recall them here for \(\ell_p\) spaces.

\begin{proposition}[Hölder's Inequality]
  Let \(\frac{1}{p} + \frac{1}{q} = 1\) where \(p, q \in (1, \infty)\). Then 
  for \(a = (a_i)_{i \in \mathbb{N}}, b = (b_i)_{i \in \mathbb{N}} \in \ell_p\), 
  we have 
  \[|\langle a, b\rangle| \le \|a\|_p \|b\|_q,\] 
  where \(\langle a, b\rangle := \sum_{i \in \mathbb{N}} a_i b_i\).
\end{proposition}

\begin{proposition}[Minkowski's Inequality]
  Let \(a, b \in \ell_p\) for some \(1 \le p \le \infty\). 
  Then \(a + b \in \ell_p\) and
  \[\|a + b\|_p \le \|a\|_p + \|b\|_p.\]
\end{proposition}

\begin{lemma}
  Let \(I \subseteq \mathbb{R}\) be an interval and 
  \(\phi : I \to \mathbb{R}^+\) be a function. 
  Then \(\phi\) is convex if and only if for all \(y \in I\), there exists 
  a \(\gamma \in \mathbb{R}\), such that for all \(x \in I\),
  \[\gamma(x - y) \le \phi(x) - \phi(y).\]
\end{lemma}

\begin{proposition}[Jensen's Inequality]
  let \(\phi \ge 0\) be convex and suppose \(\sum_{i \in \mathbb{N}} \eta_i = 1\), 
  \(|\langle \alpha \rangle| < \infty\) and \(\langle \phi(\alpha) \rangle\) 
  (where \(\langle \beta \rangle = \sum_{i \in \mathbb{N}} \eta_i \beta_i\)), 
  then 
  \[\phi(\langle \alpha \rangle) \le \langle \phi(\alpha) \rangle.\]
\end{proposition}
\begin{proof}
  By the above lemma, there exists some \(\gamma\) such that 
  \[\gamma(\alpha_j - \langle \alpha \rangle) \le 
    \phi(\alpha_j) - \phi(\langle \alpha \rangle).\]
  Thus, 
  \[\begin{split}
    0 = \sum \eta_i \gamma(\alpha_j - \langle \alpha \rangle) 
      & \le \sum \eta_i (\phi(\alpha_j) - \phi(\langle \alpha \rangle)) \\
      & = \sum \eta_i \phi(\alpha_j) - \phi(\langle \alpha \rangle) \sum \eta_i \\
      & = \langle \phi(\alpha_j) \rangle - \phi(\langle \alpha \rangle)
    \end{split}\]
  where the first equality follows as \(\gamma\) is independent of the index 
  \(i\).
\end{proof}

\begin{proposition}
  For \(p < p'\), \(\ell_p \subseteq \ell_{p'}\). On the other hand, if 
  \(\sum \eta_i = 1\), we have \(\ell_p(\eta) \supseteq \ell_{p'}(\eta)\).
\end{proposition}

\begin{definition}
  Let \(\phi\) be a convex function such that \(\phi(0) = 0\), 
  \(\phi(x) \to \infty\) as \(x \to \infty\). If \(\phi\) has the doubling 
  property such that there exists some \(M > 0\) such that for all 
  \(x \in \mathbb{R}\), \(\phi(2 |x|) \le M \phi(|x|)\), then 
  \[V := \left\{a : \mathbb{N} \to \mathbb{R} \mid \sum \eta_i 
    \phi(|a_i|) < \infty\right\}\]
  where \(\sum \eta_i = 1\) is a vector space with point-wise operations.
\end{definition}

\begin{proposition}
  Given a metric space \((X, \rho)\), there exists a complete metric space 
  \((\tilde X, \tilde \rho)\) and an isometric embedding 
  \(\iota : X \to \tilde X\) such that for all \(x, x' \in X\),
  \(\tilde \rho(\iota(x), \iota(x')) = \rho(x, x')\).
\end{proposition}
\begin{proof}
  We have seen similar ideas in the completion of \(\mathbb{Q}\) though 
  completing the space with equivalence classes of mutually Cauchy sequences 
  as elements of \(\tilde X\).
\end{proof}

As we have seen last year, the \(L_p\) spaces are complete, and thus are 
Banach spaces. Thus, we have \(\ell_p\) spaces are also complete and are 
Banach spaces. In fact, the proof that \(\ell_p\) spaces are complete is 
easier, in that one may show completeness through showing the point-wise 
limit of the sequences indeed belong to \(\ell_p\).

\begin{definition}
  We define 
  \begin{itemize}
    \item \(c_0 := \{x \in \ell_\infty \mid \lim_{n \to \infty} x_n = 0\}\),
    \item \(c := \{x \in \ell_\infty \mid \exists \lim_{n \to \infty} x_n\}\),
  \end{itemize}
  be subspaces of \(\ell_\infty\).
\end{definition}

\begin{proposition}
  \(c_0\) is complete.
\end{proposition}
\begin{proof}
  Let \((x_i^n) \subseteq c_0\) be a Cauchy sequence and let 
  \(x_i = \lim_{n \to \infty} x_i^n\), then it suffices to show 
  \(\lim_{i \to \infty} x_i = 0\). 

  Since \((x_i^n)\) is Cauchy, for all \(\epsilon > 0\), there exists some
  \(N \in \mathbb{N}\) such that for all \(n \ge N\), 
  \[\|x^n - x^N\|_\infty < \frac{\epsilon}{2}.\]
  Furthermore, as \(x_i^N \to 0\) as \(i \to \infty\), there exists some 
  \(I \in \mathbb{N}\) such that for all \(i \ge I\), 
  \(|x_i^N| < \epsilon / 2\). Thus, for all \(i \ge I\), we have 
  \[\frac{\epsilon}{2} > \|x^n - x^N\|_\infty > |x_i^n - x_i^N| >
    |x_i^n| - \frac{\epsilon}{2} \implies \epsilon > |x_i^n|,\]
  for all \(n \ge N\). Hence, taking \(n \to \infty\), we have 
  \[\epsilon > \lim_{n \to \infty} |x_i^n| = |x_i|,\]
  implying \(x_i \to 0\) as \(i \to \infty\).
\end{proof}

\begin{proposition}
  \(c\) is complete.
\end{proposition}
\begin{proof}
  Similar to above, let \((x_i^n)\) be Cauchy and define \((x^n)\) such that for 
  all \(n \in \mathbb{N}\), \(x^n = \lim_{i \to \infty} x_i^n\) and  
  \((x_i)\) such that \((x_i^n) \to (x_i)\) in \(\ell_\infty\). It is not 
  difficult to see that \((x^n)\) is Cauchy and thus converges to some \(x\).
  \[\begin{tikzcd}
    (x_i^n) \arrow{r}{n \to \infty} 
    \arrow[swap]{d}{i \to \infty} & (x_i) \arrow{d}{i \to \infty} \\
    (x^n) \arrow{r}{n \to \infty}& x
    \end{tikzcd}\]
  Indeed, as \((x_i^n)\) is Cauchy, for all \(\epsilon > 0\), there exists some 
  \(N \in \mathbb{N}\) such that for all 
  \(p, q \ge N\), \(\|x_i^p - x_i^q\|_\infty < \epsilon / 3\).
  Furthermore, there exists some \(I \in \mathbb{N}\) such that for all 
  \(j \ge I\), \(|x_j^p - x^p|, |x_j^q - x^q| < \epsilon / 3\). Thus, 
  \[|x^p - x^q| < |x_j^p - x^p| + |x_j^p - x_j^q| + |x_j^q - x^q| 
    < \frac{\epsilon}{3} + \|x_i^p - x_i^q\|_\infty + \frac{\epsilon}{3} 
    = \epsilon.\]
  Hence, it remains to show \(x_i \to x\) as \(i \to \infty\). 
  Fix \(\epsilon > 0\), then there exists \(N \in \mathbb{N}\) such that for 
  all \(n \ge N\), \(|x^n - x| < \epsilon / 3\). Furthermore, there exists 
  \(M \in \mathbb{N}\) such that for all \(n \ge M\), 
  \(\|x_i^n - x_i\|_\infty < \epsilon / 3\). Then, for all 
  \(p \ge \max\{N, M\}\), there exists \(I \in \mathbb{N}\) such that 
  for all \(j \ge I\), \(|x_j^p - x^p| < \epsilon / 3\). Finally, 
  \[|x_j - x| \le |x_j - x_j^p| + |x_j^p - x^p| + |x^p - x| \le 
    \|x_i - x_i^p\|_\infty + \frac{\epsilon}{3} + \frac{\epsilon}{3} < \epsilon,\]
  implying \(x_i \to x\) as \(i \to \infty\) and \((x_i) \in c\).
\end{proof}

\subsubsection{Separability}

In this section we will consider the separability of different classical spaces.

\begin{proposition}
  The space \(\ell_p\) is separable for all \(p \ge 1\).
\end{proposition}
\begin{proof}
  It is easy to see that the subspace of all finite sequences is dense in 
  \(\ell_p\). Indeed, if \(x \in \ell_p\) and \(x^n\) is the sequence such that 
  \(x^n_i = x_i\) for all \(i \le n\) and \(x^n_i = 0\) for all \(i > n\), 
  we have 
  \[\|x - x^n\|_p^p = \sum_{i = 0}^\infty |x_i - x_i^n|^p = 
    \sum_{i = n + 1}^\infty |x_i|^p\]
  which tends to \(0\) as \(n \to \infty\). Now, by defining 
  \(\ell_{\mathbb{Q}, n}\) as the set of rational sequences with length \(n\),
  we have \(d := \bigcup_{n \in \mathbb{N}} \ell_{\mathbb{Q}, n}\) is dense in the 
  space of all finite sequences. Now, since \(d\) is countable as it is a countable 
  union of countable sets, we have found a countable dense set of \(\ell_p\).
\end{proof}

\begin{lemma}
  A metric space is \(X\) not separable if there exists an uncountable subset \(S\) 
  such that for some some \(k > 0\), \(d(x, y) > k\) for all \(x, y \in S\).
\end{lemma}
\begin{proof}
  As subset of a separable \textit{metric} space is separable, \(S\), must be 
  separable. But as \(d(x, y) > k\) for all \(x, y \in S\), no sequences of 
  \(S\) but the constant sequence can converge. Thus, for all countable subsets 
  of \(S\), simply picking a point of \(S\) not in that subset suffices.
\end{proof}

\begin{proposition}
  \(\ell_\infty\) is not separable.
\end{proposition}
\begin{proof}
  Clearly the set containing all sequences of only \(0\) and \(1\)s is 
  a subset of \(\ell_\infty\). Furthermore, this sequence is uncountable as 
  it bijects \(\mathbb{R}\) be considering the binary representation of a real 
  number. Thus, since all distinct elements in this set have distance 1 apart, 
  the conclusion follows by the above lemma.
\end{proof}

\begin{proposition}
  \(C^k([a, b])\) (the set of \(k\)-differentiable functions from the interval 
  \([a, b]\)) for all \(k \in \mathbb{N}\) is separable.
\end{proposition}
\begin{proof}
  Recalling the Weierstass approximation theorem, we have for all continuous 
  function \(f\), there exists a sequence of polynomials \(f_n\) such that 
  \(f_n \to f\) uniformly. Thus, as the set of polynomials with rational 
  coefficients is countable, and dense in the space of all polynomials, we 
  have found a countable dense set of \(C^0([a, b])\). Now as 
  \(C^k([a, b]) \subseteq C^0([a, b])\) for all \(k \in \mathbb{N}\), we have 
  \(C^k([a, b])\) is separable as required. 
\end{proof}

\begin{definition}[Absolutely Convergent]
  Let \(X\) be a normed space and let \((x_n)\) be a sequence of \(X\), then 
  a series \(\sum_{n \in \mathbb{N}} x_n\) is said to be absolutely convergent 
  if \(\sum_{n \in \mathbb{N}} \|x_n\| < \infty\).
\end{definition}

\begin{proposition}
  A normed space \(X\) is complete if and only if for all sequences \((x_n)\) 
  of \(X\) such that \(\sum x_n\) is absolutely convergent implies \(x_n\) 
  converges. Thus, a normed space is Banach if and only if this property is 
  satisfied.
\end{proposition}
\begin{proof}
  See problem sheet.
\end{proof}

By recalling the proof of the completeness of \(L_p\) spaces, we note that 
this property was used extensively. As a consequence, we see that \(L_p(\mu)\) 
is separable if the measure \(\mu\) is separable. In particular, the spaces 
\(L_p(\mathbb{R}^n, \lambda)\) are separable for all \(n\). On the other hand, 
\(L_\infty\) is in general not separable.

\subsection{Hamel and Schauder Basis}

In this small section we will introduce two new notions of basis for infinite 
dimensional spaces. In particular, we will introduce the Hamel basis which is 
a natural extension of the definition of basis for the finite dimensional case 
and the Schauder basis which is a notion of basis that incorporates the 
topological properties of the space.

\begin{definition}[Linear Independent]
  A set \(W \subseteq V\) is linear independent if for all 
  \((\lambda_i)_{i = 1}^m \subseteq \mathbb{K}\), \((w_i)_{i = 1}^n \subseteq W\), 
  \[\sum \lambda_i w_i = 0 \implies \lambda_i = 0\]
  for all \(i = 1, \cdots, n\).
\end{definition}

\begin{definition}[Hamel Basis]
  A set \(W \subseteq V\) is a Hamel basis for a linear space \(V\) if 
  \(W\) is linearly independent and for all \(x \in V\), there exists a unique 
  finite linear combination of vectors \((w_i)_{i = 1}^n \subseteq W\), 
  \((\lambda_i)_{i = 1}^m \subseteq \mathbb{K}\) 
  such that 
  \[x = \sum \lambda_i w_i.\]
\end{definition}

\begin{proposition}
  Every linear space has a Hamel basis.
\end{proposition}
We will come back to the proof of this proposition after discussing an important 
result known as the Hahn-Banach theorem.

\begin{definition}[Schauder Basis]
  A set \(W \subseteq V\) is a Schauder basis for a normed space \(V\) if 
  \(W\) is countable, linearly independent, and for all \(x \in V\), there 
  exists a unique (possibly infinite)-sequence 
  \((\lambda_i)_{i = 1}^\infty \subseteq \mathbb{K}\) and 
  \((w_i)_{i = 1}^\infty \subseteq W\) such that
  \[x = \sum_{i = 1}^\infty \lambda_i w_i.\]
\end{definition}

\begin{proposition}
  If a Banach space \(X\) has a Schauder basis, then it is separable.
\end{proposition}

The proof of the above proposition is left as an exercise. It is notable that 
the reverse of the above is not true and was in fact a Scottish book problem 
which an counterexample was given by Per Enflo in 1972 who was awarded a live 
goose.

It is clear that for \(p \ge 1\), the set \(W := \{e_j \mid j \in \mathbb{N}\}\) 
where \((e_j)_i = \delta_{ij}\) is a Schauder basis of \(\ell_p\). 

Recall that \(c \subseteq \ell_\infty\) is the set of sequences which has 
a limit. By considering the sequence \(x_n = 1\), we see that the set of 
standard basis vectors no longer form a basis of \(c\) as 
\[\left\|x - \sum^n_{i = 1} e_i\right\|_\infty = 1,\]
for all \(n\). Defining \(W := \{e_0 := (1, 1, \cdots)\} \cup 
\{e_j \mid j \in \mathbb{N}\}\), for all \((a_n) \in c\), let 
\(\lambda_0 := \lim_{n \to \infty} a_n\) and \(\lambda_n = a_n - \lambda_0\).
Then, for all \(\epsilon > 0\), there exists some \(N \in \mathbb{N}\) such that 
for all \(n \ge N\), \(|a_n - \lambda_0| \leq \epsilon\). Thus, 
\[\left\|\lambda_0 e_0 + \sum_{i = 1}^n \lambda_i e_i - a\right\|_\infty 
  = \|(0, \cdots, 0, \lambda_0 - a_n, \lambda_0 - a_{n + 1}, \cdots)\|_\infty
  < \epsilon,\]
implying \(W\) is a Schauder basis of \(c\).

Furthermore, one may show that \(C([0, 1])\) has a Schauder basis consisting of 
all the ``spike'' functions at the points \(k2^{-n}\) for all \(n \in \mathbb{N}\), 
\(k = 1, \cdots 2^{-n}\).

\subsection{Hilbert Spaces}

Recall the definition of sesquilinear forms over some vector space \(\mathbb{H}\).

\begin{definition}[Sesquilinear Form]
  A sesquilinear form \(\langle \cdot, \cdot \rangle : 
  \mathbb{H} \times \mathbb{H} \to \mathbb{K}\) on \(\mathbb{H}\) where 
  \(\mathbb{K}\) is \(\mathbb{R}\) or \(\mathbb{C}\) is a function such that
  \begin{itemize}
    \item it is linear with respect to the second argument;
    \item and is conjugate symmetric.
  \end{itemize}
  We say \(\langle \cdot, \cdot \rangle\) is nondegenerate if \(x = 0\) if and 
  only if \(\langle x, y\rangle = 0\) for all \(y \in \mathbb{H}\). 
  Nondegenerate sesquilinear forms are called scalar products and a 
  vector space equipped with a scalar product is called a unitary space 
  (or a scalar product space).
\end{definition}

We see that a scalar product induces a norm by defining 
\(\|f\|^2 := \langle f, f \rangle\). In particular, we recall the 
\(\ell_2, C([a, b])\) and \(L_2\) are all scalar product spaces.

\begin{proposition}
  Let \(\mathbb{H}\) be a unitary space, then 
  \begin{itemize}
    \item \(\|f + g\|^2 + \|f - g\|^2 = 2 \|f\|^2 + 2\|g\|^2\) 
      (parallelogram identity);
    \item \(\langle f, g\rangle = \frac{1}{4}(\|f + g\|^2 - |f - g\|^2)\) if 
      \(\mathbb{K} = \mathbb{R}\) and 
      \(\langle f, g\rangle = \frac{1}{4} \sum_{k = 0, \cdots 3} 
      i^k \|f + i^k g\|^2\) if \(\mathbb{K} = \mathbb{C}\)
      (polarisation identity).
  \end{itemize}
\end{proposition}

\begin{definition}[Hilbert Space]
  A unitary space is a Hilbert space if it is complete with respect to its 
  induced norm.
\end{definition}

\begin{definition}[Convex Set]
  A set \(S\) is said to be convex if for all \(x, y \in S\), 
  \((1 - t)x + ty \in S\) for all \(t \in [0, 1]\).
\end{definition}

\begin{proposition}[Nearest Point Property]
  Every nonempty closed convex set \(\mathcal{K}\) in a Hilbert space 
  \(\mathbb{H}\) contains a vector of the smallest norm. Moreover, if 
  \(h \in \mathbb{H}\), there exists a unique \(h_0 \in \mathcal{K}\) 
  such that 
  \[\|h - h_0\| = \text{dist}(h, \mathcal{K}) := \inf_{k \in K} \|h - k\|.\]
\end{proposition}
\begin{proof}
  By the definition of infimum, there exists a sequence 
  \((k_n) \subseteq \mathcal{K}\) such that 
  \[\lim_{n \to \infty} \|k_n\| \to d := \inf_{k \in \mathbb{K}} \|k\|.\]
  Consider, for all \(n, m \in \mathbb{N}\), by the parallelogram identity,
  \[\left\|\frac{1}{2}(k_n - k_m)\right\|^2 = 
    \frac{1}{2}(\|k_n\|^2 + \|k_m\|^2) - \left\|\frac{1}{2}(k_n + k_m)\right\|^2.\]
  Then, as \(\mathbb{K}\) is convex, \(\frac{1}{2}(k_n + k_m) \in \mathbb{K}\) 
  and so \(\|(k_n + k_m) / 2\|^2 \ge d^2\) and 
  \[\left\|\frac{1}{2}(k_n - k_m)\right\|^2 \le 
    \frac{1}{2}(\|k_n\|^2 + \|k_m\|^2) - d^2.\]
  Thus, taking \(n, m \to \infty\), the right hand side tends to zero and so 
  \((k_n)\) is Cauchy and hence convergent. Now as \(\mathcal{K}\) is closed, 
  the limit is in \(\mathcal{K}\) and hence the statement.
\end{proof}

\begin{definition}[Orthogonal Systems]
  Let \((\mathbb{H}, \langle \cdot, \cdot \rangle)\) be an unitary space. 
  A set of vectors \(\{e_j \in \mathbb{H} \mid j \in J\}\) for some index 
  set \(J\) is called an orthogonal system if for all distinct \(i, j \in J\), 
  \[\langle e_i, e_j \rangle = 0.\]
  If furthermore, \(\langle e_i, e_i \rangle = 1\) for all \(i \in J\), then 
  we say the system is orthonormal.
\end{definition}

\begin{proposition}
  Every orthogonal system is linearly independent.
\end{proposition}
\begin{proof}
  Exercise.
\end{proof}

Using Zorn's lemma, one can show that every unitary space contains an orthogonal 
basis.

\begin{definition}[Fourier Coefficients]
  Let \((\mathbb{H}, \langle \cdot, \cdot \rangle)\) be an unitary space 
  and let \(\{e_j \in \mathbb{H} \mid j \in J\}\) be an orthogonal system. 
  Then for each \(f \in \mathbb{H}\), the Fourier coefficients with respect to 
  the orthogonal system are 
  \[c_j := \frac{\langle e_j, f \rangle}{\|e_j\|^2}.\]
\end{definition} 

\begin{proposition}
  If \(f = \sum_{k \in \mathbb{N}} \alpha_k e_k\), then \(a_j = c_j\).
\end{proposition}
\begin{proof}
  Let \(n < m\) and \(S_m := \sum_{k = 1}^m \alpha_k e_k\). Then, 
  \(\langle S_m, e_n\rangle = \overline{\alpha_n} \|e_n\|^2\). Thus, we have 
  \[|\overline{\alpha_n} \|e_n\|^2 - \langle f, e_n\rangle| = 
    |\langle S_m, e_n\rangle - \langle f, e_n\rangle | = 
    |\langle S_m - f, e_n \rangle|\le\|S_m - f\| \|e_n\| \to 0\]
  as \(m \to \infty\). 
\end{proof}

\begin{proposition}
  Suppose \((e_j)_{j \in \mathbb{N}}\) is orthogonal and 
  \(g(a_1, \cdots, a_n) := \left\| f - \sum_{j = 1}^n a_je_j\right\|^2\). 
  Then \(g\) attains its minimum at \(a_j = c_j\). Furthermore, 
  we have 
  \[\sum_{j = 1}^\infty |c_j|^2\|e_k\|^2 \le \|f\|^2.\]
  This inequality is known as Bessel's inequality.
\end{proposition}
\begin{proof}
  Exercise.
\end{proof}

\begin{definition}
  An orthogonal system \(\{e_j \mid j \in J\}\) is called complete if 
  for all \(f \in \mathbb{H}\), if \(\langle f, e_j \rangle = 0\) for all 
  \(j\), then \(f = 0\).
\end{definition}

\begin{proposition}
  The following are equivalent
  \begin{enumerate}
    \item \((e_j)_{j \in \mathbb{N}}\) is complete;
    \item \(\left\|f - \sum_{j = 1}^n c_je_j\right\| \to 0\) as \(k \to \infty\) where 
      \(c_j\) are the Fourier coefficients;
    \item \(\|f\|^2 = \sum_{j = 1}^\infty |c_j|^2\|e_k\|^2\) for all 
      \(f \in \mathbb{H}\).
  \end{enumerate}
\end{proposition}
\begin{proof}
  Exercise.
\end{proof}

\subsection{Finite Dimensional Spaces}

\begin{theorem}
  Let \((X, \|\cdot\|)\) be a finite dimensional normed space and let 
  \(\{e_i\}\) be a basis for \(X\). Then, there exists some 
  \(M, m \in \mathbb{R}^+\) such that for all \(x = \sum_{i = 1}^n a_i e_i\),
  \[m \sum_{i = 1}^n |a_i| \le \|x\| \le M \sum_{i = 1}^n|a_i|.\]
\end{theorem}
\begin{proof}
  WLOG. assume \(\sum_{i = 1}^n |a_i| = 1\) since if otherwise, we may just scale 
  \(x\) by \(1 / \sum |a_i|\). Then the function 
  \(f : (a_i) \mapsto \|\sum_i a_i e_i\|\) for all sequences \((a_i)\). Now, 
  by checking that \(f\) is continuous, and by considering that the set 
  \(\{(a_i) \mid \sum \|a_i = 1\|\}\) is a closed subspace, \(f\) must 
  attain a minimum \(m\) implying the left hand side inequality. 
  On the other hand, consider the inequality 
  \[\left\| \sum_{i = 1} a_i e_i \right\| \le \sum |a_i| \|e_j\| \le 
    \max_k \|e_k\| \sum |a_i| = \max_k \|e_k\|.\]
  Hence, it suffices to choose \(M = \max \|e_k\|\).
\end{proof}

A direct corollary is that all norms on finite dimensional spaces are equivalent.

\begin{definition}[Equivalent Norms]
  Two norms \(\|\cdot\|_1, \|\cdot\|_2\) on \(X\) are said to be equivalent 
  if there exists some \(C \in \mathbb{R}^+\) such that 
  \[\frac{1}{C}\|x\|_1 \le \|x\|_2 \le C\|x\|_1,\]
  for all \(x \in X\).
\end{definition}

\begin{proposition}
  Equivalence of norms is an equivalence relation.
\end{proposition}
\begin{proof}
  Easy check.
\end{proof}

\begin{corollary}
  Every norm on a finite dimensional space is equivalent.
\end{corollary}
\begin{proof}
  If \(\|\cdot\|_1, \|\cdot\|_2\) are two norms on \(X\), then
  let \(\{e_i\}\) be a basis of \(X\). Thus, by simply defining the norm 
  \[\|x\| = \left\| \sum_{i = 1}^n a_i e_i \right\| := 
    \sum_i |a_i|,\]
  we have \(\|\cdot\|_1\) is equivalent to \(\|\cdot\|\) which is equivalent 
  to \(\|\cdot\|_2\). Hence, \(\|\cdot\|_1\) is equivalent to \(\|\cdot\|_2\) 
  by transitivity.
\end{proof}

\begin{proposition}
  Every finite dimensional space over a complete field is complete.
\end{proposition}
\begin{proof}
  Follows by considering the individual coefficients of a Cauchy sequence with 
  respect to some basis is also Cauchy.
\end{proof}

\begin{proposition}
  Every compact set of a normed space is closed and bounded.
\end{proposition}
\begin{proof}
  As a normed space is a metric space, it suffices to consider sequential compactness. 
  In particular, if a set is not closed, then it contains a sequence converging 
  to a point not within the set. Then every sub-sequence of that sequence also 
  converges to that point outside of the set, and so is not compact. On the other 
  hand, if the set is not bounded, we can construct a sequence with norms 
  tending to \(\infty\). It is then clear that the sequence does not contain any 
  convergent subsequence.
\end{proof}

We note that the converse of the above proposition is not true for infinite 
dimensional spaces. Indeed, we see that the canonical basis of \(\ell_2\) 
is closed and bounded but not complete.

\begin{proposition}
  If \((X, \|\cdot\|)\) is a finite dimensional normed space, then every closed 
  and bounded set is compact (this property is known as the Heine-Borel property). 
\end{proposition}
\begin{proof}
  Follows by choosing a subsequence for each coefficient by using 
  Bolzano-Weierstrass.
\end{proof}

In fact this property is sufficient to determine whether or not a normed space 
is finite dimensional.

\begin{lemma}[Riesz's lemma]
  Let \(Y\) be a closed proper subspace of a subspace \(Z\) of \(X\) where 
  \((X, \|\cdot\|)\) is a normed space. Then 
  for any \(\theta \in (0, 1)\), there exists some \(z \in Z\) such that 
  \[\|z\| = 1 \text{ and } \|y - z\| \ge \theta\]
  for all \(y \in Y\). 
\end{lemma}
\begin{proof}
  Let \(v \in Z \setminus Y\), and define \(a := \inf_{y \in Y} \|y - v\| > 0\). 
  Then, as \(Y\) is closed, \(\|y_0 - v\|\) attains \(a\) for some \(y_0 \in Y\). 
  Thus, 
  \[0 < a = \|v - y_0\| \le \frac{a}{\theta}.\]
  Then, defining 
  \[z := \frac{v - y_0}{\|v - y_0\|} =: c(v - y_0),\]
  we have for all \(y \in Y\),
  \[\|z - y\| = \|c(v - y_0) - y\| = 
    c\left\|v - \left(y_0 + \frac{1}{c}y\right)\right\|.\]
  Since \(y_0 + \frac{1}{c}y =: y_1 \in Y\), we have \(\|v - y_1\| \ge a\), 
  and so 
  \[\|z - y\| = c\|v - y_1\| \ge ca = \frac{a}{\|v - y_0\|} \ge 
    \frac{a}{a / \theta} = \theta.\]
\end{proof}

\begin{proposition}
  A normed space is finite dimensional if and only if 
  \(\{x \mid \|x\| = 1\}\) is compact.
\end{proposition}
\begin{proof}
  Clearly, if \(X\) is finite dimensional, then as \(\{x \mid \|x\| = 1\}\) is 
  closed and bounded, it is compact.

  Let \((X, \|\cdot\|)\) be an infinite dimensional normed space and suppose 
  \(\{e_i\}_{i = 1}^\infty\) is a countable infinite linearly independent subset 
  of \(X\). Define \(Z_n = \text{span}\{e_i \mid i = 1, \cdots, n\}\), and 
  for each \(n\), by Riesz's lemma, define \(z_n \in Z_{n + 1}\) such that 
  \(\|z_n\| = 1\) and \(\|z_n - z_i\| \ge 1 / 2\) for all \(i = 1, \cdots, n\).
  Thus, we have defined a sequence in \(\{x \mid \|x\| = 1\}\) which does not 
  contain any convergent subsequence, and hence, \(\{x \mid \|x\| = 1\}\) is not 
  compact.
\end{proof}

\newpage
\section{Linear Operators}

\subsection{Bounded Linear Operators}

\begin{definition}[Bounded Linear Operator]
  A linear map \(T : X \to Y\) between normed spaces \((X, \|\cdot\|_X)\) and 
  \((Y, \|\cdot\|_Y)\) is a bounded linear operator if there exists some 
  \(c \in \mathbb{R}^+\) such that,
  \[\|Tx\|_Y \le C\|x\|_X\]
  for all \(x \in X\). In this case, one may define the operator norm by,
  \[\|T\| := \sup_{x \in X; x \neq 0} \frac{\|Tx\|_Y}{\|x\|_X}.\]
\end{definition}

\begin{proposition}
  Every linear map from a finite dimensional normed spaces is bounded.
\end{proposition}
\begin{proof}
  Simply bound \(T\) by the maximum of the norm of the basis.
\end{proof}

\begin{proposition}
  Let \(X, Y\) be Banach spaces and let \(T : X \to Y\) be a continuous linear map,
  then for all compact \(\mathcal{K} \subseteq X\), \(T(\mathcal{K}) \subseteq Y\) 
  is also compact.
\end{proposition}
\begin{proof}
  Follows by recalling that the continuous image of a compact space is compact 
  (by pull-back of the open-cover).
\end{proof}

\begin{proposition}
  If \(T : X \to Y\) is a linear map between normed spaces 
  \((X, \|\cdot\|_X)\) and \((Y, \|\cdot\|_Y)\), then the following are equivalent,
  \begin{enumerate}
    \item \(T\) is continuous;
    \item \(T\) is Lipschitz continuous;
    \item \(T\) is bounded;
    \item \(T\) is continuous at some point \(x_0 \in X\).
  \end{enumerate}
\end{proposition}
\begin{proof}
  Clearly \((3) \implies (2) \implies (1) \implies (4)\), so let us first show 
  \((4) \implies (1))\). If \(T\) is continuous at \(x_0\), then 
  for all \(\epsilon > 0\), there exists some \(\delta > 0\) such that 
  for all \(x \in B_\delta(x_0)\), \(\epsilon > \|Tx - Tx_0\|\).
  Then, for all \(\tilde x \in X\), we have 
  \[\epsilon > \|Tx - Tx_0\| = 
    \|Tx + T(\tilde x - x_0) - Tx_0 - T(\tilde x - x_0)\| =
    \|T(\tilde x + (x - x_0)) - T \tilde x\|.\]
  Thus, for all \(x \in B_\delta(\tilde x)\), we have 
  \(x - \tilde x + x_0 \in B_\delta(x_0)\) and hence,
  \[\epsilon > \|T(\tilde x + (x - \tilde x + x_0 - x_0)) - T \tilde x\| 
    = \|Tx - T\tilde x\|,\]
  which implies \(T\) is continuous. 

  Now, it suffices to show that \((1) \implies (3)\). Since \(T\) is continuous 
  at \(0\), there exists some \(\delta > 0\) such that for all 
  \(x \in B_\delta(0)\), \(\|Tx\| < 1\). Then for all \(x \in X\), we have 
  \[\left\|\frac{\delta x}{2\|x\|}\right\| = \frac{\delta}{2} < \delta,\]
  and so, 
  \[1 > \left\|T\left\|\frac{\delta x}{2\|x\|}\right\|\right\| = 
    \frac{\delta}{2\|x\|}\|Tx\|,\]
  implying \(Tx < 2\|x\| / \delta\) for all \(x\).
\end{proof}

\begin{definition}
  We denote the space of linear bounded operators between two normed spaces 
  \(X, Y\) equipped with the operator norm by \(\mathcal{L}(X, Y)\).
\end{definition}

\begin{proposition}
  Let \((X, \|\cdot\|_X)\) be a normed space and \((Y, \|\cdot\|_Y)\) be a 
  Banach space. Then \(\mathcal{L}(X, Y)\) is also a Banach space.
\end{proposition}
\begin{proof}
  Exercise.
\end{proof}

We will now recall the Banach contraction mapping theorem.

\begin{definition}[Contraction]
  A map \(T : X \to X\) on a metric space \((X, \rho)\) is a contraction if 
  there exists some \(\alpha\), \(0 < \alpha \le 1\) such that for all 
  \(x, y \in X\), 
  \[\rho(Tx, Ty) \le \alpha \rho(x, y).\]
  \(T\) is a strict contraction if \(a < 1\).
\end{definition}

\begin{definition}[Fixed Point]
  A point \(x \in X\) is a fixed point of the map \(T : X \to X\) if \(Tx = x\).
\end{definition}

\begin{theorem}[Banach Contraction Mapping]
  If \(X\) is a complete metric space and \(T : X \to X\) is a strict contraction,
  then \(T\) has a unique fixed point.
\end{theorem}
\begin{proof}
  Exercise/see second year analysis.
\end{proof}

\subsection{Dual Space and Dual Operators}

\subsubsection{Dual Space}

\begin{definition}[Dual Space]
  Given a normed space \((X, \| \cdot \|)\), its dual space is 
  \(X^* := \mathcal{L}(X, \mathbb{K})\). The elements of the dual spaces are 
  called continuous functionals on \(X\).
\end{definition}

\begin{proposition}
  \((\ell_p)^* = \left\{T : \ell_p : \to \mathbb{K} \mid T(x) = 
    \sum_i y_i x_i, y \in \ell_q\right\}\) where \(1 / p + 1 / q = 1\) and 
    \(\mathbb{K} = \mathbb{R}\) or \(\mathbb{C}\).
\end{proposition}
\begin{proof}
  Let \(T(x) = \sum_i y_i x_i\) for some \(y \in \ell_q\), it is clear that 
  \(T\) is linear. Consider, by Hölder's inequality
  \[\|T(x)\| = \left|\sum_i y_i x_i\right| \le \|x\|_p \|y\|_q.\]
  Thus, \(\|T\| \le \|y\|_q < \infty\), implying \(T \in (\ell_p)^*\).

  Now, let \(\phi \in (\ell_p)^*\). 
  Fixing \(y_n = \phi(e_n)\), it suffices to show 
  that \((y_n) \in \ell_q\) since it is clear that \(\phi(x) = \sum x_i y_i\) 
  by linearity. 
  Setting \(x_i = |y_i|^q / y_i\) for all \(y_i \neq 0\) and \(x_i = 0\) for 
  \(y_i = 0\), consider that \(\phi(\sum x_i e_i) = \sum x_i y_i = \sum |y_i|^q\). 
  On the other hand, 
  \[|\phi(x)| \le \|\phi\| \|x\|_p = \|f\|\left(\sum |x_i|^p\right)^{1 / p} 
  = \|\phi\| \left(\sum \frac{|y_i|^{qp}}{|y_i|^p}\right)^{1 / p}.\]
  As \(1 / p + 1 / q = 1\), we have \(qp = p + q\) and so, 
  \[|\phi(x)| \le  \|\phi\| \left(\sum \frac{|y_i|^{qp}}{|y_i|^p}\right)^{1 / p.}
  = \|\phi\| \left(\sum |y_i|^q\right)^{1 / p}.\]
  Thus, 
  \[\sum |y_i|^q \le \|\phi\| \left(\sum |y_i|^q\right)^{1 / p},\]
  implying 
  \[\|y_i\|_q = \left(\sum |y_i|^q\right)^{1 / q} = 
  \left(\sum |y_i|^q\right)^{1 - 1 / p} = 
  \frac{\sum |y_i|^q}{\left(\sum |y_i|^q\right)^{1 / p}} 
  \le \|\phi\| < \infty,\]
  and \((y_i) \in \ell_q\).
\end{proof}

\begin{proposition}
  If \(X\) is a Hilbert space, then for all \(y \in X\), 
  \[T_y(x) := \langle y, x \rangle\]
  is a bounded linear functional on \(X\).
\end{proposition}
\begin{proof}
  Follows by the Cauchy-Schwarz inequality.
\end{proof}

In year two linear algebra we saw that all linear functionals on a finite 
dimensional space are of the form as above. In fact, this is true for 
all Hilbert spaces.

\begin{lemma}
  Let \(X\) be a Hilbert space and \(Y\) a closed subspace of \(X\). Then 
  \(X = Y \oplus Y^{\perp}\).
\end{lemma}
\begin{proof}
  \(Y\) is complete since it is a closed subspace of a complete space. 
  As \(Y \cap Y^{\perp} = \{0\}\), it suffices to show that \(Y + Y^{\perp} = X\). 
  Let \(x \in X\), then as subspaces are convex, by the nearest point property, 
  there exists some \(y \in Y\) such that \(\|y - x\| = \text{dist}(x, Y)\).
  Defining \(y' = x - y\), we have \(x = y + y'\) and it remains to show 
  \(y' \in Y^\perp\). 
  
  Suppose otherwise, there exists some \(\tilde y \in Y\) 
  such that \(\langle \tilde y, y'\rangle > 0\) (we may assume the by simply 
  multiplying by a scalar). Then, for \(t > 0\), 
  \[\begin{split}
    \|(y + t \tilde y) - x\|^2 
    & = \langle y + t\tilde y - x, y + t\tilde y - x \rangle\\
    & = \langle y - x, y - x \rangle + \langle t \tilde y, y - x \rangle + 
        \langle y - x, t \tilde y \rangle + t^2  \langle \tilde y, 
        \tilde y \rangle\\
    & = \text{dist}(x, Y)^2 - 2t\langle \tilde y, x\rangle + t^2 \|\tilde y\|^2
  \end{split},\]
  were for small enough \(t\), \(2t\langle \tilde y, x\rangle > t^2 \|\tilde y\|^2\) 
  and so, \(\|(y + t \tilde y) - x\| < \text{dist}(x, Y)\), contradiction! 
  \textbf{N.B. I am unclear about the last equality!}
\end{proof}

\begin{theorem}[Riesz Representation Theorem]
  Let \(X\) be a Hilbert space and \(l \in X^*\). Then there exists a unique 
  \(z \in X\) such that for all \(x \in X\), \(l(x) = \langle z, x \rangle\).
\end{theorem}
\begin{proof}
  Uniqueness is easy to check since, if 
  \(\langle z, \cdot\rangle = \langle z', \cdot\rangle\), then 
  \(\langle z - z', \cdot\rangle = 0\) implying \(z - z' = 0\). 

  If \(l = 0\) then \(z = 0\) suffices so suppose otherwise, i.e. 
  \(N := \ker l \neq X\). Then, as \(N = l^{-1}(\{0\})\) where 
  \(\{0\}\) is closed, \(N\) is also closed, and thus, by the above lemma 
  \(N \oplus N^{\perp} = X\) and so, there exists some \(z_0 \in N^\perp\) 
  \(z_0 \neq 0\). Define 
  \[v = l(x)z_0 - l(z_0) x.\]
  We see that \(l(v) = l(x)l(z_0) - l(z_0)l(x) = 0\) so \(v \in N\), and 
  hence \(\langle z_0, v \rangle = 0\). Unfolding the definition, we have 
  \[0 = \langle z_0, v \rangle = \langle z_0, l(x)z_0 - l(z_0) x \rangle 
      = l(x)\|z_0\|^2 - l(z_0)\langle z_0, x \rangle,\]
  implying 
  \[l(x) = \frac{l(z_0)\langle z_0, x \rangle}{\|z_0\|^2} = 
    \left\langle \frac{l(z_0)}{\|z_0\|^2}z_0, x \right\rangle.\]
  Hence, picking \(z = (l(z_0) /\|z_0\|^2)z_0\) suffices.
\end{proof}

With this, is easy to check that the map \(\phi : X \to X^* : 
v \mapsto \langle \cdot, v \rangle\) is in fact a isometric anti-isomorphism.

Let us now consider the dual space of \(\ell_p\) for \(p \in [1, \infty]\). 
In the case that \(p = 2\), we know that \(\ell_2\) is a Hibert space, and 
so the Riesz representation theorem provides an isomorphism 
\(\ell_2 \cong (\ell_2)^*\). In the case that \(p \in (1, \infty)\), as 
we have seen every linear functional of \(\ell_p\) is of the form 
\[T : \ell_p \to \mathbb{K} : x \mapsto \sum_i x_i y_i,\]
for some \(y \in \ell_q\), \(1 / p + 1 / q = 1\), and thus, by checking 
linearity, we fine \((\ell_p)^* \cong \ell_q\).

In the case the \(p = 1\), for all \(y \in \ell_\infty\), 
\(f_y : \ell_1 \to \mathbb{K}\) defined to be \(f_y(x) = \sum_i x_i y_i\) is 
a continuous linear functional since 
\[|f_y(x)| = \left| \sum_i x_i y_i \right| \le 
  \|y\|_\infty \left|\sum_i x_i \right| = \|y\|_\infty\|x\|_1 < \infty.\]
In particular, this provides an injection 
\(\ell_\infty \hookrightarrow (\ell_1)^*\). By the same argument, we can 
show \(\ell_1 \hookrightarrow (\ell_\infty)^*\).

\subsubsection{Dual Operators}

\begin{definition}[Dual Operator]
  Given a Hilbert space \(X\), and \(L \in \mathcal{L}(X, X)\), we define the 
  dual operator \(L^* : X^* \to X^*\) by 
  \[L^* f := f \circ L.\]
  We equip \(L^*\) with the operator norm 
  \[\|L^*\| = \sup_{f \neq 0} \frac{\|L^* f\|}{\|f\|}.\]
\end{definition}

\begin{proposition}
  \(\|L^*\| = \|L\|\) and the following properties hold,
  \begin{itemize}
    \item \((S + T)^* = S^* + T^*\);
    \item for all \(\alpha \in \mathbb{K}, (\alpha T)^* = \overline{\alpha} T^*\);
    \item \((T^*)^* = T\);
    \item \((ST)^* = T^* S^*\). 
  \end{itemize}
\end{proposition}

Consider \(X\) a Hilbert space and let \(T \in \mathcal{L}(X, X)\). Then 
by Riesz representation, for all \(f \in X^*\), there exists some \(z_{T^*f} \in X\) 
such that \(T^* f(x) = \langle z_{T^*f}, x \rangle\). On the other hand, as 
\(T^*f(x) = f(Tx)\), by Riesz, there exists some \(z_f\) such that 
\(T^*f(x) = f(Tx) = \langle z_f, Tx \rangle\). Comparing both equations, we 
have 
\[\langle z_{T* f}, x \rangle = \langle z_f, Tx\rangle = 
  \langle T^\dagger z_f, x\rangle,\]
where \(T^\dagger\) is the adjoint of \(T\). Thus, \(T^\dagger z_f = z_{T^* f}\).

\begin{definition}[Dual Operator again]
  If \(X_1, X_2\) are Hilbert spaces and \(T \in \mathcal{L}(X_1, X_2)\), 
  the dual operator of \(T\) is 
  \[T^* : X_2^* \to X_1^* : f \mapsto f \circ T.\]
\end{definition}

By Riesz representation, \(X_i^* \cong X_i\), and we may view \(T^*\) as a 
linear operator on \(X_2 \to X_1\). In particular, for all \(f \in X_2^*\), 
\(T^* f \in X_1^*\) so there exists some \(z_{T^* f} \in X_1\) such that 
\(T^* f(x) = \langle z_{T^* f}, x \rangle\). On the other hand, 
\(T^* f(x) = f(Tx)\) and so, there exists some \(z_f \in X_2\) such that 
\(T^* f(x) = \langle z_f, T x\rangle\). So, 
\[\langle z_f, T x \rangle = \langle z_{T^*f}, x\rangle.\]
Then, we may define \(T^\dagger : X_2 \to X_1\) such that 
\(T^\dagger z_f = z_{T^* f}\).

Consider \(\grad : D(\grad) \to L^2\) where 
\(D(\grad) \subseteq L^2\). Suppose for now that \(D(\grad) = C_0^\infty\), i.e. 
smooth, compactly supported functions. As demonstrated above, one may find the 
dual operator \(\grad^*\) of \(\grad\) such that 
\[\langle g, \grad \phi\rangle = \langle \grad^* g, \phi\rangle\]
for all \(g \in L^2\), \(\phi \in C_0^\infty\). This operator is known as 
the weak derivative and we shall treat this topic 
properly later in this course

\begin{proposition}
  Given a linear operator \(T \in \mathcal{L}(X_2, X_1)\) 
  where \(X_1, X_2\) are normed spaces, the map 
  \[h : X_1 \times X_2 \to \mathbb{K}: 
    (x, y) \in X_1 \times X_2 \mapsto \langle x, Ty \rangle\] 
  is a sesquilinear form.
\end{proposition}

\begin{definition}[Bounded Sesquilinear Form]
  A sesquilinear form \(h\) on a normed space is bounded if 
  \[\|h(\cdot, \cdot)\| := \sup_{x, y \neq 0} \frac{|h(x, y)|}{\|x\|\|y\|} < \infty.\]
\end{definition}

\begin{proposition}
  Let \(X_1, X_2\) be Hilbert spaces and let \(h\) be a bounded sesquilinear 
  form. Then \(h\) has the representation 
  \[h(x, y) = \langle Sx, y \rangle,\]
  for some \(S \in \mathcal{L}(X_1, X_2)\) such that \(\|S\| = \|h\|\).
\end{proposition}
\begin{proof}
  For all \(x \in X_1\), as \(h(x, \cdot) : X_2 \to \mathbb{K} \in X_2^*\), 
  by Riesz, there exists some \(z_x\) such that \(h(x, \cdot) = 
  \langle z_x, \cdot \rangle\). Thus, simply defining 
  \(S : X_1 \to X_2 : x \mapsto z_x\) suffices after simple checks for linearity.

  Consider 
  \[\begin{split}
    \|h\| & = \sup_{x, y \neq 0} \frac{|h(x, y)|}{\|x\|\|y\|} 
      = \sup_{x, y \neq 0} \frac{|\langle Sx, y \rangle|}{\|x\|\|y\|}
      \ge \sup_{x \neq 0} \frac{|\langle Sx, Sx \rangle|}{\|x\|\|Sx\|}
      = \sup_{x \neq 0} \frac{\|Sx\|}{\|x\|} = \|S\|.
  \end{split}\]
  On the other hand, 
  \[\|h\| = \sup_{x, y \neq 0} \frac{|\langle Sx, y \rangle|}{\|x\|\|y\|}
    \le \sup_{x, y \neq 0} \frac{\|Sx\|\|y\|}{\|x\|\|y\|} = \|S\|,\]
  where the inequality is due to Cauchy-Schwarz. Thus \(\|h\| = \|S\|\) as 
  required.
\end{proof}

\subsection{Tietze-Urysohn and Hahn-Banach}

\begin{lemma}[Urysohn's Lemma]
  For \(A, B \subseteq X\) closed with \(A \cap B = \varnothing\), then there 
  exists a continuous function such that \(h\mid_A = 1\) and \(h\mid_B = 0\).
\end{lemma}
\begin{proof}
  Consider 
  \[h(x) := \frac{\text{dist}(x, B)}{\text{dist}(x, A) + \text{dist}(x, B)}.\]
\end{proof}

\begin{theorem}[Tietze-Urysohn]
  If \(f : A \to \mathbb{R}\) is a continuous function where \(A\) is a 
  closed set in a metric space \(X\), then there exists a continuous function 
  \(\tilde f : X \to \mathbb{R}\) such that \(\tilde f\mid_A = f\) and 
  \(\|\tilde f\| = \|f\|\). 
\end{theorem}
\begin{proof}
  We may assume \(\|f\| = 1\) and \(0 \le f \le 1\) by scaling the function. 

  Define \(f_0 = f\) and \(f_{n + 1} = f_n - g_n\mid_A\) where \(g_n\) is defined 
  using Urysohn's Lemma such that 
  \[g_n(x) = \begin{cases}
    0, & x \in f_n^{-1}([0, \frac{1}{3}(\frac{2}{3})^n]);\\
    \frac{1}{3}\left(\frac{2}{3}\right)^n, & x \in 
      f_n^{-1}([(\frac{2}{3})^n, (\frac{2}{3})^{n + 1}]).
  \end{cases}\]
  In some sense, what we are doing taking \(g_{n + 1}\) to approximate 
  the errors of \(f_n - g_n\mid A\). 
  
  By construction, we see that \(0 \le f_n \le (\frac{2}{3})^n\), and 
  \(\sum^n_{i = 0} g_i = f - f_{n + 1}\), defining \(\tilde f = \sum_i g_i\) 
  (which exists as \(0 \le g_n \le \frac{1}{3}(\frac{2}{3})^n\), implying 
  \(\sum^n g_i\) is Cauchy, hence convergent by the completeness of \(C(A)\)), 
  we have 
  \[\sum_{i = 0}^n g_i\mid_C - f = - f_{n + 1} \to 0,\]
  and so \(\tilde f\mid_C = f\) as required.
\end{proof}

\begin{definition}
  A functional \(p : X \to \mathbb{R}\) is called sublinear if 
  \begin{itemize}
    \item \(p(x + y) \le p(x) + p(y)\);
    \item for all \(\alpha \ge 0\), \(p(\alpha x) = \alpha p(x)\).
  \end{itemize}
\end{definition}

\begin{theorem}[Hahn-Banach]
  Let \(X\) be a normed space, \(Z\) a proper subspace of \(X\), 
  \(p : X \to \mathbb{R}\) a sublinear functional and \(f : Z \to \mathbb{R}\) 
  such that \(f(x) \le p(x)\) for all \(x \in Z\). Then, there exists some 
  linear functional \(\tilde f : X \to \mathbb{R}\) such that 
  \[\tilde f\mid_Z = f \text{ and } \tilde f(x) \le p(x)\]
  for all \(x \in X\).
\end{theorem}
\begin{proof}
  Let us first consider the special case of extending \(f\) by 1-dimension.
  Let \(v \in X \setminus Z\) and \(W := \text{span}(Z, v)\). Define 
  \[\tilde f : W \to \mathbb{R} : z + \lambda v \mapsto f(z) + \lambda \alpha\]
  for some \(\alpha \in \mathbb{R}\). It suffices to show that 
  \[\tilde f(z + \lambda v) \le p(z + \lambda v).\]
  In particular, if \(\lambda > 0\), then 
  \(\alpha \le p\left(\frac{1}{\lambda} z + v\right) - 
    \tilde f\left(\frac{1}{\lambda}z\right)\),
  and if \(\lambda < 0\), then 
  \(\alpha \ge -p\left(\frac{1}{-\lambda}z - v\right) + 
    \tilde f\left(\frac{1}{-\lambda}z\right)\). Combining the two, we have 
  for all \(\lambda > 0\), 
  \[-p\left(\frac{1}{\lambda}z - v\right) + 
  \tilde f\left(\frac{1}{\lambda}z\right)
    \le \alpha \le p\left(\frac{1}{\lambda} z + v\right) - 
  \tilde f\left(\frac{1}{\lambda}z\right),\]
  and by rescaling, the condition becomes 
  \[-p\left(z - v\right) + \tilde f\left(z\right)
    \le \alpha \le p\left( z + v\right) - \tilde f\left(z\right).\]
  This holds if and only if 
  \[-p(z_1 - v) + \tilde f(z_1) \le p(z_2 + v) + \tilde f(z_2)\]
  for all \(z_1, z_2 \in Z\). Indeed, this holds as,
  \[\begin{split}
    \tilde f(z_1) + \tilde f(z_2) & = \tilde f(z_1 + z_2) 
      \le p(z_1 + z_2) = p((z_1 - v) + (z_2 + v)) \le p(z_1 - v) + p(z_2 + v).
  \end{split}\]
  With this 1-dimensional extension in mind, the result follows by Zorn's lemma.
  In particular, given two extensions of \(f\), \((M_1, g_1), (M_2, g_2)\) 
  where \(Z \subseteq M_1, Z \subseteq M_2\) and \(g_1\mid_Z = g_2\mid_Z = f\), 
  we define the partial order \(\prec\) such that \((M_1, g_1)\prec (M_2, g_2)\) 
  if and only if \(M_1 \subseteq M_2\) and \(g_2\mid_{M_1} = g_1\). Then, 
  for any chain of extensions \((M_\gamma, g_\gamma)_{\gamma \in T}\), define 
  \(M = \bigcup_{\gamma \in T} M_\gamma\) and 
  \[G : M \to \mathbb{R} : x \mapsto g_\gamma(x) \text{ if } x \in M_\gamma.\]
  It is clear that \(G\) is continuous and for all \((M_\gamma, g_\gamma)\), 
  we have \((M_\gamma, g_\gamma) \prec (M, G)\). Hence, every chain of extensions 
  has an upperbounds, and thus, by Zorn's lemma, there is an maximal extension 
  \((M, G)\). Finally, it suffices to show that \(M = X\) which is clear since 
  if otherwise, we may extend \(G\) with an additional dimension, contradicting 
  the maximality of \((M, G)\).
\end{proof}

\subsection{Application of Hahn-Banach}

In this section we will demonstrate some applications of the Hahn-Banach theorem.

\begin{corollary}[Existence of tangent functional]
  Let \(X\) be a normed space and let \(x \in X\), then there exists some 
  \(l \in X^*\) such that \(\|l\| = 1\) and \(l(x) = \|x\|\).
\end{corollary}
\begin{proof}
  Apply Hahn-Banach to \(f : \mathbb{R} \cdot x \to \mathbb{R} : 
    \lambda \mapsto \lambda \|x\|\).
\end{proof}

\begin{corollary}
  Let \(X\) be a normed space, for all \(x \in X\), 
  \[\|x\| = \sup \{|l(x)| \mid l \in X^*, \|l\| = 1\}.\]
\end{corollary}
\begin{proof}
  For all \(\|l\| = 1\), we have \(|l(x)| \le \|l\| \|x\| = \|x\|\). 
  Now, by the previous corollary, \(\|x\|\) is achieved by some 
  \(l \in X^*, \|l\| = 1\) and so,  
  \(\|x\| = \sup \{|l(x)| \mid l \in X^*, \|l\| = 1\}\) as required.
\end{proof}

\begin{theorem}[Banach Limit Theorem]
  There exists \(L \in (\ell_\infty)^*\) such that 
  \begin{itemize}
    \item \(\|L\| = 1\);
    \item if \(x \in c \subseteq \ell_\infty\), 
      then \(L(x) = \lim_{n \to \infty} x_n\);
    \item if \(x \in \ell_\infty\) and \(x_i \ge 0\) for all \(i\), then 
      \(L(x) \ge 0\);
    \item if \(x \in \ell_\infty\) and \(x'_n := x_{n + 1}\), then 
      \(L(x) = L(x')\).
  \end{itemize}
\end{theorem}

\begin{theorem}[Müntz-Szäsz]
  Suppose \(0 < \lambda_1 < \lambda_2 < \cdots\), and let \(X\) be the 
  closure of the set of linear combinations of \(t^{\lambda_j}\) in 
  \(\mathcal{C}[0, 1]\). Then 
  \begin{itemize}
    \item if \(\sum_j \frac{1}{\lambda_j} = \infty\), 
      then \(X = \mathcal{C}[0, 1]\);
    \item if \(\sum_j \frac{1}{\lambda_k} < \infty\), then for 
      \(\lambda \notin \{\lambda_j\}_{j \in \mathbb{N}}, t^\lambda \notin X\), 
      and so \(X \subsetneq \mathcal{C}[0, 1]\). 
  \end{itemize}
\end{theorem}

\begin{proposition}
  Let \(Y\) be a closed proper subspace of a normed space \(X\). Then, there 
  exists some \(f \in X^*\) such that \(\|f\| = 1\) and \(f(y) = 0\) for all 
  \(y \in Y\).
\end{proposition}
\begin{proof}
  Let \(x \in X \setminus Y\). If \(X = \text{span}(\{x\} \cup Y)\), then 
  the map \(\lambda x + y \mapsto \lambda\) suffices. On the other hand, 
  defining \(l : \lambda x + y \mapsto \lambda\), with Hahn-Banach, there exists 
  some \(f \in X^*\) such that \(f\mid_Y = l = 0\) and \(\|f\| = \|l\| = 1\).
\end{proof}

\begin{proposition}
  If \(X^*\) is separable, then so is \(X\).
\end{proposition}
\begin{proof}
  Let \(S^* := \{ \phi \in X^* \mid \|\phi\| = 1\}\). Then, as a subset of a 
  separable space is separable, \(S^*\) is separable by some
  \(\{\phi_n\} \subseteq S^*\). By definition, for each \(n\), we may define 
  \(x_n\) such that \(\|x_n\| = 1\) and \(\phi_n(x_n) > 1 / 2\). Now, 
  let 
  \[\mathcal{D} := \overline{\text{span}\{x_n \mid n \in \mathbb{N}\}}.\]
  By definition, it is clear that \(\mathcal{D}\) is dense by considering 
  the density of \(\mathbb{Q}\) in \(\mathbb{R}\) and thus, it suffices to 
  show \(\mathcal{D} = X\). 

  Suppose \(\mathcal{D} \neq X\), then by the previous proposition, 
  there exists a linear functional \(\phi \in S^*\) such that \(\phi(y) = 0\) 
  for all \(y \in \mathcal{D}\). Now, as \(\{\phi_n\}\) is dense in \(S^*\), 
  there exists \(n\) such that \(\|\phi - \phi_n\| < 1/2\). Hence,
  \[\begin{split}
    |\phi(x_n)| & \ge |\phi_n(x_n)| - |\phi_n(x_n) - \phi(x_n)| 
      > 1/2 - |(\phi_n - \phi)(x_n)| \\
    & \ge 1 / 2 - \|\phi - \phi_n\| \|x_n\| > 1 / 2 - 1 / 2 = 0.
  \end{split}\]
  But \(x_n \in \mathcal{D}\) implying \(\phi(x_n) = 0\), contradiction!
\end{proof}

\subsubsection{Stieltjes Integral}

Stieltjes integral is a method of defining integration similar to that 
of the Riemann integral (one may construct a similar version to the Lebesgue 
integral as well). In particular, given some function \(\omega\), 
we replace the individual segments \(x_{i + 1} - x_i\) with 
\(|\omega(x_{i + 1}) - \omega(x_i)|\).

\begin{definition}[Total Variation of Functions]
  The total variation of a function \(\omega\) on the interval \([a, b]\) is 
  \[V(\omega) := \sup_{p \in \mathcal{P}}
    \sum_{i = 1}^{|p| - 1} |\omega(x_{i + 1}) - \omega(x_i)|,\]
  where \(\mathcal{P}\) is the set of all partitions on \([a, b]\).
\end{definition}

In the case that the total variation is finite, we may define the lower 
Stieltjes sum of the function \(f\) to be 
\[s_{\omega}(p) := \sum_{i = 1}^{|p| - 1} 
  f(\underline{x_i})(\omega(x_{i + 1}) - \omega(x_i)),\]
and the upper sum to be 
\[S_{\omega}(p) := \sum_{i = 1}^{|p| - 1} 
  f(\overline{x_i})(\omega(x_{i + 1}) - \omega(x_i)),\]
where \(p\) is a partition, 
\(f(\underline{x_i}) = \min_{x \in [x_i, x_{i + 1}]} f(x)\) and
\(f(\overline{x_i}) = \max_{x \in [x_i, x_{i + 1}]} f(x)\).
We observe that 
\(|s_{\omega}|, |S_{\omega}| \le \|f\|_\infty V(\omega)\).

If \(\inf S_{\omega} = \sup s_\omega\) as the size of the segments of 
the partition tends to zero, then we define the Stieltjes integral of 
\(f\) with respect to \(\dd \omega\) to be 
\[\int_a^b f \dd \omega := \inf S_\omega.\]
In the case that \(\omega\) is differentiable then we see 
\(\int f \dd\omega = \int f \omega' \dd x\).

\begin{proposition}
  Let \(l \in \mathcal{C}([a, b])^*\) can be represented as 
  \[l(f) = l_\omega(f) := \int f \dd \omega,\]
  for some \(\omega\) with finite total variation and 
  \(\|l_\omega\| = V(\omega)\).
\end{proposition}
\begin{proof}
  By the Hahn-Banach theorem, as \(\mathcal{C}([a, b]) \subseteq 
  \mathcal{B}([a, b])\), there exists some \(\tilde l \in \mathcal{B}([a, b])^*\) 
  such that \(\|l\| = \|\tilde l\|\) and \(\tilde l\mid_{\mathcal{C}(a, b)} = l\).
  Define \(\omega(t) := \tilde l(\chi_{[a, t]})\) where \(\chi\) is the indicator 
  function, we see that \(\omega\) has bounded variation as 
  \[\begin{split}
    V(\omega) & = \sum |\omega(x_{i + 1} - x_i)| 
      = \sum |\tilde l(\chi_{[a, t]}(x_{i + 1}) - \chi_{[a, t]}(x_i))|\\
    & = \sum s_i (\tilde l(\chi_{[a, t]}(x_{i + 1}) - \chi_{[a, t]}(x_i))
      = \tilde l \sum s_i(\chi_{[a, t]}(x_{i + 1}) - \chi_{[a, t]}(x_i)\\
    & \le \|\tilde  l\| \left\|\sum s_i(\chi_{[a, t]}(x_{i + 1}) - \chi_{[a, t]}(x_i)\right\|
      = \|\tilde  l\| \left\|\sum s_i \chi_{[x_i, x_{i+1}]}(x)\right\|
  \end{split}
  \]
  where \(s_i = \text{sign}(\tilde l(\chi_{[a, t]}(x_{i + 1}) - \chi_{[a, t]}(x_i))\).
  Then, for all \(f \in \mathcal{C}([a, b])\) and a partition \(p\), define 
  \[g_n = \sum f(x_i) (\chi_{[a, t]}(x_{i + 1}) - \chi_{[a, t]}(x_i)).\]
  In particular, we note that \(\|g_n - f\|_\infty \to 0\) as \(n \to \infty\) 
  and thus, 
  \[l(f) = \tilde l(f) \leftarrow \tilde l(g_n) = 
  \sum f(x_i) (\chi_{[a, t]}(x_{i + 1}) - \chi_{[a, t]}(x_i)) = \int_a^b f \dd \omega. \]
\end{proof}

\newpage
\section{Uniform Boundedness Principle}

In this section we will prove the uniform boundedness principle, also 
known as the Banach-Steinhaus theorem. To achieve this we will first prove 
the Baire's category theorem. We will also provide some application of this 
principle.

\subsection{Baire's Category Theorem}

\begin{definition}[Baire's Category]
  Let \((X, \rho)\) be a metric space. Then a subspace \(M \subseteq X\) is 
  said to be 
  \begin{itemize}
    \item nowhere dense if \((\overline{M})^\circ = \varnothing\);
    \item of the 1\textsuperscript{st} category (or meager) if there exists 
      a sequence of nowhere dense set \((M_n)_{n \in \mathbb{N}}\) such that
      \(M = \bigcup_{n \in \mathbb{N}} M_n\);
    \item of the 2\textsuperscript{nd} category (or nonmeager) if it is 
      not of the first category.
  \end{itemize}
\end{definition}

An example of a meager set is \(\mathbb{Q} \subseteq \mathbb{R}\). Indeed, as 
\(\mathbb{Q}\) is countable, it is a countable union of singletons, each of 
which are nowhere dense.

\begin{theorem}[Baire's Category Theorem]
  If a metric space \(X \neq \varnothing\) is complete, then it is of 
  the 2\textsuperscript{nd} category. Thus, if \(X \neq \varnothing\) and 
  \[X = \bigcup_{k = 1}^\infty A_k,\]
  for some sequence of sets, then at least one \(A_k\) is not nowhere 
  dense, i.e. contains a non-empty open subset.
\end{theorem}
\begin{proof}
  Suppose otherwise, i.e. \(X = \bigcup A_n\) where \(A_n\) is nowhere dense 
  for all \(n \in \mathbb{N}\). By definition \(\overline{A_n}\) does not 
  contain a non-empty open set and so, \(\overline{A_n} \neq X\) (as 
  \(X\) contains itself which is a non-empty open set) and 
  \(\overline{A_n}^c \neq \varnothing\) is open. 

  Let \(p_1 \in \overline{A_1}^c\) and let \(\epsilon_1 \in (0, 1/2)\) such 
  that \(B_{\epsilon_1}(p_1) \subseteq \overline{A_1}^c\). Now, as 
  \(A_2\) is nowhere dense, \(B_{\epsilon_1}(p_1) \not\subseteq A_2\) and 
  so \(\overline{A_2}^c \cap B_{\epsilon_1}(p_1)\) is a non-empty open 
  set. Then, we may choose \(p_2 \in \overline{A_2}^c \cap B_{\epsilon_1}(p_1)\) 
  and \(\epsilon_2 \in (0, 1/4)\) such that \(B_{\epsilon_2}(p_2) \subseteq 
  \overline{A_2}^c \cap B_{\epsilon_1}(p_1)\). Repeating this process, we 
  may find \(p_{n + 1} \in \overline{A_{n + 1}}^c \cap B_{\epsilon_n}(p_n)\) 
  and \(\epsilon_{n + 1} \in (0, 1/2^{n + 1})\) such that 
  \[B_{\epsilon_{n + 1}}(p_{n + 1}) \subseteq 
  \overline{A_{n + 1}}^c \cap B_{\epsilon_n}(p_n).\]
  It is clear that \((p_n)_{n \in \mathbb{N}}\) is Cauchy as for all 
  \(n, m \ge N\), \(p_n, p_m \in B_{\epsilon_N}(p_N) \subseteq 
  B_{1/2^N}(p_N)\) and thus,
  \[d(p_n, p_m) \le d(p_n, p_N) + d(p_N, p_m) < 
  \frac{1}{2^N} + \frac{1}{2^N} = \frac{1}{2^{N - 1}} \to 0\]
  as \(N \to \infty\). Hence, as \(X\) is complete, there exists some 
  \(p \in X\) such that \(p_n \to p\) as \(n \to \infty\). Now, by the 
  triangle inequality, for all \(n\), 
  \[d(x, x_n) \le d(x, x_{n + k}) + d(x_{n + k}, x_n) < 
  d(x, x_{n + k}) + \frac{1}{2^n},\]
  and so, taking \(k\to \infty\), \(d(x, x_{n + k}) \to 0\) and so, 
  \(d(x, x_n) < 1/2^n\) implying \(x \in B_{\epsilon_n}(p_n)\). But then, 
  \(x \in \bigcap B_{\epsilon_n}(p_n) \subseteq \bigcap \overline{A_n}^c\) 
  implying \(x \not\in \bigcup A_n\), and hence, \(\bigcup A_n \neq X\).
\end{proof}

\subsection{Banach-Steinhaus Theorem}

\begin{theorem}[Banach-Steinhaus]
  Let \((T_n)_{n \in \mathbb{N}}\) be a sequence of bounded linear operators 
  from a Banach space \(X\) to a normed space \(Y\) such that for all 
  \(x \in X\), there exists \(c_x \in (0, \infty)\) and 
  \[\sup_{n \in \mathbb{N}} \|T_n x\| \le c_x.\]
  Then, there exists some \(c \in (0, \infty)\) such that 
  \[\sup_{n \in \mathbb{N}} \|T_n\| \le c.\]
\end{theorem}
\begin{proof}
  For \(k \in \mathbb{N}\), define 
  \[A_k := \{x \in X \mid \|T_n x\| \le k, n \in \mathbb{N}\}.\]
  It is easy to see that \(A_k\) is closed for each \(k\). Indeed, if 
  \(x_n \in A_k\) such that \(x_n \to x \in X\), then \(\|T_n x\| \le k\) 
  for all \(n\) by the continuity of the \(\|T_n(\cdot)\|\). Furthermore, 
  as \(X = \bigcup A_k\) (as \(T_n\) is pointwise uniformly bounded), 
  by the Baire's category theorem, there exists some \(k_0 \in \mathbb{N}\),
  \(x_0 \in X\), \(r > 0\), such that 
  \[B_r(x_0) \subseteq A_{k_0}.\]
  Then, for all \(x \in X\setminus\{0\}\), let \(\gamma_x := r / (2\|x\|)\) 
  and \(z := x_0 + \gamma x\). It is clear that \(z \in B_r(x_0)\) and so, 
  for all \(n \in \mathbb{N}\),
  \[k_0 \ge \|T_n z\| = \|T_n (x_0 + \gamma x)\| \ge 
  |\gamma \|T_n x\| - \|T_n x_0\|| \ge 
  r \frac{\|T_n x\|}{\|x\|} - \|T_n x_0\|.\]
  Hence, \(\|T_n x\|/\|x\| \le (k_0 + \|T_n x_0\|)/r\), and since 
  \(\|T_n x_0\| \le \sup_n \|T_n x_0\| \le c_{x_0}\), we have the bound 
  \(\|T_n x\|/\|x\| \le (k_0 + c_{x_0})/r\). Thus, \(n\) and \(x\) are 
  arbitrary, we have 
  \[\sup_{n \in \mathbb{N}} \|T_n\| \le \frac{1}{r}(k_0 + c_{x_0}).\]
\end{proof}

\begin{proposition}
  Let \(X\) be the space of polynomials equipped with the norm 
  \(\|a_0 + a_1 t + \cdots + a_n t^n\| := \max_{i = 0, \cdots, n} |a_i|\). 
  Then \(X\) is not complete.
\end{proposition}
\begin{proof}
  Define \(T_n(p(t)) := \sum_{i = 0}^n a_i\) where \(p(t) = \sum_{i = 0}^m a_i t^i\).
  Clearly, \(T_n\) is linear and for all \(p(t) = \sum_{i = 0}^m a_i t^i\in X\), 
  \(\|p\| = 1\), \(a_i \le 1\) for all \(i = 1, \cdots, m\) implying 
  \(\|T_n(p)\| = |\sum^n a_i| \le n + 1\). Now, fixing 
  \(p(t) = \sum_{i = 0}^m a_i t^i\in X\), 
  we have \(\|T_n p\| \le \sum_{i = 0}^m |a_i|\),
  and so, if \(X\) is complete, by the Banach-Steinhaus theorem, 
  \(\sup_n \|T_n\|\) is bounded. But for all \(n\), we have 
  \[\|T_n(1 + x + \cdots + x^n)\| = \sum_{i = 0}^n 1 = n + 1,\]
  where \(\|1 + x + \cdots + x^n\| = 1\), we have \(\|T_n\| \ge n + 1\). 
  Thus, \(\sup_n \|T_n\| \ge \sup_n (n + 1) = \infty\) implying 
  \(\sup_n \|T_n\|\) is not bounded and hence, \(X\) is not complete.
\end{proof}

The Banach-Steinhaus theorem also allows us to show the existence of a 
continuous function which has Fourier series divergent at a point.

Let \(u \in \mathcal{C}([0, 2\pi], \mathbb{R})\), and recall that it 
has Fourier coefficients defined by 
\[a_m := \frac{1}{\pi} \int_0^{2\pi} u(t) \cos(mt) \dd t; \ 
  b_m := \frac{1}{\pi} \int_0^{2\pi} u(t) \sin(mt) \dd t,\]
and its Fourier series is defined by 
\[\mathcal{F}_u := \sum_{m \in \mathbb{N}}(a_m \cos(mt) + b_m\sin(mt)).\]
Let \(X := \mathcal{C}([0, 2\pi], \mathbb{R})\) and define the sequence of 
linear operator \(T_n\) on \(X\) such that, for all \(x \in X\),
\[T_n x := \frac{1}{2}a_0 + \sum_{m = 1}^n a_m = 
  \frac{1}{\pi}\int_0^{2\pi}x(t)\left(\frac{1}{2} + 
  \sum_{m = 1}^n \cos(mt)\right) \dd t.\]
That is \(T_n x\) is the Fourier series at \(t = 0\) resumed only up to 
the \(n\)-th term. We will find some \(x \in X\) such that \(\|T_n x\|\) 
is not uniformly bounded, and so, the Fourier series of \(x\) diverges.
Noting that 
\[\begin{split}
  2 \sin\left(\frac{1}{2}t\right) \sum_{m = 1}^n \cos(mt) & = 
  \sum_{m = 1}^n \left(
    \sin\left(\left(m + \frac{1}{2}\right)t\right) - 
    \sin\left(\left(m - \frac{1}{2}\right)t\right)
  \right)\\
  & = \sin\left(\left(n + \frac{1}{2}\right)t\right) - 
    \sin \left(\frac{1}{2}\right),
\end{split}\]
as the second sum is telescoping, we have 
\[1 + 2 \sum_{m = 1}^n \cos(mt) = 
\frac{\sin\left(\left(n + \frac{1}{2}\right)t\right)}
{\sin\left(\frac{1}{2}t\right)} =: q_n(t).\]
Thus, 
\[T_n x = \frac{1}{2\pi}\int_0^{2\pi} x(t)q_n(t) \dd t.\]
With this representation in mind, we see that 
\[\|T_n x\| \le \frac{1}{2\pi} \|x\| \int_0^{2\pi}|q_n(t)| \dd t,\]
and so, \(\|T_n\| \le \frac{1}{2\pi}\int_0^{2\pi}|q_n(t)| \dd t\).
On the other hand, write \(y(t) := \text{sign}(q_n(t))\) so that 
\(|q(t)| = |y(t)q(t)|\). While \(y\) is not continuous, it can be 
approximated by continuous functions of norm 1 in the sense that for all 
\(\epsilon > 0\), there exists a continuous function \(x\) of norm 1 such that 
\[\epsilon > \frac{1}{2\pi}\left|\int_0^{2\pi} (x(t) - y(t))q_n(t) \dd t \right|
  = \left|T_n x - \frac{1}{2\pi} \int_0^{2\pi} |q_n(t)|\right|.\]
Thus, as \(\int_0^{2\pi} |q_n(t)| \dd t \to \infty\) as \(n \to \infty\), 
there must be some \(x \in X\) such that \(\|T_n x\|\) is not uniformly 
bounded as otherwise, this would contradict Banach-Steinhaus. 

Let us now turn our focus back to \(\ell_p\) spaces. 

\begin{proposition}
  Let \(p, q \in (1, \infty)\) 
  such that \(1 / p + 1 / q = 1\) and suppose \((a_n)_{n \in \mathbb{N}}\) 
  is a sequence of complex numbers such that 
  \(|\sum a_n x_n | < \infty\) for all \((x_n) \in \ell_p\).
  Then \((a_n) \in \ell_q\).
\end{proposition}
\begin{proof}
  WLOG. let us assume \((a_n)\) does not have any 0 terms.
  Let \(f_n(x) := \sum_{i = 0}^n a_i x_i\). It is not difficult to see that 
  \((f_n)\) are bounded linear and is pairwise uniformly bounded. Now, 
  by Hölder, we have 
  \[|f_n(x)| \le \left(\sum_{i = 1}^n |a_i|^q\right)^{1 / q} \|x\|p,\]
  and so,
  \(\|f_n\| \le \left(\sum_{i = 1}^n |a_i|^q\right)^{1 / q}\). On the other 
  hand, this bound is achieved by taking 
  \[x_i := \begin{cases}
    \frac{\overline{a_k}|a_i|^{q - 2}}{(\sum_j^n |a_j|^q)^{1 - q / q}}, 
    \ k = 1,\cdots, n,\\
    0, \ \text{otherwise}.
  \end{cases}\]
  So, \(\left(\sum_{i = 1}^n |a_i|^q\right)^{1 / q} = \|f_n\|\) and by 
  Banach-Steinhaus, 
  \[\|a\|_q = \left(\sum_{i = 1}^\infty |a_i|^q\right)^{1 / q} = 
    \sup_n \|f_n\| < \infty.\]
\end{proof}

\subsection{Open Mapping Theorem}

\begin{definition}
  Let \(X, Y\) be metric spaces, then the function \(T : D(T) \subseteq X \to Y\) 
  is an open mapping if for all \(U \subseteq D(T)\) open, \(T(U)\) is open 
  in \(Y\).
\end{definition}

\begin{theorem}[The Open Mapping Theorem]
  A bounded linear operator \(T : X \to Y\) between Banach spaces \(X, Y\) 
  is an open mapping if it is surjective.
\end{theorem}
\begin{proof}
  Let \(B^X_r(x), B^Y_r(y)\) denote open balls of radius \(r\) in \(X\) and \(Y\) 
  respectively and \(B^X_r, B^Y_r\) denote open balls centred at the origin with 
  radius \(r\). 
  
  Since \(T\) is linear, it suffices to show that \(T(B_1^X)\) contains an 
  open ball containing \(0\) in \(Y\). Indeed, for all open 
  \(A \subseteq D(T)\), if \(y \in T(A)\), by definition, there exists some 
  \(x \in A\) such that \(T(x) = y\). Now, as \(A\) is open, there exists some 
  \(B_r^X(x) \subseteq A\), and so 
  \(B^X_1 = \frac{1}{r} (B_r^X(x) - x) \subseteq \frac{1}{r} (A - x)\). 
  Then, by assumption, there exists some \(\epsilon > 0\) such that 
  \[B_\epsilon^Y \subseteq T(B_1^X) \subseteq T\left(\frac{1}{r}(A - x)\right) = 
    \frac{1}{r}(T(A) - T(x)) = \frac{1}{r}(T(A) - y).\]
  Thus, \(r B_\epsilon^Y + y \subseteq T(A)\). Now, since 
  \(y \in r B_\epsilon^Y + y \subseteq T(A)\) we have found an open ball 
  containing \(y\) which is contained in \(T(A)\). Hence, \(T(A)\) is open 
  and \(T\) is an open mapping.

  We will first show \(T(B^X_{1 / 2}(0))\) contains an open ball \(B^*\) not 
  necessary centred at the origin. Consider 
  \[Y = T(X) = T\left(\bigcup_{k = 1}^\infty k B_{1 / 2}(0)\right) 
    = \bigcup_{k = 1}^\infty k T(B_{1 / 2}(0))
    = \bigcup_{k = 1}^\infty \overline{k T(B_{1 / 2}(0))}.\]
  By the Baire category theorem, at least on of the 
  \(\overline{k T(B_{1 / 2}(0))}\) must contain an open ball, and thus, by 
  rescaling, we obtain an open ball \(B^* = B_\epsilon(y_0) 
  \subseteq \overline{T(B_{1 / 2}(0))}\). 
  
  Now, as \(y_0 \in B_\epsilon(y_0) \subseteq \overline{T(B_{1 / 2}(0))}\) 
  and, for all \(y \in B_\epsilon(0) = B_\epsilon(y_0) - y_0\), we have 
  \(y + y_0 \in B_\epsilon(y_0) \subseteq \overline{T(B_{1 / 2}(0))}\),
  taking sequences \((u_n), (v_n) \subseteq T(B_{1 / 2}(0))\) such that 
  \(u_n \to y + y_0\) and \(v_n \to y_0\), there 
  exists some \((w_n), (z_n) \subseteq B_{1 / 2}(0)\) such that 
  \(T w_n = u_n \to y + y_0, T z_n = v_n \to y_0\). Then, 
  by sequential continuity
  \[T(w_n - z_n) = T w_n - T z_n = u_n - v_n \to y,\]
  and \(w_n - z_n \in B_1(0)\) as \(\|w_n - z_n\| \le \|w_n\| + \|z_n\| 
  < 1 / 2 + 1 / 2 = 1\). Hence, \(y \in \overline{T(B_1(0))}\) and thus, 
  \(B_\epsilon(0) \subseteq \overline{T(B_1(0))}\).
  
  Finally, we will show that \(B_{\epsilon / 2}(0) \subseteq T(B_1(0))\). 
  By linearity, we have 
  \[V_n := B_{\epsilon / 2^n}(0) \subseteq 2^{-n} \overline{T(B_1(0))} = 
  \overline{T(B_{1 / 2^n}(0))} =: \overline{T(B_n)}.\]
  Let \(y \in B_{\epsilon / 2} = V_1\), then \(y \in \overline{T(B_1)}\) 
  and thus, there exists some \(x_1 \in B_1\) such that \(\|y - Tx_1\| < \epsilon / 4\).
  This means that \(y - Tx_1 \in V_2 \subseteq \overline{T(B_2)}\) and there 
  exists some \(x_2 \in B_2\), \(\|y - Tx_1 - Tx_2\| < \epsilon / 8\). 
  Repeating this process, we find a sequence \((x_n)\) such that \(x_n \in B_n\) 
  and 
  \[\left\| y - \sum_{k = 1}^n Tx_k \right\| < \frac{\epsilon}{2^{n + 1}}.\]
  Considering for \(n > m\), 
  \[\left\|\sum_{k = 1}^n Tx_k - \sum_{k = 1}^m Tx_k\right\| = 
    \left\|\sum_{k = m + 1}^n Tx_k\right\| \le 
    \sum_{k = m + 1}^n \| Tx_k\| \le \sum_{k = m + 1}^\infty \|T x_k\| < 
    \sum_{k = m + 1}^\infty \frac{1}{2^k} \to 0,\]
  as \(m \to \infty\), we have \(S_n := \sum_{k = 1}^n T x_k\) is Cauchy, and 
  thus converges to some \(s \in Y\). As \(\|y - S_n\| < \epsilon / 2^{n + 1}\), 
  \(\|y - s\| \le \|y - S_n\| + \|S_n - s\| \to 0\) as \(n \to \infty\), 
  \(\|y - s\| = 0\) and thus, \(y = s\).
  
  Now, \(\sum_{k = 1}^n x_k\) converges to some \(x \in X\) as it is 
  absolutely convergent since \(x_n \in B_n\) implying \(\|x_n\| < 1 / 2^n\) 
  and so \(\sum \|x_n\| < \sum 1 / 2^n = 1 < \infty\). Thus, by sequential 
  continuity, we have 
  \(T(\sum_{k = 1}^n x_k) \to T x\) as \(n \to \infty\) and so, \(Tx = y\).
  Finally, \(x \in B_1(0)\) since addition is continuous, and so 
  \[\|x\| = \lim_{n \to \infty} \left\| \sum_{k = 1}^n x_n \right\|
    \le \lim_{n \to \infty} \sum_{k = 1}^n \|x_n\| < 
    \lim_{n \to \infty} \sum_{k = 1}^n \frac{1}{2^k} = 1,\]
  implying \(x \in B_1(0)\) and thus, as \(y = Tx\), \(y \in T(B_1(0))\).
\end{proof}

With the open mapping theorem, we see that a bijective bounded linear operator 
is automatically a homeomorphism and \(T^{-1}\) is bounded.

\begin{theorem}[Bounded Inverse Theorem]
  Let \(X, Y\) be Banach spaces and suppose \(T : X \to Y\) is a bijective, 
  bounded linear operator. Then, so is \(T^{-1} : Y \to X\).
\end{theorem}
\begin{proof}
  The inverse of a bijective linear operator is linear and so it suffices to 
  show boundedness. However, this follows, as boundedness is equivalent to 
  continuous which follows by the open mapping theorem.
\end{proof}

\subsection{Closed Graph Theorem}

\begin{definition}[Closed Linear Operator]
  Let \(X, Y\) be normed spaces and \(T : D(T) \subseteq X \to Y\) be a linear 
  operator. The graph of \(T\) is defined to be the set 
  \[\mathcal{G}(T) := \{(x, Tx) \in X \times Y \mid x \in D(T)\}.\]
  \(T\) is said to be a closed linear operator if its graph is closed 
  in the normed space equipped with the norm \(\|(x, y)\| = \|x\|_X + \|y\|_Y\).
\end{definition}

\begin{proposition}
  Let \(X, Y\) be normed spaces and \(T : D(T) \subseteq X \to Y\) be a 
  linear operator. Then \(T\) is a closed linear operator if and only if 
  for all \((x_n) \subseteq D(T)\), \(x_n \to x\) and \(T x_n \to y\), 
  we have \(x \in D(T)\) and \(Tx = y\).
\end{proposition}
\begin{proof}
  Suppose \((x_n, Tx_n) \in \mathcal{G}(T)\) such that \((x_n, Tx_n) \to 
  (x, y) \in X \times Y\). Then 
  \[0 \leftarrow \|(x_n, T x_n) - (x, y)\| = \|(x_n - x, T x_n - y)\| 
    = \|x_n - x\|_X + \|T x_n - y\|_Y,\]  
  and so, \(x_n \to x\) and \(T x_n \to y\) and thus, \(T x = y\) implying 
  \((x, y) \in \mathcal{G}(T)\) and \(\mathcal{G}(T)\) is closed.

  Conversely, if \(\mathcal{G}(T)\) is closed, then for all \(x_n \to x\), 
  \(T x_n \to y\), \((x, y)\) is a limit point of \(\mathcal{G}(T)\). Thus,,
  as \(\mathcal{G}(T)\) is closed, \((x, y) \in \mathcal{G}(T)\) and so, 
  \(y = Tx\).
\end{proof}

\begin{corollary}
  Let \(X, Y\) be normed spaces and \(T : D(T) \subseteq X \to Y\) be a 
  continuous linear operator. Then \(T\) is a closed linear operator if 
  \(D(T)\) is closed.
\end{corollary}
\begin{proof}
  Suppose \(x_n \to x \in X\) and \(T x_n \to y\). As \(D(T)\) is closed, 
  \(x \in D(T)\), and by sequential continuity, \(T x = y\).
\end{proof}

\begin{theorem}[Closed Graph Theorem]
  Let \(X, Y\) be Banach spaces and \(T : D(T) \subseteq X \to Y\) be a closed 
  linear operator. Then if \(D(T)\) is closed in \(X\), then \(T\) is bounded.
\end{theorem}
\begin{proof}
  We first observe that \(X \times Y\) is complete. Indeed, if 
  \(((x_n, y_n))_{n \in \mathbb{N}}\) is Cauchy, then so are \((x_n), (y_n)\). 
  Thus, as both \(X, Y\) are Banach, there exists some \(x \in X, y \in Y\), 
  \(x_n \to x, y_n \to y\) and so, 
  \[\|(x_n, y_n) - (x, y)\| = \|x_n - x\| + \|y_n - y\| \to 0.\]
  Furthermore, since \(\mathcal{G}(T)\) and \(D(T)\) are by assumption closed, 
  they are also complete as they are closed subspaces of a Banach space.

  Now, define 
  \[P : \mathcal{G}(T) \to D(T) : (x, Tx) \to x.\]
  It is clear that \(P\) is linear and is bounded since for all \((x, Tx) \in 
  \mathcal{G}(T)\), \(\|(x, Tx)\| = 1\), we have 
  \(1 = \|(x, Tx)\| = \|x\| + \|Tx\| \ge \|x\| = \|P(x, Tx)\|\). Furthermore, 
  \(P\) is bijective with the inverse \(P^{-1}(x) = (x, Tx)\) which is also 
  a bounded linear operator (as the inverse of a bounded linear operator is 
  bounded). But, this implies 
  \[\|Tx\| \le \|x\| + \|Tx\| = \|(x, Tx)\| = \|P^{-1}(x)\| \le \|P^{-1}\|\|x\|,\]
  proving \(T\) is bounded.
\end{proof}

\newpage
\section{Spectral Theory}

In this section we will introduce some tools applicable for the study of PDEs.

\subsection{Weak Convergence}

\begin{definition}[Weak Convergence]
  A sequence \((x_n)\) in a normed space \(X\) is said to be weakly convergent 
  if there exists some \(x \in X\) such that for all \(f \in X^*\), 
  \[\lim_{n \to \infty} f(x_n) \to f(x).\]
  In this case, we say \((x_n)\) converges weakly to \(x\) and denote it by 
  \(x_n \weak x\). In contrast to this, if \((x_n)\) converges in the norm
  sense, then we say \((x_n)\) converges strongly.
\end{definition}

\begin{proposition}
  Let \((x_n)\) be a sequence in a normed space \(X\) which converges weakly, 
  then 
  \begin{itemize}
    \item it has a unique weak limit;
    \item every subsequence of \((x_n)\) converges to the same limit;
    \item \(\{\|x_n\|, n \in \mathbb{N}\}\) is bounded.
  \end{itemize}
\end{proposition}
\begin{proof}
  If \(x_n \weak x\) and \(x_n \weak y\), then by definition, for all 
  \(f \in X^*\), \(f(x) \leftarrow f(x_n) \to f(y)\) and so, as limits are 
  unique in the underlying field, \(f(x) = f(y)\) and so \(f(x - y) = 0\).
  If \(x \neq y\) then we may define a non-zero linear functional from the 
  span of \(x - y\) which may be extended to the whole space by Hahn-Banach. 
  Thus, \(x - y = 0\) and so \(x = y\).

  A subsequence of \(f(x_n)\) converges to \(f(x)\) for all \(f \in X^*\) through 
  usual convergence and so the claim is as required.

  Since \(f(x_n) \to f(x)\) for all \(f \in X^*\), \(|f(x_n)|\) is bounded by 
  some \(c_f\). Define \(g_n \in (X^*)^* : f \mapsto f(x_n)\), we have 
  \(\sup_{n \in \mathbb{N}} \|g_n(f)\| \le c_f\), and so by Banach-Steinhaus, 
  there exists some \(c\) such that \(\sup_{n \in \mathbb{N}} \|g_n\| \le c\).
  Now, since \(g_n = \phi(x_n)\) where \(\phi : X \to (X^*)^*\) is the canonical 
  isometry, we have \(\|g_n| = \|\phi(x_n)\| = \|x_n\|\). Thus, \(\|x_n\|\) is 
  bounded by \(c\). 
\end{proof}

\begin{proposition}
  Let \((x_n)\) be a sequence in the normed space \(X\) and \(x_n \to x\), 
  then \(x_n \weak x\).
\end{proposition}
\begin{proof}
  Clear by sequential continuity.
\end{proof}

Weak convergence is a infinite dimensional construct as in the finite dimensional 
case, weak convergence and convergence in norms is equivalent.

\begin{proposition}
  Let \((x_n)\) be a sequence in the finite dimensional normed space \(X\) 
  such that \(x_n \weak x\). Then \(x_n \to x\).
\end{proposition}
\begin{proof}
  Let \(\{e_i\}_{i = 1}^n\) be a basis of \(X\). Recall that every linear map 
  is continuous in a finite dimensional normed space and so, the linear map 
  defined by \(f_i(e_j) = \delta_{ij}\) is in \(X^*\). Thus, if we write 
  \(x_n = \sum_i \lambda_i^n e_i, x = \sum_i \lambda_i e_i\), we have 
  \[\lambda_i^n = f_i(x_n) \to f_i(x) = \lambda_i.\]
  Thus, 
  \[\|x_n - x\| = \left\|\sum_{i = 1}^n (\lambda_i^n - \lambda_i)e_i \right\| 
  \le \sum_i |\lambda_i^n - \lambda_i| \|e_i\| \to 0.\] 
\end{proof}

As one would hope, these two notions are not equivalent in the infinite dimensional 
case. Consider the sequence \(e_i\) in \(\ell_2\) where 
\(e_i = (\delta_{ij})_{j = 1}^\infty\). By recalling that there is a 
canonical isomorphism between \((\ell_p)^*\) and \(\ell_q\) for all 
\(1 / p + 1 / q = 1\), we have \((\ell_2)^* \cong \ell_2\). Thus, for all 
\(f \in (\ell_2)^*\), \(f(e_i) = \sum_j f_je_i = f_i \to 0\) as \(i \to \infty\).
Thus, \(e_i \weak 0\). On the other hand, \(e_i\) does not converge to \(0\) since 
\(\|e_i - 0\| = \|e_i\| = \sqrt{2}\) for all \(i\).

While weak convergence seems uncontrollable, we have a good understanding 
of it behaviour in some classical spaces and in Hilbert spaces. The next 
two proposition illustrates this.

\begin{proposition}
  Let \((x_n)\) be a sequence in \(X = c_0\) or \(\ell_p\) for \(p \in (1, \infty)\).
  Then, \(x_n \weak x\) if and only if \((x_n)\) is bounded and converges to 
  \(x\) pointwise, i.e. \(x_n(j) \to x(j)\) for all \(j \in \mathbb{N}\).
\end{proposition}
\begin{proof}[Proof of the \(\ell_p\) case]
  The forward direction is obvious so let us consider the converse. Suppose 
  \((x_n) \in \ell_p\) is bounded by \(M\) (so that \(\|x_n - x\|_p \le 
  \|x_n\|_p + \|x\|_p \le 2M\)) and converges point-wise to \(x\). 
  Then, for all \(f \in (\ell_p)^* \cong \ell_q\), \(\epsilon > 0\), 
  choose \(N \in \mathbb{N}\) such that 
  \[\left(\sum_{j = N + 1}^\infty |f_j|^q\right)^{1 / q} < \frac{\epsilon}{2M}.\]
  Then, for sufficiently large \(n\), we have 
  \[\left|\sum_{j = 1}^N (x_n(j) - x(j))f_j\right| < \frac{\epsilon}{2}.\]
  Thus, 
  \[\begin{split}
    |f(x_n) - f(x)| & = \left|\sum_{j = 1}^\infty(x_n(j) - x(j))f_j\right|
      \le \left|\sum_{j \le N}(x_n(j) - x(j))f_j\right| + 
          \left|\sum_{j > N}(x_n(j) - x(j))f_j\right|\\
    & < \frac{\epsilon}{2} + \left(\sum_{j > N}(x_n(j) - x(j))^p\right)^{1 / p}
        \left(\sum_{j > N} |f_j|^q\right)^{1 / q}
      < \frac{\epsilon}{2} + 2M \frac{\epsilon}{2M} = \epsilon
  \end{split}\]
  implying \(x_n \weak x\) as required.
\end{proof}

\begin{proposition}
  Let \((x_n)\) be a sequence in a Hilbert space \(\mathcal{H}\) such that 
  \(x_n \weak x\) and \(\|x_n\| \to \|x\|\), then \(x_n \to x\) in 
  \(\mathcal{H}\).
\end{proposition}
\begin{proof}
  Consider \(\|x_n - x\|^2 = \|x_n\|^2 + \|x\|^2 - \langle x_n, x \rangle - 
  \langle x, x_n \rangle\). Defining \(f_x(y) := \langle y, x\rangle\), 
  we have \(f_x \in \mathcal{H}^*\) and so, \(\langle x_n, x \rangle = 
  f_x(x_n) \to f_x(x) = \langle x, x\rangle = \|x\|^2\). On the other hand, 
  \(\langle x, x_n \rangle = \overline{f_x(x_n)} \to \overline{\|x\|^2} = 
  \|x\|^2\). Thus, 
  \[\|x_n - x\|^2 = \|x_n\|^2 + \|x\|^2 - \langle x_n, x \rangle - 
  \langle x, x_n \rangle \to 2\|x\|^2 - 2\|x\|^2 = 0.\]
\end{proof}

\begin{definition}[Weak Topology]
  Let \(X\) be a Banach space. The weak topology on \(X\) is the topology with 
  the base with elements of the form 
  \[U_{\epsilon_i}^{f_i}(x) := \{y \in X \mid |f_i(y) - f_i(x)| < \epsilon_i\}\]
  for some \(\epsilon_1, \cdots, \epsilon_n > 0\) and \(f_1, \cdots, f_n \in X^*\).
\end{definition}

In other words, the weak topology on \(X\) is the coarsest topology on which 
every linear functional in \(X^*\) remains continuous. 

As one might expect, a sequence converge in the weak topology if and only if 
it converges weakly.

\begin{proposition}
  A convex set \(C \subseteq X\) is strongly closed (i.e. closed with respect 
  to the topology induced by the Norm) if and only if it is weakly closed 
  (i.e. closed with respect to the weak topology).
\end{proposition}

\begin{definition}[Weak* Topology]
  Let \(X\) be a Banach space. The weak* topology is a topology on \(X^*\) 
  with the base with elements of the form 
  \[U_{\epsilon_i}^{x_i} := \{g \in X^* \mid |g(x_i) - f(x_i)| < \epsilon_i\}\]
  for some \(\epsilon_1, \cdots, \epsilon_n > 0\), \(x_1, \cdots, x_n \in X\).
\end{definition}

One may show that the weak* topology is Hausdorff as pairs of distinct 
functionals can be separated by elements of \(X\). Also, a sequence of 
linear functionals which converge in the weak* topology is necessary bounded 
by the Banach-Steinhaus theorem.

The weak topology is in general not metrizable. On the other hand, one 
may show that if \(X\) is separable, the the unit ball in \(X^*\) endowed 
with the topology induced from the weak* topology is metrizable.

Lastly, we recall that the unit ball is never compact in a infinite dimensional 
normed space, the following theorem demonstrate that the weak* topology allows 
us to recover some sense of compactness in normed spaces.

\begin{theorem}[Banach-Alaoglu]
  The unit ball of a dual space is compact in the weak* topology.
\end{theorem}

\subsection{Compact Sets in Functional Spaces}

\begin{definition}[\(\epsilon\)-cover]
  An \(\epsilon\)-cover of a metric space for some \(\epsilon > 0\) is a cover 
  of the space consisting of open sets with diameter at most \(\epsilon\). 
  A metric space is totally bounded if it admits a finite \(\epsilon\)-cover 
  for every \(\epsilon > 0\).
\end{definition}

\begin{lemma}\label{tb}
  Let \(X\) be a metric space, such that for every \(\epsilon > 0\), there 
  exists some \(\delta > 0\) and a metric space \(W\), a continuous map 
  \(\Phi : X \to W\) with \(\Phi(X)\) is totally bounded such that for all 
  \(x, y \in X\), \(d(\Phi(x), \Phi(y)) < \delta\) implies 
  \(d(x, y) < \epsilon\). Then \(X\) is totally bounded. 
\end{lemma}
\begin{proof}
  Follows immediately by pulling back the \(\delta\)-covers of \(\Phi(X)\) along 
  \(\Phi\). 
\end{proof}

\begin{proposition}
  A metric space is compact if and only if it is compact and totally bounded.
\end{proposition}
\begin{proof}
  Exercise.
\end{proof}

\begin{proposition}
  The subset \(\mathcal{G} \subseteq \ell_p\) where \(1 \le p < \infty\) is 
  totally bounded if and only if it is point-wise bounded and for every 
  \(\epsilon > 0\) there exists some \(n\) such that for all \(x \in \mathcal{G}\), 
  \[\sum_{j > n} |x_j|^p < \epsilon^p.\]
\end{proposition}
\begin{proof}
  Suppose \(\mathcal{G}\) is totally bounded. Then, it is point-wise 
  bounded since totally boundedness implies boundedness which 
  in turn implies point-wise boundedness. On the other hand, for all 
  \(\epsilon > 0\), there exists a finite \(\epsilon / 2\)-cover 
  \((B_{\epsilon / 2}(x^i))_{i = 1}^n\) of \(\mathcal{G}\). Then, for all 
  \((y_n^i) \in B_{\epsilon / 2}(x^i)\), by definition 
  \[\frac{\epsilon}{2} > \|(y_n^i) - x^i\|_p = 
    \left(\sum_{n = 1}^\infty |y_n^i - x_n^i|^p\right)^{1 / p}.\]
  Then, taking \(N_i \in \mathbb{N}\) such that \(\sum_{n > N_i} |x_n^i|^p < 
  \frac{\epsilon^p}{2^p}\), we can define \(\tilde x^i, \tilde y^i \in \ell_p\),
  \[\tilde x_n^i := 
  \begin{cases}
    0, & \ n \le N_i,\\
    x_n^i, & \ n > N_i.
  \end{cases} \
    \tilde y_n^i := 
  \begin{cases}
    0, & \ n \le N_i,\\
    y_n^i, & \ n > N_i.
  \end{cases}\]
  So, by Hölder's inequality, 
  \[\begin{split}
    \left(\sum_{n > N_i} |y_n^i|^p\right)^{1 / p} & =
    \|\tilde y^i\|_p \le \|\tilde y^i - x^i\|_p + \|\tilde x^i\| < 
    \left(\sum_{n > N_i} |y_n^i - x_n^i|^p\right)^{1 / p} + 
    \left(\sum_{n > N_i} |x_n^i|^p\right)^{1 / p}\\
    & < \left(\sum_{n = 1}^\infty |y_n^i - x_n^i|^p\right)^{1 / p} + 
    \frac{\epsilon}{2} < \epsilon.
  \end{split}\]
  Thus, \(\sum_{n > N_i} |y_n^i|^p < \epsilon^p\) and hence, taking 
  \(N := \max_i N_i\) suffices.

  Conversely, for all \(\epsilon > 0\), let \(n \in \mathbb{N}\) 
  such that for all \(x \in \mathcal{G}\), 
  \(\sum_{j > n} |x_j|^p < (\epsilon / 4)^p\). Define 
  \[\Phi : \mathcal{G} \to \mathbb{R}^n : x \mapsto (x_1, \cdots, x_n).\]
  As \(\mathcal{G}\) is point-wise bounded, \(\Phi(\mathcal{G})\) is totally 
  bounded. Furthermore, if \(x, y \in \mathcal{G}\) such that
  \[\|\Phi(x) - \Phi(y)\|_p = 
    \left(\sum_{i = 1}^n |x_i - y_i|^p\right)^{1 / p} < \frac{\epsilon}{2},\]
  then 
  \[\|x - y\|_p \le \|\Phi(x) - \Phi(y)\|_p + 
  \left(\sum_{i > n} |x_i - y_i|^p\right)^{1 / p} < 
    \frac{\epsilon}{2} + \frac{\epsilon}{2} = \epsilon\]
  where the first inequality is due to a similar construction as the forward 
  direction along with Hölder's inequality. Hence, by lemma \ref{tb}, 
  \(\mathcal{G}\) is totally bounded.
\end{proof}

\begin{proposition}
  Let \(\mathcal{H}\) be a separable Hilbert space and suppose 
  \(D \subseteq \mathcal{H}\) is bounded and for some orthonormal basis 
  \(\{e_n\}_{n \in \mathbb{N}}\) of \(\mathcal{H}\), we have, for all 
  \(\epsilon > 0\), there exists some \(N \in \mathbb{N}\) such that 
  for all \(f \in D\),
  \[\sum_{n \ge N} |\langle f, e_n \rangle|^2 < \epsilon^2.\]
  Then \(D\) is precompact (i.e. \(\overline{D}\) is compact) in \(\mathcal{H}\).
\end{proposition}
\begin{proof}
  See official notes for full proof. 
  
  Outline: given a sequence in \(D\), it 
  suffices to show it has a convergent subsequence in \(\mathcal{H}\). This 
  follows by projecting the sequence on to \(\{e_1, \cdots, e_n\}\) for 
  appropriatly choosing \(n\). Then, with Heine-Borel, we obtain a 
  convergent subsequence in the projection. Pulling back, we obtain a 
  convergent subsequence in \(\mathcal{H}\).
\end{proof}

We recall the Arzela-Ascoli theorem in a more general context.

\begin{theorem}[Arzela-Ascoli]
  Let \(\Omega\) be a compact topological space. Then a subset 
  \(\mathcal{F} \subseteq \mathcal{C}(\Omega)\) is totally bounded in the 
  supremum norm if and only if it is point-wise bounded and equicontinuous 
  (i.e. for all \(x \in \Omega, \epsilon > 0\), there exists a open neighbourhood
  \(V\) of \(x\) such that, for all \(y \in V\), \(|f(x) - f(y)| < \epsilon\) 
  for all \(f \in \mathcal{F}\)).
\end{theorem}
\begin{proof}
  Similar to the \(\mathbb{R}^n\) case.
\end{proof}

\begin{theorem}[Kolmogorov-Riesz]
  A subset \(\mathcal{F}\) of \(L_p(\mathbb{R}^2)\) where \(1 \le p < \infty\) 
  is totally bounded if and only if 
  \begin{itemize}
    \item \(\mathcal{F}\) is bounded;
    \item for all \(\epsilon > 0\), there exists some \(R > 0\) such that for 
      all \(f \in \mathcal{F}\), 
      \[\int_{|x| > R} |f(x)|^p \lambda(\dd x) < \epsilon^p;\]
    \item for all \(\epsilon > 0\), there exists some \(\rho > 0\) such that 
      for all \(f \in \mathcal{F}, y \in \mathbb{R}^n, \|y\| < \rho\), 
      \[\int_{\mathbb{R}^n} |f(x + y) - f(x)|^p \lambda(\dd x) < \epsilon^p.\]
  \end{itemize}
\end{theorem}

\subsection{Compact Operators}

\begin{definition}
  Let \(X, Y\) be normed spaces. An operator \(T : X \to Y\) is a compact operator 
  if for all \(B \subseteq X\) bounded, \(\overline{T(B)}\) is compact.
\end{definition}

\begin{proposition}
  Let \(X, Y\) be normed spaces and \(T : X \to Y\) be an operator. Then 
  \(T\) is compact if and only if for all bounded sequence \((x_n)\) of 
  \(X\), \((Tx_n)\) has a convergent subsequence.
\end{proposition}
\begin{proof}
  Follows by considering the criterion for relatively compactness in metric 
  spaces. In particular, given a subset \(B\) of a metric space, 
  \(\overline{B}\) is compact if and only if all sequences in \(B\) contain a 
  subsequence which converges in \(\overline{B}\). Indeed, if \((x_n)\) is a 
  sequence in \(\overline{B}\), then we may construct a sequence \((y_n) 
  \subseteq B\) such that \(d(x_n, y_n) < 1 / n\). Then, if 
  \((y_{n_i}) \subseteq (y_n)\) such that \(y_{n_i} \to y \in \overline{B}\), 
  \[d(x_{n_i}, y) \le d(x_{n_i}, y_{n_i}) + d(y_{n_i}, y) < 
    \frac{1}{n_i} + d(y_{n_i}, y) \to 0\]
  as \(i \to \infty\). Thus, \(\overline{B}\) is compact.
\end{proof}

\begin{proposition}
  Let \(X, Y\) be normed spaces. Then the set of compact operators from \(X\) 
  to \(Y\) form an linear space.
\end{proposition}
\begin{proof}
  Exercise.
\end{proof}

\begin{proposition}
  A linear operator \(T : X \to Y\) where \(X, Y\) are normed spaces is compact 
  if \(T\) is bounded and \(\dim T(X) < \infty\) (such operators are known 
  as finite rank operators). Thus, if \(X\) is 
  finite-dimensional, every bounded linear operator is compact.
\end{proposition}
\begin{proof}
  Recall that for finite dimensional spaces, closed and bounded implies compactness. 
  Thus, for all bounded \(B\), as \(T(B)\) is bounded, \(\overline{T(B)}\) is 
  closed and bounded and hence, \(T\) is compact.
\end{proof}

\begin{proposition}
  Let \((T_n)\) be a sequence of compact linear operators from a normed space 
  \(X\) to a Banach space \(Y\). Then if \(T_n \to T\) with respect to the 
  operator norm, \(T\) is also compact.
\end{proposition}
\begin{proof}
  Let \((x_n)\) be a bounded sequence in \(X\) by some \(c \in \mathbb{R}^+\). 
  By assumption, as \(T_1\) is compact, there exists a subsequence 
  \((x_n^{(1)})\) of \((x_n)\) such that \(T_1 x_n^{(1)}\) converges. Then, 
  as \(T_2\) is compact, there is a subsequence \((x_n^{(2)})\) of \((x_n^{(1)})\)
  such that \((T_2x_n^{(2)})\) converges as well. Repeating this process, we 
  can find \((x_n^{(k)})\) a subsequence of \((x_n^{(l)})\) for all \(l \le k\) 
  such that \(T_i x_n^{(k)}\) converges for \(i = 1, 2, \cdots, k\). Now, defining 
  \(y_n := x_n^n\), I claim \((y_n)\) is a subsequence of \((x_n)\) such that 
  \(T_i y_n\) converges for all \(i \in \mathbb{N}\). Indeed, for all 
  \(i \in \mathbb{N}\), once \(n \ge i\), the remaining sequence of \((y_n)\) 
  is contained \((x_n^{(i)})\). Thus, \(T_i y_n\) is eventually contained in 
  \(T_i x_n^{(i)}\) which converges.

  Now, as \(T_n \to T\), for all \(\epsilon > 0\), there exists \(N\) such that 
  for all \(n \ge N\), \(\|T_n - T\| < \epsilon / (3c)\). On the other hand, 
  as \((T_N y_n)\) is convergent, it is Cauchy, and so, there exists 
  sufficiently large \(M\), such that \(\|T_N y_i - T_N y_j\| < \epsilon / 3\) 
  for all \(i, j \ge M\). Hence, for all \(i, j \ge M\),
  \[\|Ty_i - Ty_j\| \le \|Ty_i - T_N y_i\| + \|T_N y_i - T_N y_j\| + 
    \|T_N y_j - Ty_j\| < \epsilon.\]
  Therefore \((Ty_n)\) is Cauchy and so, \(T\) is compact.
\end{proof}

\begin{theorem}
  Let \(X, Y\) be normed spaces and suppose \(T : X \to Y\) is a compact 
  operator. Then, if \((x_n) \subseteq X\) is a sequence which converges weakly 
  to \(x \in X\), \(T x_n\) converges strongly to \(Tx\).
\end{theorem}
\begin{proof}
  We see that \(Tx_n \weak Tx\) since for all \(g \in Y^*\), \(g \circ T\) is 
  a linear functional of \(X^*\), and so, \(g(Tx_n) \to g(Tx)\) implying 
  \(Tx_n \weak Tx\). Suppose \(Tx_n\) does not converge to \(Tx\) strongly, 
  then, there exists some \(\epsilon > 0\) and a subsequence \((x_{n_i})\) of 
  \((x_n)\) such that for all \(i\), \(\|Tx_{n_i} - Tx\| \ge \epsilon\). 
  Now, since \((x_{n_i})\) is bounded, and \(T\) is compact, there exists a 
  subsequence \((x_{n_{i_j}})\) of \((x_{n_i})\) such that \(Tx_{n_{i_j}} \to 
  y\), for some \(y \in Y\) and so, \(Tx_{n_{i_j}} \weak y\). But 
  \(Tx_{n_{i_j}} \weak Tx\), so by the uniqueness of the weak limit, 
  \(y = Tx\). But, this contradicts \(\|Tx_{n_{i_j}} - Tx\| \ge \epsilon\) 
  for all \(j\). Thus, \(T x_n \to Tx\) strongly.
\end{proof}

\begin{corollary}
  If \(T\) is a compact operator on \(X\) and \((x_n)\) is a sequence of 
  \(X\) which converges weakly to 0. Then \(T x_n\) converges strongly to 0.
\end{corollary}

\begin{proposition}
  Let \(X, Y\) be Banach spaces and let \(T : X \to Y\) be a bounded linear 
  operator. Then \(T\) is compact if and only if \(T^* : Y \to X\) is compact.
\end{proposition}

\begin{proposition}
  Let \(\mathcal{H}\) be a separable Hilbert space. Then every compact operator
  \(T\) on \(\mathcal{H}\) is a strong limit of operators of finite rank.
\end{proposition}
\begin{proof}
  Let \((e_j)\) be an orthonormal basis of \(\mathcal{H}\) and let us define 
  \[\lambda_n := \sup_{f \perp \{e_j \mid j = 1, \cdots n\}, \|f\| = 1} \|Tf\|.\]
  It is clear that \((\lambda_n)\) is bounded and decreasing, and so, it converges 
  to some \(\lambda \ge 0\). By definition, there exists a sequence \((f_n)\), 
  \(f_n \perp \{e_j \mid j = 1, \cdots n\}\) and \(\|f_n\| = 1\) such that 
  \(\|Tf_n\| \ge \lambda / 2\) for all \(n\). Then, as \(f_n \weak 0\) since 
  \(T\) is compact, we have \(Tf_n \weak 0\) and thus, \(\lambda = 0\).

  Thus, defining 
  \[T^{(n)} := \sum_{j = 1}^n \langle e_j, \cdot \rangle T(e_j),\]
  we have \(\lambda_n = \|T - T^{(n)}\|\) and so, \(\|T - T^{(n)}\| \to 0\) 
  as \(n \to \infty\) where \(T^{(n)}\) are finite rank opertors.
\end{proof}

\begin{theorem}[Fredholm Alternative]
  If \(A\) is a compact operator on a Hilbert space, then either \(A\Psi = \Psi\) 
  has a solution of \(\text{id} - A\) is invertible.
\end{theorem}

\begin{proposition}
  Let \(D \subseteq \mathbb{C}\) be open and connected. Let \(\Psi : D \to 
  \mathcal{L}(\mathcal{H})\) be an operator valued function such that 
  \(\Psi(z)\) is compact for any \(z \in D\). Then, either 
  \(\text{id} - \Psi(z)\) is not invertible for any \(z \in D\) or 
  \(\text{id} - \Psi(z)\) is invertible for all \(z \in D \setminus S\) where 
  \(S\) is any set with no limit points in \(D\).
\end{proposition}

\subsubsection{Dirichlet Problem}

\begin{definition}[Dirichlet Problem]
  \(u\) is called a solution of the Dirichlet problem in an open domain if 
  \(u \in \mathcal{C}^2(D) \cap \mathcal{C}(\overline{D})\) and 
  \[\begin{cases}
    (\Delta u)\mid_x = 0, \ & x \in D,\\
    u(x) = f(x), \ & x \in \partial D,
  \end{cases}\]
  where \(\Delta\) is the Laplacian operator and \(f\) is some (say smooth) 
  function.
\end{definition}

Define the space 
\[\tilde{\mathcal{C}}_0^1(\overline{D}) := \{\text{continuously differentiable 
function on \(\overline{D}\) vanishing on \(\partial D\)}\},\]
and equip it with the inner product 
\[\langle u, v \rangle := \int_D \nabla u \nabla v \dd x = 
  \int_D \sum_k (\partial_{x_k} u)(\partial_{x_k} v) \dd x.\]
Suppose that \(u \in \mathcal{C}^2(\overline{D})\) is a solution 
to the Dirichlet problem (DP) where \(f \in C^0(\overline{D})\), then by 
Stokes' theorem, 
\[\langle u, v \rangle = \int_D \nabla u \nabla v \dd x = 
  \int_{\partial D} v \Delta u \dd x = \int_{\partial D} v f \dd x,\]
for all \(v \in \tilde{\mathcal{C}}_0^1(\overline{D})\). 
Thus, the above is a necessary condition of \(u\) being a solution of 
DP. With this, we define the weak solution.

\begin{definition}
  Let \(u \in H^1_0(D)\) (where \(H^1_0(D)\) is the completion of 
  \(\tilde{\mathcal{C}}_0^1(\overline{D})\)). We say \(u\) is a weak 
  solution of DP if 
  \[\langle u, v\rangle = \int vf \dd x\]
  for all \(v \in H^1_0(D)\).
\end{definition}

Given the above formulation, we see that the map \(v \mapsto \int vf \dd x\) 
is a linear functional from \(H_0^1(D)\) and thus, by the Riesz representation 
theorem, there exists a unique \(u \in H_0^1(D)\) such that 
\(\langle u, v \rangle = \int v f \dd x\) for all \(v \in H_0^1(D\). This 
demonstrates the existence of the weak solution in \(H_0^1(D)\) and provides 
a good candidate for the solution of DP.

% \subsubsection{Hilbert-Schmidt Integral Operator}

% \begin{definition}[Hilbert-Schmidt Integral Operator]
%   The Hilbert-Schmidt integral operator \(T : H \to H\) where 
%   \(H := L^2(\mathbb{R}^d)\) is the linear operator 
%   \[Tf : x \mapsto \int_{\mathbb{R}^d} K(x, y) f(y) \dd \lambda(y),\]
%   for some kernel map \(K \in L^2(\mathbb{R}^d \times \mathbb{R}^d)\).
% \end{definition}

% The map \(y \mapsto K(x, y)f(y)\) is integrable for all \(x\) since by 
% Fubini, the map \(y \mapsto K(x, y)\) is square integrable for all \(x\) 
% and thus, 
% \[\|K(x, \cdot)f(\cdot)\|_1 = \langle K(x, \cdot), f \rangle^2 \le 
%   \|K(x, \cdot)\|_2 \|f\|_2.\]
% We note that this also implies the operator is bounded with the bound 
% \(\|T\| \le \|K\|_2\).

\subsection{Spectrum of Operators}

\begin{definition}[Resolvent]
  Let \(X\) be a complex normed space and let \(T : D(T) \subseteq X \to X\) 
  be a bounded operator. Then, for all \(\lambda \in \mathbb{C}\), denote 
  \(T_\lambda := T - \lambda \cdot \text{id}_X\). If \(T_\lambda\) has 
  a inverse then we denote \(R_\lambda(T) := T_\lambda^{-1}\) and we call 
  it the resolvent of \(T\).
\end{definition}

We see straight a way the existence of a resolvent for some value \(\lambda\) 
implies \(\lambda\) is not an eigenvalue of the operator.

\begin{definition}[Regular Value]
  \(\lambda \in \mathbb{C}\) is a regular value of the operator \(T\) if 
  the \(\lambda\)-resolvent of \(T\) exists and is bounded and it is defined 
  on a dense subspace of \(X\).
\end{definition}

\begin{definition}[Resolvent Set and Spectrum]
  Given a bounded operator \(T : D(T) \subseteq X \to X\), the \(T\)-resolvent 
  set is 
  \[\rho(T) := \{\lambda \in \mathbb{C} \mid \lambda \text{ is a regular value}\}.\]
  The complement of the resolvent set 
  \[\sigma(T) := \mathbb{C} \setminus \rho(T)\]
  is called the spectrum of \(T\). 
\end{definition}

By observing the definition of the regular value, the spectrum can be decomposed 
into the following three categories.

\begin{definition}\ 
  \begin{itemize}
    \item The discrete spectrum is the set \(\sigma_p(T)\) containing 
      \(\lambda \in \mathbb{C}\) for which \(R_\lambda(T)\) does not exist.
      We call \(\lambda \in \sigma_p(T)\) the eigenvalues of \(T\).
    \item The continuous spectrum is the set \(\sigma_c(T)\) containing 
      \(\lambda \in \mathbb{C}\) for which \(R_\lambda(T)\) exists and is 
      defined on a dense subset of \(X\) but is not bounded.
    \item The residual spectrum is the set \(\sigma_r(T)\) containing 
      \(\lambda \in \mathbb{C}\) for which \(R_\lambda(T)\) exists but is not 
      defined on a dense subset of \(X\).
  \end{itemize}
\end{definition}

\begin{proposition}
  \(\rho(T)\) is open and as a result \(\sigma(T)\) is closed.
\end{proposition}
\begin{proof}
  Let \(\lambda \in \rho(T)\) and suppose \(|\lambda_0 - \lambda| < 
  \|R_{\lambda_0}(T)\|^{-1}\). Then, we see 
  \[T - \lambda \cdot \text{id}_X = (T - \lambda_0 \cdot \text{id}_X)
    (\text{id}_X + (\lambda_0 - \lambda)R_{\lambda_0}(T)),\]
  and since, 
  \[(\text{id}_X + (\lambda_0 - \lambda)R_{\lambda_0}(T))^{-1} = 
    \sum_{n = 0}^\infty (\lambda - \lambda_0)^nR_{\lambda_0}^n(T)\]
  where the right hand side is convergent in operator norm. Thus, 
  \[(T - \lambda \cdot \text{id}_X)^{-1} = 
  (\text{id}_X + (\lambda_0 - \lambda)R_{\lambda_0}(T))^{-1} R_{\lambda_0}(T) 
  = \sum_{n = 0}^\infty (\lambda - \lambda_0)^nR_{\lambda_0}^{n + 1}(T)\] 
  implying \(\lambda \in \rho(T)\) as required.
\end{proof}

We recall that in the finite dimensional case over \(\mathbb{C}\), as 
\(\mathbb{C}\) is algebraically closed, the characteristic polynomial 
always has a root and thus, every operator has at least 1 eigenvalie. 
The similar is true for Hilbert spaces.

\begin{lemma}[Neumann Series]
  If \(A\) is a bounded linear operator such that \(\|A\| < 1\) and 
  \(\text{id} - A\) is invertible, then 
  \[(\text{id} - A)^{-1} = \sum_{k = 0}^\infty A^k.\]
\end{lemma}
\begin{proof}
  Exercise.
\end{proof}

\begin{corollary}
  \(R_\lambda(T) = \sum_{k = 0}^\infty \lambda^{- k - 1} T^k.\)
\end{corollary}

\begin{lemma}
  The resolvent \(R_\lambda(T)\) is continuous and analytic with 
  respect to \(\lambda\).
\end{lemma}
\begin{proof}
  Given \(\lambda, \mu \in \mathbb{C}\), as an exercise, one may show 
  \(R_\lambda(T) - R_\mu(T) = (\lambda - \mu)R_\lambda(T) R_\mu(T)\).
  Then 
  \[\|R_\lambda(T) - R_\mu(T)\| = |\lambda - \mu| \|R_\lambda(T)\|\|R_\mu(T)\| \to 0\]
  as \(\lambda \to \mu\) and so \(R_\lambda(T)\) is continuous with respect 
  to \(\lambda\) provided we can find a bound of \(\|R_\lambda(T)\) for \(\lambda\) 
  in a neighbourhood of \(\mu\). By looking at the previous proof, we see that 
  for \(|\lambda - \mu|\) sufficiently small,
  \[(T - \lambda \cdot \text{id})^{-1} = 
  \sum_{n = 0}^\infty (\lambda - \mu)^nR_{\mu}^{n + 1}(T)\] 
  and so \(\|R_\lambda(T)\| \to \|R_\mu(T)\|\) as \(\lambda \to \mu\).

  \(R_\lambda(T)\) is analytic since it is holomorphic with respect to 
  \(\lambda\) by the same argument through dividing both sides with 
  \(\lambda - \mu\).
\end{proof}

\begin{theorem}
  The spectrum of a bounded operator in a Hilbert space is non-empty.
\end{theorem}
\begin{proof}
  Suppose \(A\) is a bounded operator in the Hilbert space \(\mathcal{H}\). 
  Then \(R_\lambda(A)\) is an analytic and so, the so is the function 
  \[f : \mathbb{C} \to \mathbb{C} : \lambda \mapsto \langle y, R_\lambda(A)x\rangle,\]
  analytic on \(\rho(A)\) for all \(x, y \in \mathcal{H}\). Furthermore, by 
  the Neumann series representation of the resolvent, we see that 
  \(\lim_{\lambda \to \infty} f(\lambda) = 0\). Then, if 
  \(\sigma(A) = \varnothing\), 
  we have \(\rho(A) = \mathbb{C}\) and hence, we have found a bounded analytic 
  function \(f\) on \(\mathbb{C}\) and thus, by Liouville's theorem, 
  \(f = 0\). But as \(x, y\) was arbitrary, this implies \(R_\lambda(A) = 0\), 
  contraction! Hence \(\sigma(A)\) is non-empty.
\end{proof}


\end{document}

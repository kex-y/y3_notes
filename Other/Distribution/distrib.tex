% Options for packages loaded elsewhere
\PassOptionsToPackage{unicode}{hyperref}
\PassOptionsToPackage{hyphens}{url}
\PassOptionsToPackage{dvipsnames,svgnames*,x11names*}{xcolor}
%
\documentclass[]{article}
\usepackage{lmodern}
\usepackage{amssymb,amsmath}
\usepackage{ifxetex,ifluatex}
\ifnum 0\ifxetex 1\fi\ifluatex 1\fi=0 % if pdftex
  \usepackage[T1]{fontenc}
  \usepackage[utf8]{inputenc}
  \usepackage{textcomp} % provide euro and other symbols
\else % if luatex or xetex
  \usepackage{fontspec}
  \setmainfont{Bitstream Charter}
  % Cabin -- round
  % Bitstream Charter -- almost computer modern
  % \usepackage{unicode-math}
  % \defaultfontfeatures{Scale=MatchLowercase}
  % \defaultfontfeatures[\rmfamily]{Ligatures=TeX,Scale=1}
\fi
% Use upquote if available, for straight quotes in verbatim environments
\IfFileExists{upquote.sty}{\usepackage{upquote}}{}
\IfFileExists{microtype.sty}{% use microtype if available
  \usepackage[]{microtype}
  \UseMicrotypeSet[protrusion]{basicmath} % disable protrusion for tt fonts
}{}
\makeatletter
\@ifundefined{KOMAClassName}{% if non-KOMA class
  \IfFileExists{parskip.sty}{%
    \usepackage{parskip}
  }{% else
    \setlength{\parindent}{0pt}
    \setlength{\parskip}{6pt plus 2pt minus 1pt}}
}{% if KOMA class
  \KOMAoptions{parskip=half}}
\makeatother
\usepackage{xcolor}
\IfFileExists{xurl.sty}{\usepackage{xurl}}{} % add URL line breaks if available
\IfFileExists{bookmark.sty}{\usepackage{bookmark}}{\usepackage{hyperref}}
\hypersetup{
  pdftitle={Conditional Expectation},
  pdfauthor={Kexing Ying},
  colorlinks=true,
  linkcolor=Maroon,
  filecolor=Maroon,
  citecolor=Blue,
  urlcolor=red,
  pdfcreator={LaTeX via pandoc}}
\urlstyle{same} % disable monospaced font for URLs
\usepackage[margin = 1.5in]{geometry}
\usepackage{graphicx}
\makeatletter
\def\maxwidth{\ifdim\Gin@nat@width>\linewidth\linewidth\else\Gin@nat@width\fi}
\def\maxheight{\ifdim\Gin@nat@height>\textheight\textheight\else\Gin@nat@height\fi}
\makeatother
% Scale images if necessary, so that they will not overflow the page
% margins by default, and it is still possible to overwrite the defaults
% using explicit options in \includegraphics[width, height, ...]{}
\setkeys{Gin}{width=\maxwidth,height=\maxheight,keepaspectratio}
% Set default figure placement to htbp
\makeatletter
\def\fps@figure{htbp}
\makeatother
\setlength{\emergencystretch}{3em} % prevent overfull lines
\providecommand{\tightlist}{%
  \setlength{\itemsep}{0pt}\setlength{\parskip}{0pt}}
\setcounter{secnumdepth}{5}
\usepackage{tikz}
\usepackage{physics}
\usepackage{amsthm}
\usepackage{mathtools}
\usepackage{esint}
\usepackage[ruled,vlined]{algorithm2e}
\theoremstyle{definition}
\newtheorem{theorem}{Theorem}
\newtheorem{definition*}{Definition}
\newtheorem{prop}{Proposition}
\newtheorem{corollary}{Corollary}[theorem]
\newtheorem*{remark}{Remark}
\theoremstyle{definition}
\newtheorem{definition}{Definition}
\newtheorem{lemma}{Lemma}[section]
\newtheorem{proposition}{Proposition}[section]
\newtheorem{example}{Example}[section]
\newcommand{\diag}{\mathop{\mathrm{diag}}}
\newcommand{\Arg}{\mathop{\mathrm{Arg}}}
\newcommand{\hess}{\mathop{\mathrm{Hess}}}
\newcommand\eqae{\mathrel{\overset{\makebox[0pt]{\mbox{\normalfont\tiny\sffamily a.e.}}}{=}}}

\title{Distribution vs. Signed Measures}
\author{Kexing Ying}
\date{July 24, 2021}

\begin{document}
\maketitle

Distributions are interpreted as generalized functions with the motivating example 
of the \(\delta\)-function. The \(\delta\)-function, arising naturally from physics 
is a function with the property that it is 0 everywhere but at a single point 
and has integral 1. Namely a function
\(\delta_x : \mathbb{R} \to \mathbb{R}\) which satisfies 
\begin{itemize}
  \item \(\delta_x(y) = 0\) for all \(y \neq x\) and, 
  \item \(\int \delta_x = 1\). 
\end{itemize}
It is clear such a function does not exist as the first property implies 
\(\delta_x = 0\) almost everywhere which implies it has integral 0. Distributions 
are suppose to solve this problem by generalizing functions to the space of 
distributions.

\begin{definition}[Test Function]
  The space of test function \(\mathcal{D}\) is the space of infinitely 
  differentiable functions \(f : \mathbb{R} \to \mathbb{R}\) with compact support.
\end{definition}

\begin{definition}[Distribution]
  A distribution on \(\mathbb{R}\) is a continuous linear function of the space 
  of test functions \(\mathcal{D}\). Namely, the space of distributions is the 
  dual of \(\mathcal{D}\) and we denote it by \(\mathcal{D}'\). 
\end{definition}

Functions can be naturally embedded into the space of distributions via the map 
\(f \mapsto T_f\) where 
\[T_f : \mathcal{D} \to \mathbb{R} : \phi \mapsto \int f \phi \dd \lambda,\]
where \(\lambda\) is the Lebesgue measure. For this embedding to make sense, 
we require \(f\) to be locally integrable. This is a rather relaxed requirement 
as any measurable function which is bounded on a compact set is locally integrable. 

The \(delta\) function can be represented naturally as a distribution. 
In particular, we take \(\delta_x \in \mathcal{D}'\) such that 
\(\delta'(\phi) := \phi(x)\). This represents the \(\delta\)-function we had 
in mind by considering that \textit{if} such a function \(f\) which satisfy 
the aforementioned two properties, 
\[T_f(\phi) = \int f \phi \dd \lambda = 
  \int \mathbf{1}_{\{x\}}f \phi \dd \lambda = \phi(x)\int f \dd \lambda = \phi(x).\]
However, we now note a different method of representing the \(\delta\)-function. 
Namely, the Dirac measures. The Dirac measure at \(x\) is a probability measure 
defined as \(\delta_x(A) = 1\) if \(x \in A\) and 0 otherwise. This is similar 
to the distribution version as for all \(\phi\) measurable, 
\[\int \phi \dd \delta_x = \phi(x)\]
and so, by considering the map \(\mu \mapsto T_\mu \in \mathcal{D}'\) where 
\[T_\mu : \mathcal{D} \mapsto \mathbb{R} : \phi \mapsto \int \phi \dd \mu,\]
we have a natural embedding of measures into the space of distributions.

However, unlike the space of distributions, only non-negative functions 
\(f : \mathbb{R} \to \mathbb{R}_+\) induces a measure with \(f \lambda\). Thus, 
in order to embed general functions, we need to consider signed measures. 
This process is easy by mapping 
\[f \mapsto A \mapsto \int_A f \dd \lambda = \int_A f^+ \dd \lambda - 
  \int_A f^- \dd \lambda,\]
i.e. \(f \mapsto f^+ \lambda - f^- \lambda\). With this, we have the following 
embeddings 
\[(\mathbb{R} \to \mathbb{R}) \hookrightarrow \mathbb{R}-\text{signed measures}
  \hookrightarrow \mathcal{D}'.\]
One can show that every distribution which is non-negative on non-negative functions 
defines a measure which commutes with our embedding. However, not all distributions 
arises from a signed measure. Indeed, the derivative of the 
\(\delta\) distribution (\(f \mapsto -f'(0)\)) cannot be represented as a signed 
measure.

\end{document}

% Options for packages loaded elsewhere
\PassOptionsToPackage{unicode}{hyperref}
\PassOptionsToPackage{hyphens}{url}
\PassOptionsToPackage{dvipsnames,svgnames*,x11names*}{xcolor}
%
\documentclass[]{article}
\usepackage{lmodern}
\usepackage{amssymb,amsmath}
\usepackage{ifxetex,ifluatex}
\ifnum 0\ifxetex 1\fi\ifluatex 1\fi=0 % if pdftex
  \usepackage[T1]{fontenc}
  \usepackage[utf8]{inputenc}
  \usepackage{textcomp} % provide euro and other symbols
\else % if luatex or xetex
  \usepackage{fontspec}
  \setmainfont{Bitstream Charter}
  % Cabin -- round
  % Bitstream Charter -- almost computer modern
  % \usepackage{unicode-math}
  % \defaultfontfeatures{Scale=MatchLowercase}
  % \defaultfontfeatures[\rmfamily]{Ligatures=TeX,Scale=1}
\fi
% Use upquote if available, for straight quotes in verbatim environments
\IfFileExists{upquote.sty}{\usepackage{upquote}}{}
\IfFileExists{microtype.sty}{% use microtype if available
  \usepackage[]{microtype}
  \UseMicrotypeSet[protrusion]{basicmath} % disable protrusion for tt fonts
}{}
\makeatletter
\@ifundefined{KOMAClassName}{% if non-KOMA class
  \IfFileExists{parskip.sty}{%
    \usepackage{parskip}
  }{% else
    \setlength{\parindent}{0pt}
    \setlength{\parskip}{6pt plus 2pt minus 1pt}}
}{% if KOMA class
  \KOMAoptions{parskip=half}}
\makeatother
\usepackage{xcolor}
\IfFileExists{xurl.sty}{\usepackage{xurl}}{} % add URL line breaks if available
\IfFileExists{bookmark.sty}{\usepackage{bookmark}}{\usepackage{hyperref}}
\hypersetup{
  pdftitle={Algebraic Topology Revision Notes},
  pdfauthor={Kexing Ying},
  colorlinks=true,
  linkcolor=Maroon,
  filecolor=Maroon,
  citecolor=Blue,
  urlcolor=red,
  pdfcreator={LaTeX via pandoc}}
\urlstyle{same} % disable monospaced font for URLs
\usepackage[margin = 1.5in]{geometry}
\usepackage{graphicx}
\makeatletter
\def\maxwidth{\ifdim\Gin@nat@width>\linewidth\linewidth\else\Gin@nat@width\fi}
\def\maxheight{\ifdim\Gin@nat@height>\textheight\textheight\else\Gin@nat@height\fi}
\makeatother
% Scale images if necessary, so that they will not overflow the page
% margins by default, and it is still possible to overwrite the defaults
% using explicit options in \includegraphics[width, height, ...]{}
\setkeys{Gin}{width=\maxwidth,height=\maxheight,keepaspectratio}
% Set default figure placement to htbp
\makeatletter
\def\fps@figure{htbp}
\makeatother
\setlength{\emergencystretch}{3em} % prevent overfull lines
\providecommand{\tightlist}{%
  \setlength{\itemsep}{0pt}\setlength{\parskip}{0pt}}
\setcounter{secnumdepth}{5}
\usepackage{tikz}
\usepackage{tikz-cd}
\usepackage{physics}
\usepackage{amsthm}
\usepackage{mathtools}
\usepackage{esint}
\usepackage[ruled,vlined]{algorithm2e}
\theoremstyle{definition}
\newtheorem*{theorem}{Theorem}
\newtheorem*{corollary}{Corollary}
\newtheorem*{remark}{Remark}
\newtheorem*{definition}{Definition}
\newtheorem*{lemma}{Lemma}
\newtheorem*{proposition}{Proposition}
\newtheorem*{example}{Example}
\newcommand{\diag}{\mathop{\mathrm{diag}}}
\newcommand{\Arg}{\mathop{\mathrm{Arg}}}
\newcommand{\hess}{\mathop{\mathrm{Hess}}}
\newcommand\eqae{\mathrel{\overset{\makebox[0pt]{\mbox{\normalfont\tiny\sffamily a.e.}}}{=}}}

\title{Algebraic Topology Revision Notes}
\author{Kexing Ying}

\begin{document}
\maketitle

\section*{Algebra}

Fundamental groups is a topological invariant. Abelianization is a group invariant.

The abelianization of a presentation \(\langle S \mid R \rangle\) is simply that 
\[\langle S \mid R \rangle = \langle S \mid R \cup [a, b], a, b \in S \rangle.\]

\section*{Notes on Shapes}

\begin{proposition}
  If \(A\) is a deformation retract of \(X\), then the inclusion map \(i : A \hookrightarrow X\) 
  induces a group isomorphism \(i_* : \pi_1(A) \tilde \to \pi_1(X)\).
\end{proposition}

\begin{proposition}
  If \(A\) is a deformation retract of \(X\), then for all \(n\), the inclusion map \(i : A \hookrightarrow X\) 
  induces a group isomorphisms \(i_* : H_n(A) \tilde \to H_n(X)\).
\end{proposition}

\begin{definition}
  Given \((X, x)\) a pointed space, there is the natural map 
  \[\phi : \pi_1(X, x) \to H_1(X)\]
  by interpreting each loop as a 1-chain. 
\end{definition}

\begin{proposition}
  As \(H_1(X)\) is abelian, by the universal property, there exists a unique \(\Phi\) 
  such that 
  \[\begin{tikzcd}
    {\pi_1(X,x)} && {H_1(X)} \\
    {\pi_1(X,x)_{ab}}
    \arrow["{\text{ab}}", from=1-1, to=2-1]
    \arrow["\phi"', from=1-1, to=1-3]
    \arrow["\Phi"', dashed, from=2-1, to=1-3]
  \end{tikzcd}\]
  commutes. Furthermore, if \(X\) is path-connected, \(\Phi\) is an isomorphism.
\end{proposition}

\begin{itemize}
  \item \(S^n\): simply connected for \(n \ge 1\) so \(\pi_1(S^n) = 0\). 
  \[H_k(S^n) = \begin{cases}
    \mathbb{Z}, \ & k = 0, n;\\
    0, \ & \text{otherwise}.
  \end{cases}\]
  \item \(T^2 = S^1 \times S^1\): \(\pi_1(T^2) = \mathbb{Z} \times \mathbb{Z}\) as 
    \(\pi_1(X \times Y, (x, y)) = \pi_1(X, x) \times \pi_1(Y, y)\).
    \[H_0(T^2) = H_2(T^2) = \mathbb{Z}, H_1(T^2) = \mathbb{Z}^2, 0 \text{ otherwise}.\]
  \item \(S^1 \times D^2\) (filled torus): deformation retracts to \(S^1\).
  \item \(S^1 \cup_f S^1\) where \(f\) identifies two points from each \(S^1\), (i.e. a venn diagram):
    collapsing two points from one side provides a homotopy equivalence. This results in 
    \(S^1 \vee S^1 \vee S^1\).
  \item \(S\{1, \cdots, n\}\) (suspension of \(n\) elements): possible to deform into \(n - 1\)-circles 
    in the shape of 8. Thus, \(\pi_1(S\{1, \cdots, n\}) = F_{n - 1}\). Generalizes the above point.
\end{itemize}

Let \(X \subseteq \mathbb{R}^n \subseteq S^n = \mathbb{R}^n \cup \{\infty\}\). Under some conditions 
\(S^n \setminus X^\circ \simeq X\). This is true for \(X\) a solid sphere or solid torus 
(probably true for more spaces).

\section*{General Facts}

Path-connected does not imply locally path conencted.

If \(f\) and \(g\) are homotopic, then their induced maps are equal at both the fundamental group 
level as well as the homology level.

A simply connected space only has homeomorphisms as its covering maps (\textbf{Use converse to show 
not simply connected}). Induced map of a covering map is injective by the homotopy lifting theorem 
and so, the space is simply connected, the covering space must also be simply connected, namely 
the covering is universal. Thus, by uniqueness, this must be a homeomorphism as the space covers 
itself universally.

Covering maps are open maps since they can be written as a quotient map.

\textbf{A method to find the fundamental group}. Let \(X\) be some space and suppose it has the universal 
cover \(\tilde X\). Then, if \(\Gamma\) is a group acting nicely on \(\tilde X\) and 
\(X \simeq \tilde X / \Gamma\), \(X\) has fundamental group \(\Gamma\). 

\begin{theorem}[Lifting Criterion]
  Let \(\pi : (Y, y) \to (X, x)\) be a covering map and let \(f : (Z, z) \to (X, x)\) 
  be continuous where \(Z\) path-connected and locally path-connected. Then, 
  there exists a \textbf{unique} \(\tilde f : (Z, z) \to (Y, y)\) such that 
  \[\begin{tikzcd}
    && {(Y, y)} \\
    {(Z, z)} && {(X,x)}
    \arrow["\pi", from=1-3, to=2-3]
    \arrow["f"', from=2-1, to=2-3]
    \arrow["{\tilde f}", dashed, from=2-1, to=1-3]
  \end{tikzcd}\]
  commutes \textbf{if and only if} 
  \[f_*\pi_1(Z, z) \subseteq p_* \pi_1(Y, y)\]
  as subgroups.
\end{theorem}

For \(X\) path-connected, \(H_0(X, A) = 0\) for all \(\varnothing \neq A \subseteq X\) by considering 
paths from \(A\) to any point in \(X\) as 1-chains.

\begin{theorem}[Long Exact Sequence for Quotients]
  Given \(A \subseteq X\) where \((X, A)\) is a good pair. We have the long exact sequence 
  \[\cdots \to H_n(A) \to H_n(X) \to H_n(X / A) \to H_{n - 1}(A) \to \cdots\]
  with the end 
  \[\cdots \to H_1(A) \to H_1(X) \to H_1(X / A) \to \tilde H_0(A) \to \tilde H_0(X) \to \tilde H_0(X / A),\]
  where \(\tilde H_0(X) \simeq 0 \simeq \tilde H_0(X / A)\) if \(X\) is path connected.
\end{theorem}

\end{document}

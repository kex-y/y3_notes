% Options for packages loaded elsewhere
\PassOptionsToPackage{unicode}{hyperref}
\PassOptionsToPackage{hyphens}{url}
\PassOptionsToPackage{dvipsnames,svgnames*,x11names*}{xcolor}
%
\documentclass[]{article}
\usepackage{lmodern}
\usepackage{amssymb,amsmath}
\usepackage{ifxetex,ifluatex}
\ifnum 0\ifxetex 1\fi\ifluatex 1\fi=0 % if pdftex
  \usepackage[T1]{fontenc}
  \usepackage[utf8]{inputenc}
  \usepackage{textcomp} % provide euro and other symbols
\else % if luatex or xetex
  \usepackage{fontspec}
  \setmainfont{Bitstream Charter}
  % Cabin -- round
  % Bitstream Charter -- almost computer modern
  % \usepackage{unicode-math}
  % \defaultfontfeatures{Scale=MatchLowercase}
  % \defaultfontfeatures[\rmfamily]{Ligatures=TeX,Scale=1}
\fi
% Use upquote if available, for straight quotes in verbatim environments
\IfFileExists{upquote.sty}{\usepackage{upquote}}{}
\IfFileExists{microtype.sty}{% use microtype if available
  \usepackage[]{microtype}
  \UseMicrotypeSet[protrusion]{basicmath} % disable protrusion for tt fonts
}{}
\makeatletter
\@ifundefined{KOMAClassName}{% if non-KOMA class
  \IfFileExists{parskip.sty}{%
    \usepackage{parskip}
  }{% else
    \setlength{\parindent}{0pt}
    \setlength{\parskip}{6pt plus 2pt minus 1pt}}
}{% if KOMA class
  \KOMAoptions{parskip=half}}
\makeatother
\usepackage{xcolor}
\IfFileExists{xurl.sty}{\usepackage{xurl}}{} % add URL line breaks if available
\IfFileExists{bookmark.sty}{\usepackage{bookmark}}{\usepackage{hyperref}}
\hypersetup{
  pdftitle={Group Representation Revision Notes},
  pdfauthor={Kexing Ying},
  colorlinks=true,
  linkcolor=Maroon,
  filecolor=Maroon,
  citecolor=Blue,
  urlcolor=red,
  pdfcreator={LaTeX via pandoc}}
\urlstyle{same} % disable monospaced font for URLs
\usepackage[margin = 1.5in]{geometry}
\usepackage{graphicx}
\makeatletter
\def\maxwidth{\ifdim\Gin@nat@width>\linewidth\linewidth\else\Gin@nat@width\fi}
\def\maxheight{\ifdim\Gin@nat@height>\textheight\textheight\else\Gin@nat@height\fi}
\makeatother
% Scale images if necessary, so that they will not overflow the page
% margins by default, and it is still possible to overwrite the defaults
% using explicit options in \includegraphics[width, height, ...]{}
\setkeys{Gin}{width=\maxwidth,height=\maxheight,keepaspectratio}
% Set default figure placement to htbp
\makeatletter
\def\fps@figure{htbp}
\makeatother
\setlength{\emergencystretch}{3em} % prevent overfull lines
\providecommand{\tightlist}{%
  \setlength{\itemsep}{0pt}\setlength{\parskip}{0pt}}
\setcounter{secnumdepth}{5}
\usepackage{tikz}
\usepackage{physics}
\usepackage{amsthm}
\usepackage{mathtools}
\usepackage{esint}
\usepackage[ruled,vlined]{algorithm2e}
\theoremstyle{definition}
\newtheorem*{theorem}{Theorem}
\newtheorem*{corollary}{Corollary}
\newtheorem*{remark}{Remark}
\newtheorem*{definition}{Definition}
\newtheorem*{lemma}{Lemma}
\newtheorem*{proposition}{Proposition}
\newtheorem*{example}{Example}
\newcommand{\diag}{\mathop{\mathrm{diag}}}
\newcommand{\Arg}{\mathop{\mathrm{Arg}}}
\newcommand{\hess}{\mathop{\mathrm{Hess}}}
\newcommand\eqae{\mathrel{\overset{\makebox[0pt]{\mbox{\normalfont\tiny\sffamily a.e.}}}{=}}}

\title{Group Representation Revision Notes}
\author{Kexing Ying}

\begin{document}
\maketitle

\section*{Group Representation}

Finding 1-dimensional subrepresentations in \((\mathbb{C}[G], \rho_{\text{reg}})\): 
Denote the 1-dimensional representations of \(G\) by \((\mathbb{C}, \theta)\), 
then, define \(v_\theta := \sum_{g \in G} \overline{\theta(g)} g \in \mathbb{C}[G]\), 
we observe, for all \(h \in G\), \(hv_\theta = \theta(h)v_\theta\). Hence, 
\(v_\theta\) is a shared eigenvector of \(\rho_{\text{reg}}(g)\) with eigenvalue \(\theta(g)\), 
implying \(\langle v_\theta \rangle\) is a 1-dimensional subrepresentation of 
\(\mathbb{C}[G]\) isomorphic to \(\theta\).

\section*{Symmetric and Dihedral Groups}

The dihedral group \(D_n\) is the finite group with the presentation 
\[D_n = \langle x, y \mid x^n = 1 = y^2, y x y^{-1} = 1 \rangle.\]
It has the following important properties:
\begin{itemize}
  \item \(|D_n| = 2n\);
  \item \(D_n\) has \((n  + 3)/2\) conjugacy classes if \(n\) is odd and \((n + 6)/2\) if \(n\) is even 
    (this tells us how many irreducible representations there are);
  \item \((D_n)_{ab} = C_2\) if \(n\) is odd and \((D_n)_{ab} = C_2 \times C_2\) if \(n\) is even;
  \item geometrically, elements of the dihedral group corresponds to rotations and reflections. 
    In particular, for \(n\) even, this includes all reflections along opposite vertices and edges;
  \item \(D_n\) always has the two-dimensional irreducible representation \((\mathbb{C}^2, \rho_{\mathbb{C}^2})\) 
    given by 
    \[\rho_{\mathbb{C}^2}(x) = \begin{pmatrix}
      \cos \frac{2\pi}{n} & -\sin \frac{2\pi}{n} \\
      \sin \frac{2\pi}{n} & \cos \frac{2\pi}{n}
    \end{pmatrix},  \
    \rho_{\mathbb{C}^2}(y) = \begin{pmatrix}
      \cos \frac{4\pi}{n} & \sin \frac{4\pi}{n} \\
      \sin \frac{4\pi}{n} & -\cos \frac{4\pi}{n}
    \end{pmatrix}.\]
  \item The elements which commute with elements of \(\text{End}(\mathbb{C}^2)\) (\((\mathbb{C}^2, \rho_{\mathbb{C}^2})\) 
    is the representation in the above point) are the identity and rotation by \(\pi / 2\);
  \item \(D_3 \simeq S_3\).
\end{itemize}

To find all irreducible representations of \(D_n\), we can construct the following 
homomorphisms (note that these might not be automorphisms), 
\[\phi_k : D_n \to D_n : x^a y^b \mapsto x^{ka}y^b.\]
It is clear that \(\rho_{\mathbb{C}^2} \circ \phi_k\) is an irreducible representation. 
Furthermore, these are non-isomorphic for \(1 \le k < n / 2\) by considering their characters. 
These and the aforementioned one-dimensional representations must be all of the irreducible ones 
by sum of squares. 

Denoting \(\mathbf{n} := \{1, \cdots, n\}\), the symmetric group \(S_n\) is the set of bijections 
between \(\mathbf{n}\) to itself. It has the following properties:
\begin{itemize}
  \item \(|S_n| = n!\);
  \item for all \(\sigma \in S_n\), the conjugacy class \([\sigma]\) contains all elements of \(S_n\) 
    which have the same cycle type as \(\sigma\). Thus, to find the number of conjugacy classes, 
    one count the number of possible cycle types/partitions; 
  \item \(\text{sgn} : S_n \to \{\pm 1\} : \sigma \mapsto \text{sgn}(\sigma)\) is a group homomorphism;
  \item \(\ker \text{sgn} = A_n\)  where \(A_n\) is the alternating group;
  \item \(C_2 \simeq \{\pm 1\} \simeq S_n / A_n \simeq (S_n)_{ab}\); 
  \item hence \(S_n\) has only two one-dimensional representations, namely the trivial and the sign 
    representation;
  \item defining \(P_\sigma \in GL_n(\mathbb{C})\) the permutation matrix corresponding to \(\sigma\),
    \((\mathbb{C}^n, \rho_{\text{perm}})\) where \(\rho_{\text{perm}} : \sigma \mapsto P_{\sigma}\) is 
    a representation known as the permutation representation;
  \item the permutation representation is reducible and, in particular,
    \[(\mathbb{C}^n, \rho_{\text{perm}}) = (\mathbb{C}, \rho_{\text{triv}}) \oplus 
      (\mathbb{C}^{n - 1}, \rho_{\text{refl}}),\]
    where the reflection representation \((\mathbb{C}^{n - 1}, \rho_{\text{refl}})\) is 
    a subrepresentation of the permutation representation on the sub-linear space 
    \(\{x \in \mathbb{C}^n \mid \sum_{i = 1}^n x_i = 0\}\).
\end{itemize}

As \(A_n\) is a subgroup of \(S_n\), we may compute the number of conjugacy classes of 
\(A_n\) from that of \(S_n\). 

\begin{theorem}
  A conjugacy class of \(S_n\) splits into two disjoint conjugacy classes of \(A_n\) if and only 
  if its cycle type consists of distinct odd integers. Otherwise, its simply remains a single 
  conjugacy class in \(A_n\).
\end{theorem}

We have the following surjection \(q : S_4 \to S_3\) such that 
\[q((12)) = (12), q((23)) = (23), q((34)) = (12)\]
which can be restricted such that \(q|_{A_4} : A_4 \to A_3 \simeq C_3\) is a surjection.

\section*{Tensor and Dual}

For arbitrary representations \((V_1, \rho_1), (V_2, \rho_2), (W, \rho_W)\) of \(G\), 
we have the linear isomorphisms 
\[\text{Hom}_G(V_1 \oplus V_2, W) \simeq \text{Hom}_G(V_1, W) \oplus \text{Hom}_G(V_2, W),\]
\[\text{Hom}_G(W, V_1 \oplus V_2) \simeq \text{Hom}_G(W, V_1) \oplus \text{Hom}_G(W, V_2).\]

There is a canonical linear injection 
\[V^* \otimes W \to \text{Hom}(V, W)\] 
which is an isomorphism if \(V\) is finite dimensional.

\section*{Character Theory}

Characters of \(g\) with order \(n\) is the sum (some, possibly all) of \(n\)-th roots of unity.

Inner product on class functions (which contains characters) is defined by 
\[\langle \chi_1, \chi_2 \rangle := \frac{1}{|G|} \sum_{g \in G} \chi_1(g)\overline{\chi_2(g)}.\]
The set of characters of irreducible representations form an orthonormal basis with respect to 
this inner product. Thus, by Maschke's, for all representations \(V = \bigoplus_i V_i^{\oplus n_i}\),
where \(V_i\) are irreducible representations, we can find the multiplicity \(n_i\) with 
\[\langle \chi_V, \chi_{V_i} \rangle = \langle \sum_j n_j \chi_{V_j}, \chi_{V_i}\rangle = 
  \sum_j n_j \delta_{ji} = n_i.\]
To find the last row of the character table, we have the following identity
\[\chi_{V_j}(g) = -(\dim V_j)^{-1} \sum_{i \neq j} \dim V_i \chi_{V_i}(g).\]
To find the size of the conjugacy classes given a character table, we have 
\[\frac{|G|}{|\mathcal{C}_i|} = \sum_{k = 1}^m |\chi_{V_k}(g_i)|^2\]
where \(V_1, \cdots, V_m\) are all the irreducible representations and \(g_i \in \mathcal{C}_i\).

If \((V, \rho)\) is the regular representation of \(G\), it has character \(\chi(e) = |G|\) and 
\(\chi(g) = 0\) for all \(g \neq e\).

A group \(G\) is \textbf{not} simple iff there exists a nontrivial character \(\chi\) such that 
\(\chi(g) = \chi(e)\) for some \(g \neq e\).

For \(g \in G\) of finite order, \(|\chi_V(g)| \le \dim V\) with equality iff 
\(\rho_V(g)\) is a scalar multiple of the identity.

Normal subgroups of \(G\) are precisely the subgroups \(N_J\) of the form
\[N_J := \{n \in G \mid \chi_{V_j}(n) = \chi_{V_j}(e), \forall j \in J\},\]
for \(J \subseteq \{1, \cdots, m\}\).

\section*{Algebra Representations}

If \((V, \rho)\) is a finite dimensional representation of \(G\), then \(\rho(G)\) spans 
\(\text{End}(V)\) if and only if \((V, \rho)\) is irreducible. The reverse direction 
requires semisimple algebras.

Similar to the group case, if \(\rho_V : A \to \text{End}(V)\) is surjective 
for a \(A\)-module \((V, \rho_V)\), the \((V, \rho_V)\) is simple. If \(V\) is 
finite dimensional, the converse also holds.

For modules \(V, W\), \(V \simeq W\) implies \(\chi_V = \chi_W\). The converse is 
true if \(A\) is semisimple.

For \(W \le V\) a submodule, we have \(\chi_V = \chi_W + \chi_{V / W}\). This provides 
a counter example to the converse of the above statement if \(V\) is not semisimple. 

\end{document}

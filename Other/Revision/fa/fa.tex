% Options for packages loaded elsewhere
\PassOptionsToPackage{unicode}{hyperref}
\PassOptionsToPackage{hyphens}{url}
\PassOptionsToPackage{dvipsnames,svgnames*,x11names*}{xcolor}
%
\documentclass[]{article}
\usepackage{lmodern}
\usepackage{amssymb,amsmath}
\usepackage{ifxetex,ifluatex}
\ifnum 0\ifxetex 1\fi\ifluatex 1\fi=0 % if pdftex
  \usepackage[T1]{fontenc}
  \usepackage[utf8]{inputenc}
  \usepackage{textcomp} % provide euro and other symbols
\else % if luatex or xetex
  \usepackage{fontspec}
  \setmainfont{Bitstream Charter}
  % Cabin -- round
  % Bitstream Charter -- almost computer modern
  % \usepackage{unicode-math}
  % \defaultfontfeatures{Scale=MatchLowercase}
  % \defaultfontfeatures[\rmfamily]{Ligatures=TeX,Scale=1}
\fi
% Use upquote if available, for straight quotes in verbatim environments
\IfFileExists{upquote.sty}{\usepackage{upquote}}{}
\IfFileExists{microtype.sty}{% use microtype if available
  \usepackage[]{microtype}
  \UseMicrotypeSet[protrusion]{basicmath} % disable protrusion for tt fonts
}{}
\makeatletter
\@ifundefined{KOMAClassName}{% if non-KOMA class
  \IfFileExists{parskip.sty}{%
    \usepackage{parskip}
  }{% else
    \setlength{\parindent}{0pt}
    \setlength{\parskip}{6pt plus 2pt minus 1pt}}
}{% if KOMA class
  \KOMAoptions{parskip=half}}
\makeatother
\usepackage{xcolor}
\IfFileExists{xurl.sty}{\usepackage{xurl}}{} % add URL line breaks if available
\IfFileExists{bookmark.sty}{\usepackage{bookmark}}{\usepackage{hyperref}}
\hypersetup{
  pdftitle={Functional Analysis Revision Notes},
  pdfauthor={Kexing Ying},
  colorlinks=true,
  linkcolor=Maroon,
  filecolor=Maroon,
  citecolor=Blue,
  urlcolor=red,
  pdfcreator={LaTeX via pandoc}}
\urlstyle{same} % disable monospaced font for URLs
\usepackage[margin = 1.5in]{geometry}
\usepackage{graphicx}
\makeatletter
\def\maxwidth{\ifdim\Gin@nat@width>\linewidth\linewidth\else\Gin@nat@width\fi}
\def\maxheight{\ifdim\Gin@nat@height>\textheight\textheight\else\Gin@nat@height\fi}
\makeatother
% Scale images if necessary, so that they will not overflow the page
% margins by default, and it is still possible to overwrite the defaults
% using explicit options in \includegraphics[width, height, ...]{}
\setkeys{Gin}{width=\maxwidth,height=\maxheight,keepaspectratio}
% Set default figure placement to htbp
\makeatletter
\def\fps@figure{htbp}
\makeatother
\setlength{\emergencystretch}{3em} % prevent overfull lines
\providecommand{\tightlist}{%
  \setlength{\itemsep}{0pt}\setlength{\parskip}{0pt}}
\setcounter{secnumdepth}{5}
\usepackage{tikz}
\usepackage{physics}
\usepackage{amsthm}
\usepackage{mathtools}
\usepackage{esint}
\usepackage[ruled,vlined]{algorithm2e}
\theoremstyle{definition}
\newtheorem*{theorem}{Theorem}
\newtheorem*{corollary}{Corollary}
\newtheorem*{remark}{Remark}
\newtheorem*{definition}{Definition}
\newtheorem*{lemma}{Lemma}
\newtheorem*{proposition}{Proposition}
\newtheorem*{example}{Example}
\newcommand{\diag}{\mathop{\mathrm{diag}}}
\newcommand{\Arg}{\mathop{\mathrm{Arg}}}
\newcommand{\hess}{\mathop{\mathrm{Hess}}}
\newcommand\eqae{\mathrel{\overset{\makebox[0pt]{\mbox{\normalfont\tiny\sffamily a.e.}}}{=}}}

\title{Functional Analysis Revision Notes}
\author{Kexing Ying}

\begin{document}
\maketitle

If \(d\) is a metric and \(\eta : \mathbb{R}_+ \to \mathbb{R}_+\) is concave (and increasing?) and 
\(\eta(0) = 0\), then \(\eta \circ d\) is also a metric.

Beware the properties of \textbf{inner product spaces}. Notably: 
\begin{itemize}
  \item (Cauchy-Schwarz) \(|\langle x, y \rangle| \le \|x\|\|y\|\).
  \item (Parallelogram law) \(\|x + y\|^2  + \|x - y\|^2 = 2\|x\|^2 + 2\|y\|^2\).
  \item (Polarization identity) \(\langle x, y \rangle = \frac{1}{4}(\|x + y\|^2 - \|x - y\|^2)\) over \(\mathbb{R}\) 
    and \(\langle x, y \rangle = \frac{1}{4}\sum_{k = 0}^3 i^k\|x + i^ky\|^2\) over \(\mathbb{C}\).
  \item (Ptolemy's inequality) \(\|x - y\|\|z\| + \|y - z\| \|x\| \ge \|x- z\|\|y\|\).
\end{itemize}

\(C[0, 1]^*\) is not separable as \(\{\delta_x \mid x \in [0, 1]\}\) where \(\delta_x : \phi \mapsto \phi(x)\) 
is a uncountable set in \(C[0, 1]^*\), and for any \(\delta_x \neq \delta_y\), there exists 
\(\phi \in C[0, 1]^*\) such that \(\phi(x) = 1, \phi(y) = 0\) implying \(\|\delta_x - \delta_y\| \ge 1\). 

If \(X\) is reflexive, \(X\) is separable iff. \(X^*\) is (reverse direction does not require 
reflexivity. c.f. Prop 3.11 in notes). 

A Banach space with a Schauder basis is separable. The converse is not necessarily true.

For \(p \le q\), \(\ell^p \subseteq \ell^q\).

For finite measure space \(\Omega\), \(p \le q\), \(L^q(\Omega) \subseteq L^p(\Omega)\).

\section*{Banach Fixed Point Theorem}

For operators in the form \(T : u(x) \mapsto \nu(x) + \beta \int \kappa(x, y) \alpha(u(y)) \lambda(\dd y)\), 
to show strict contraction, we use the FTC to observe
\[\alpha(u(y)) - \alpha(v(y)) = \int_0^1 \dv{\tau}\alpha(\tau u(x) + (1 - \tau)v(x)) \dd \tau,\]
should \(|\dv{\alpha}{\tau}|\) be bounded.

\section*{Baire Category and Banach-Steinhaus}

Any proper \textit{closed} linear subspace of a normed space is nowhere dense 
(in fact true in any TVS).

There cannot be a countable Hamel basis as a result.

The space of polynomials is not complete wrt. any norm by considering \(A_n := \langle 1, x, \cdots, x^n\rangle\) 
and Baire's category theorem.

Closed graph theorem useful to show equivalence of norms by considering the 
identity 
\[\text{id} : (X, \|\cdot\|_1) \to (X, \|\cdot\|_2).\]

\section*{Compact Operators}

The following properties are exercieses from sheet 8:
\begin{itemize}
  \item The dual and adjoint of a compact operator is compact.
  \item Bounded, self adjoint operator \(T\) on a Hilbert space is compact if 
    \(T^n\) is compact for some \(n\) \((*)\).
  \item Any bounded sequence \((x_l\)) in a reflexive space has a weakly convergent subsequence.
  \item \textbf{Corollary}. Any bounded sequence \((x_l\)) in a Hilbert space has a weakly convergent subsequence.
  \item Finite rank operators are compact.
  \item \(T\) is compact in a Hilbert space if and only if \(x_n \rightharpoonup x\) implies \(Tx_n \to Tx\). 
    The backwards direction is from lectures while the forward is from the problem sheet.
\end{itemize}

\((*)\) One method of proof is follows. Clearly \(T\) compact implies \(T^n\) is 
compact for any \(n\) and so, \(T^{2^N}\) is compact for sufficiently large \(N\). 
Then, it suffices to show \(T^{2^N}\) is compact implies \(T^{2^{N - 1}}\) is 
compact.

\section*{Weak Convergence}

A sequence \((x_n\)) converges weakly to \(x\) in a Hilbert space if and only if it is bounded and 
\(\langle x_n, e_i \rangle \to \langle x, e_i\rangle\) for all \(i\) where \(\{e_i\}_{i = 1}^\infty\)
is an orthonormal basis for the space.

\end{document}

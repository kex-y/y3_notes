% Options for packages loaded elsewhere
\PassOptionsToPackage{unicode}{hyperref}
\PassOptionsToPackage{hyphens}{url}
\PassOptionsToPackage{dvipsnames,svgnames*,x11names*}{xcolor}
%
\documentclass[]{article}
\usepackage{lmodern}
\usepackage{amssymb,amsmath}
\usepackage{ifxetex,ifluatex}
\ifnum 0\ifxetex 1\fi\ifluatex 1\fi=0 % if pdftex
  \usepackage[T1]{fontenc}
  \usepackage[utf8]{inputenc}
  \usepackage{textcomp} % provide euro and other symbols
\else % if luatex or xetex
  \usepackage{unicode-math}
  \defaultfontfeatures{Scale=MatchLowercase}
  \defaultfontfeatures[\rmfamily]{Ligatures=TeX,Scale=1}
\fi
% Use upquote if available, for straight quotes in verbatim environments
\IfFileExists{upquote.sty}{\usepackage{upquote}}{}
\IfFileExists{microtype.sty}{% use microtype if available
  \usepackage[]{microtype}
  \UseMicrotypeSet[protrusion]{basicmath} % disable protrusion for tt fonts
}{}
\makeatletter
\@ifundefined{KOMAClassName}{% if non-KOMA class
  \IfFileExists{parskip.sty}{%
    \usepackage{parskip}
  }{% else
    \setlength{\parindent}{0pt}
    \setlength{\parskip}{6pt plus 2pt minus 1pt}}
}{% if KOMA class
  \KOMAoptions{parskip=half}}
\makeatother
\usepackage{xcolor}
\IfFileExists{xurl.sty}{\usepackage{xurl}}{} % add URL line breaks if available
\IfFileExists{bookmark.sty}{\usepackage{bookmark}}{\usepackage{hyperref}}
\hypersetup{
  pdftitle={Quantum Mechanics I},
  pdfauthor={Kexing Ying},
  colorlinks=true,
  linkcolor=Maroon,
  filecolor=Maroon,
  citecolor=Blue,
  urlcolor=red,
  pdfcreator={LaTeX via pandoc}}
\urlstyle{same} % disable monospaced font for URLs
\usepackage[margin = 1.5in]{geometry}
\usepackage{graphicx}
\makeatletter
\def\maxwidth{\ifdim\Gin@nat@width>\linewidth\linewidth\else\Gin@nat@width\fi}
\def\maxheight{\ifdim\Gin@nat@height>\textheight\textheight\else\Gin@nat@height\fi}
\makeatother
% Scale images if necessary, so that they will not overflow the page
% margins by default, and it is still possible to overwrite the defaults
% using explicit options in \includegraphics[width, height, ...]{}
\setkeys{Gin}{width=\maxwidth,height=\maxheight,keepaspectratio}
% Set default figure placement to htbp
\makeatletter
\def\fps@figure{htbp}
\makeatother
\setlength{\emergencystretch}{3em} % prevent overfull lines
\providecommand{\tightlist}{%
  \setlength{\itemsep}{0pt}\setlength{\parskip}{0pt}}
\setcounter{secnumdepth}{5}
\usepackage{tikz}
\usepackage{physics}
\usepackage{amsthm}
\usepackage{mathtools}
\usepackage{esint}
\usepackage[ruled,vlined]{algorithm2e}
\theoremstyle{definition}
\newtheorem{theorem}{Theorem}
\newtheorem{definition*}{Definition}
\newtheorem{prop}{Proposition}
\newtheorem{corollary}{Corollary}[theorem]
\newtheorem*{remark}{Remark}
\theoremstyle{definition}
\newtheorem{definition}{Definition}[section]
\newtheorem{lemma}{Lemma}[section]
\newtheorem{proposition}{Proposition}[section]
\newtheorem{example}{Example}[section]
\newcommand{\diag}{\mathop{\mathrm{diag}}}
\newcommand{\Arg}{\mathop{\mathrm{Arg}}}
\newcommand{\hess}{\mathop{\mathrm{Hess}}}

\title{Quantum Mechanics I}
\author{Kexing Ying}
\date{July 24, 2021}

\begin{document}
\maketitle

{
\hypersetup{linkcolor=}
\setcounter{tocdepth}{2}
\tableofcontents
}
\newpage

\section{Classical Mechanics}

In order to later compare quantum mechanics, let us first introduce some 
classical mechanics. 

In classical mechanics, we study classical objects/particles 
which has a mass \(m \in \mathbb{R}\) and a state. In particular, the state 
of the particle is represented by its position, commonly \(r \in \mathbb{R}^3\), 
and its velocity \(v = \dot r \in \mathbb{R}^3\). More conveniently, we can also 
represent the velocity in terms of its momentum \(p = mv\).

We recall Newton's second law which describes how the state of a particle changes 
in time in the presence of external forces. That is, 
\[\dot p = F(r),\]
where \(F\) is the external force depending on \(r\).

As the state of a particle is represented by its position and momentum, visually 
the state of a particle can be represented by a phase-space with a trajectory 
corresponding to \((r(t), p(t))\). 

An another formulation of classical mechanics is Hamilton's formulation. While 
Hamilton's formulation is very powerful, it does not apply to every classical system. 
In particular, Hamilton's formulation requires the system to be conservative. 

\begin{definition}[Conservative]
  A classical system is said to be conservative if 
  \[F(r) = - \nabla V(r),\]
  where \(V\) is the potential given the position.
\end{definition}

\begin{definition}[Hamiltonian Function]
  The Hamiltonian function \(H\) is defined as 
  \[H(p, q) = \frac{p^2}{2m} + V(q),\]
  where \(p^2/2m\) is the kinetic energy and \(V\) the potential.
\end{definition}

Thus, with the definition of conservative in mind, we see that for a one 
dimensional system with position given by \(q \in \mathbb{R}\), we have
\[\dot p = F(q) = - \pdv{V}{q} \text{ and } \dot q = \frac{p}{m}.\]
Writing in terms of the Hamiltonian function, we obtain, 
\[\dot p = - \pdv{H}{q} \text{ and } \dot q = \pdv{H}{p}.\]
These two equations are known as Hamilton's canonical equations and describe 
the motion of a particle in a conservative system. The theory itself is more 
general in which we simply require \(p, q\) to be canonically conjugate 
variables.

\begin{example}[Free Particle]
  Consider a free particle with \(V(q) = 0\) (thus, \(H = p^2 / 2m\)), we have 
  the canonical equations \(\dot p = 0\) and \(\dot q = p / m\), and thus, 
  \(p(t) = p(0)\) and \(q(t) = q(0) + \frac{p}{m} t\).
\end{example}

\begin{example}[Harmonic Oscillator]
  A harmonic oscillator is described by \(V(q) \propto q^2\). By similar calculation 
  we find \(\ddot q = - \frac{2k}{m} q\) for some \(k\) such that \(V = k q^2\).
\end{example}

\subsection{Poisson Brackets}

As for a particle in classical mechanics, the state is given by its position and 
momentum, any measurable quantity \(A\) is given as a function \(A(p, q)\) 
such that 
\[\dv{A}{t} = \pdv{A}{p}\dot p + \pdv{A}{q} \dot q + \pdv{A}{t}.\]
Substituting the Hamiltonian equations, we have 
\[\dv{A}{t} = - \pdv{A}{p} \pdv{H}{q} + \pdv{A}{q} \pdv{H}{p} + \pdv{A}{t}
  = \pdv{A}{q} \pdv{H}{p} - \pdv{A}{p} \pdv{H}{q} + \pdv{A}{t}.\]
As the first term of this equation is very common, we denote it as \(\{H, A\}\) 
such that 
\[\dv{A}{t} = \{H, A\} + \pdv{A}{t}.\] 
Similarly, for general variables \(F, G\), 
\[\{F, G\} := \sum_{n = 1}^N \pdv{F}{p_n}\pdv{G}{q_n} - \pdv{F}{q_n}\pdv{G}{q_n},\]
and is known as the Poisson bracket of \(F\) and \(G\).

\begin{definition}[Poisson Bracket]
  A Poisson bracket is simply any bracket of functions satisfying 
  \begin{itemize}
    \item \(\{A, A\} = 0\);
    \item \(\{c_1 A + c_2 B, C\} = c_1\{A, C\} + c_2\{B, C\}\);
    \item \(\{A, B\} = -\{B, A\}\).
    \item \(\{c, A\} = 0\) for any constant \(c\);
    \item \(\{AB, C\} = A\{B, C\} + \{A, C\}B\) (Leibniz rule);
    \item \(\{A, \{B, C\}\} + \{B, \{C, A\}\} + \{C, \{A, B\}\} = 0\) (Jacobi identity).
  \end{itemize}
\end{definition}
As an exercise, one may check that the Poisson bracket defined above is indeed 
a Poisson bracket.

\begin{proposition}
  \(\{p, q\} = 1\) and in higher dimensions. \(\{p_i, q_j\} = \delta_{ij}\).
\end{proposition}

\begin{definition}[Canonical Conjugate Variables]
  \(P(p, q), Q(p, q)\) are called canonical conjugate variables if \(\{P, Q\} = 1\). 
  Similarly, for higher dimensions, \(P, Q\) are canonical conjugates if 
  \(\{P_i, Q_j\} = \delta_{ij}\).
\end{definition}

\begin{proposition}
  For any pair of canonical conjugate variables \(P, Q\), we have 
  \[\dot P_j = - \pdv{H}{Q_j} = \{H, P_j\} \text{ and } 
    \dot Q_j = \pdv{H}{P_j} = \{H, Q_j\}.\]
\end{proposition}

\end{document}
